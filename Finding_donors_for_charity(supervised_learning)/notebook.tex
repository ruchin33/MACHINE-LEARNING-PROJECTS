
% Default to the notebook output style

    


% Inherit from the specified cell style.




    
\documentclass[11pt]{article}

    
    
    \usepackage[T1]{fontenc}
    % Nicer default font (+ math font) than Computer Modern for most use cases
    \usepackage{mathpazo}

    % Basic figure setup, for now with no caption control since it's done
    % automatically by Pandoc (which extracts ![](path) syntax from Markdown).
    \usepackage{graphicx}
    % We will generate all images so they have a width \maxwidth. This means
    % that they will get their normal width if they fit onto the page, but
    % are scaled down if they would overflow the margins.
    \makeatletter
    \def\maxwidth{\ifdim\Gin@nat@width>\linewidth\linewidth
    \else\Gin@nat@width\fi}
    \makeatother
    \let\Oldincludegraphics\includegraphics
    % Set max figure width to be 80% of text width, for now hardcoded.
    \renewcommand{\includegraphics}[1]{\Oldincludegraphics[width=.8\maxwidth]{#1}}
    % Ensure that by default, figures have no caption (until we provide a
    % proper Figure object with a Caption API and a way to capture that
    % in the conversion process - todo).
    \usepackage{caption}
    \DeclareCaptionLabelFormat{nolabel}{}
    \captionsetup{labelformat=nolabel}

    \usepackage{adjustbox} % Used to constrain images to a maximum size 
    \usepackage{xcolor} % Allow colors to be defined
    \usepackage{enumerate} % Needed for markdown enumerations to work
    \usepackage{geometry} % Used to adjust the document margins
    \usepackage{amsmath} % Equations
    \usepackage{amssymb} % Equations
    \usepackage{textcomp} % defines textquotesingle
    % Hack from http://tex.stackexchange.com/a/47451/13684:
    \AtBeginDocument{%
        \def\PYZsq{\textquotesingle}% Upright quotes in Pygmentized code
    }
    \usepackage{upquote} % Upright quotes for verbatim code
    \usepackage{eurosym} % defines \euro
    \usepackage[mathletters]{ucs} % Extended unicode (utf-8) support
    \usepackage[utf8x]{inputenc} % Allow utf-8 characters in the tex document
    \usepackage{fancyvrb} % verbatim replacement that allows latex
    \usepackage{grffile} % extends the file name processing of package graphics 
                         % to support a larger range 
    % The hyperref package gives us a pdf with properly built
    % internal navigation ('pdf bookmarks' for the table of contents,
    % internal cross-reference links, web links for URLs, etc.)
    \usepackage{hyperref}
    \usepackage{longtable} % longtable support required by pandoc >1.10
    \usepackage{booktabs}  % table support for pandoc > 1.12.2
    \usepackage[inline]{enumitem} % IRkernel/repr support (it uses the enumerate* environment)
    \usepackage[normalem]{ulem} % ulem is needed to support strikethroughs (\sout)
                                % normalem makes italics be italics, not underlines
    

    
    
    % Colors for the hyperref package
    \definecolor{urlcolor}{rgb}{0,.145,.698}
    \definecolor{linkcolor}{rgb}{.71,0.21,0.01}
    \definecolor{citecolor}{rgb}{.12,.54,.11}

    % ANSI colors
    \definecolor{ansi-black}{HTML}{3E424D}
    \definecolor{ansi-black-intense}{HTML}{282C36}
    \definecolor{ansi-red}{HTML}{E75C58}
    \definecolor{ansi-red-intense}{HTML}{B22B31}
    \definecolor{ansi-green}{HTML}{00A250}
    \definecolor{ansi-green-intense}{HTML}{007427}
    \definecolor{ansi-yellow}{HTML}{DDB62B}
    \definecolor{ansi-yellow-intense}{HTML}{B27D12}
    \definecolor{ansi-blue}{HTML}{208FFB}
    \definecolor{ansi-blue-intense}{HTML}{0065CA}
    \definecolor{ansi-magenta}{HTML}{D160C4}
    \definecolor{ansi-magenta-intense}{HTML}{A03196}
    \definecolor{ansi-cyan}{HTML}{60C6C8}
    \definecolor{ansi-cyan-intense}{HTML}{258F8F}
    \definecolor{ansi-white}{HTML}{C5C1B4}
    \definecolor{ansi-white-intense}{HTML}{A1A6B2}

    % commands and environments needed by pandoc snippets
    % extracted from the output of `pandoc -s`
    \providecommand{\tightlist}{%
      \setlength{\itemsep}{0pt}\setlength{\parskip}{0pt}}
    \DefineVerbatimEnvironment{Highlighting}{Verbatim}{commandchars=\\\{\}}
    % Add ',fontsize=\small' for more characters per line
    \newenvironment{Shaded}{}{}
    \newcommand{\KeywordTok}[1]{\textcolor[rgb]{0.00,0.44,0.13}{\textbf{{#1}}}}
    \newcommand{\DataTypeTok}[1]{\textcolor[rgb]{0.56,0.13,0.00}{{#1}}}
    \newcommand{\DecValTok}[1]{\textcolor[rgb]{0.25,0.63,0.44}{{#1}}}
    \newcommand{\BaseNTok}[1]{\textcolor[rgb]{0.25,0.63,0.44}{{#1}}}
    \newcommand{\FloatTok}[1]{\textcolor[rgb]{0.25,0.63,0.44}{{#1}}}
    \newcommand{\CharTok}[1]{\textcolor[rgb]{0.25,0.44,0.63}{{#1}}}
    \newcommand{\StringTok}[1]{\textcolor[rgb]{0.25,0.44,0.63}{{#1}}}
    \newcommand{\CommentTok}[1]{\textcolor[rgb]{0.38,0.63,0.69}{\textit{{#1}}}}
    \newcommand{\OtherTok}[1]{\textcolor[rgb]{0.00,0.44,0.13}{{#1}}}
    \newcommand{\AlertTok}[1]{\textcolor[rgb]{1.00,0.00,0.00}{\textbf{{#1}}}}
    \newcommand{\FunctionTok}[1]{\textcolor[rgb]{0.02,0.16,0.49}{{#1}}}
    \newcommand{\RegionMarkerTok}[1]{{#1}}
    \newcommand{\ErrorTok}[1]{\textcolor[rgb]{1.00,0.00,0.00}{\textbf{{#1}}}}
    \newcommand{\NormalTok}[1]{{#1}}
    
    % Additional commands for more recent versions of Pandoc
    \newcommand{\ConstantTok}[1]{\textcolor[rgb]{0.53,0.00,0.00}{{#1}}}
    \newcommand{\SpecialCharTok}[1]{\textcolor[rgb]{0.25,0.44,0.63}{{#1}}}
    \newcommand{\VerbatimStringTok}[1]{\textcolor[rgb]{0.25,0.44,0.63}{{#1}}}
    \newcommand{\SpecialStringTok}[1]{\textcolor[rgb]{0.73,0.40,0.53}{{#1}}}
    \newcommand{\ImportTok}[1]{{#1}}
    \newcommand{\DocumentationTok}[1]{\textcolor[rgb]{0.73,0.13,0.13}{\textit{{#1}}}}
    \newcommand{\AnnotationTok}[1]{\textcolor[rgb]{0.38,0.63,0.69}{\textbf{\textit{{#1}}}}}
    \newcommand{\CommentVarTok}[1]{\textcolor[rgb]{0.38,0.63,0.69}{\textbf{\textit{{#1}}}}}
    \newcommand{\VariableTok}[1]{\textcolor[rgb]{0.10,0.09,0.49}{{#1}}}
    \newcommand{\ControlFlowTok}[1]{\textcolor[rgb]{0.00,0.44,0.13}{\textbf{{#1}}}}
    \newcommand{\OperatorTok}[1]{\textcolor[rgb]{0.40,0.40,0.40}{{#1}}}
    \newcommand{\BuiltInTok}[1]{{#1}}
    \newcommand{\ExtensionTok}[1]{{#1}}
    \newcommand{\PreprocessorTok}[1]{\textcolor[rgb]{0.74,0.48,0.00}{{#1}}}
    \newcommand{\AttributeTok}[1]{\textcolor[rgb]{0.49,0.56,0.16}{{#1}}}
    \newcommand{\InformationTok}[1]{\textcolor[rgb]{0.38,0.63,0.69}{\textbf{\textit{{#1}}}}}
    \newcommand{\WarningTok}[1]{\textcolor[rgb]{0.38,0.63,0.69}{\textbf{\textit{{#1}}}}}
    
    
    % Define a nice break command that doesn't care if a line doesn't already
    % exist.
    \def\br{\hspace*{\fill} \\* }
    % Math Jax compatability definitions
    \def\gt{>}
    \def\lt{<}
    % Document parameters
    \title{finding\_donors}
    
    
    

    % Pygments definitions
    
\makeatletter
\def\PY@reset{\let\PY@it=\relax \let\PY@bf=\relax%
    \let\PY@ul=\relax \let\PY@tc=\relax%
    \let\PY@bc=\relax \let\PY@ff=\relax}
\def\PY@tok#1{\csname PY@tok@#1\endcsname}
\def\PY@toks#1+{\ifx\relax#1\empty\else%
    \PY@tok{#1}\expandafter\PY@toks\fi}
\def\PY@do#1{\PY@bc{\PY@tc{\PY@ul{%
    \PY@it{\PY@bf{\PY@ff{#1}}}}}}}
\def\PY#1#2{\PY@reset\PY@toks#1+\relax+\PY@do{#2}}

\expandafter\def\csname PY@tok@w\endcsname{\def\PY@tc##1{\textcolor[rgb]{0.73,0.73,0.73}{##1}}}
\expandafter\def\csname PY@tok@c\endcsname{\let\PY@it=\textit\def\PY@tc##1{\textcolor[rgb]{0.25,0.50,0.50}{##1}}}
\expandafter\def\csname PY@tok@cp\endcsname{\def\PY@tc##1{\textcolor[rgb]{0.74,0.48,0.00}{##1}}}
\expandafter\def\csname PY@tok@k\endcsname{\let\PY@bf=\textbf\def\PY@tc##1{\textcolor[rgb]{0.00,0.50,0.00}{##1}}}
\expandafter\def\csname PY@tok@kp\endcsname{\def\PY@tc##1{\textcolor[rgb]{0.00,0.50,0.00}{##1}}}
\expandafter\def\csname PY@tok@kt\endcsname{\def\PY@tc##1{\textcolor[rgb]{0.69,0.00,0.25}{##1}}}
\expandafter\def\csname PY@tok@o\endcsname{\def\PY@tc##1{\textcolor[rgb]{0.40,0.40,0.40}{##1}}}
\expandafter\def\csname PY@tok@ow\endcsname{\let\PY@bf=\textbf\def\PY@tc##1{\textcolor[rgb]{0.67,0.13,1.00}{##1}}}
\expandafter\def\csname PY@tok@nb\endcsname{\def\PY@tc##1{\textcolor[rgb]{0.00,0.50,0.00}{##1}}}
\expandafter\def\csname PY@tok@nf\endcsname{\def\PY@tc##1{\textcolor[rgb]{0.00,0.00,1.00}{##1}}}
\expandafter\def\csname PY@tok@nc\endcsname{\let\PY@bf=\textbf\def\PY@tc##1{\textcolor[rgb]{0.00,0.00,1.00}{##1}}}
\expandafter\def\csname PY@tok@nn\endcsname{\let\PY@bf=\textbf\def\PY@tc##1{\textcolor[rgb]{0.00,0.00,1.00}{##1}}}
\expandafter\def\csname PY@tok@ne\endcsname{\let\PY@bf=\textbf\def\PY@tc##1{\textcolor[rgb]{0.82,0.25,0.23}{##1}}}
\expandafter\def\csname PY@tok@nv\endcsname{\def\PY@tc##1{\textcolor[rgb]{0.10,0.09,0.49}{##1}}}
\expandafter\def\csname PY@tok@no\endcsname{\def\PY@tc##1{\textcolor[rgb]{0.53,0.00,0.00}{##1}}}
\expandafter\def\csname PY@tok@nl\endcsname{\def\PY@tc##1{\textcolor[rgb]{0.63,0.63,0.00}{##1}}}
\expandafter\def\csname PY@tok@ni\endcsname{\let\PY@bf=\textbf\def\PY@tc##1{\textcolor[rgb]{0.60,0.60,0.60}{##1}}}
\expandafter\def\csname PY@tok@na\endcsname{\def\PY@tc##1{\textcolor[rgb]{0.49,0.56,0.16}{##1}}}
\expandafter\def\csname PY@tok@nt\endcsname{\let\PY@bf=\textbf\def\PY@tc##1{\textcolor[rgb]{0.00,0.50,0.00}{##1}}}
\expandafter\def\csname PY@tok@nd\endcsname{\def\PY@tc##1{\textcolor[rgb]{0.67,0.13,1.00}{##1}}}
\expandafter\def\csname PY@tok@s\endcsname{\def\PY@tc##1{\textcolor[rgb]{0.73,0.13,0.13}{##1}}}
\expandafter\def\csname PY@tok@sd\endcsname{\let\PY@it=\textit\def\PY@tc##1{\textcolor[rgb]{0.73,0.13,0.13}{##1}}}
\expandafter\def\csname PY@tok@si\endcsname{\let\PY@bf=\textbf\def\PY@tc##1{\textcolor[rgb]{0.73,0.40,0.53}{##1}}}
\expandafter\def\csname PY@tok@se\endcsname{\let\PY@bf=\textbf\def\PY@tc##1{\textcolor[rgb]{0.73,0.40,0.13}{##1}}}
\expandafter\def\csname PY@tok@sr\endcsname{\def\PY@tc##1{\textcolor[rgb]{0.73,0.40,0.53}{##1}}}
\expandafter\def\csname PY@tok@ss\endcsname{\def\PY@tc##1{\textcolor[rgb]{0.10,0.09,0.49}{##1}}}
\expandafter\def\csname PY@tok@sx\endcsname{\def\PY@tc##1{\textcolor[rgb]{0.00,0.50,0.00}{##1}}}
\expandafter\def\csname PY@tok@m\endcsname{\def\PY@tc##1{\textcolor[rgb]{0.40,0.40,0.40}{##1}}}
\expandafter\def\csname PY@tok@gh\endcsname{\let\PY@bf=\textbf\def\PY@tc##1{\textcolor[rgb]{0.00,0.00,0.50}{##1}}}
\expandafter\def\csname PY@tok@gu\endcsname{\let\PY@bf=\textbf\def\PY@tc##1{\textcolor[rgb]{0.50,0.00,0.50}{##1}}}
\expandafter\def\csname PY@tok@gd\endcsname{\def\PY@tc##1{\textcolor[rgb]{0.63,0.00,0.00}{##1}}}
\expandafter\def\csname PY@tok@gi\endcsname{\def\PY@tc##1{\textcolor[rgb]{0.00,0.63,0.00}{##1}}}
\expandafter\def\csname PY@tok@gr\endcsname{\def\PY@tc##1{\textcolor[rgb]{1.00,0.00,0.00}{##1}}}
\expandafter\def\csname PY@tok@ge\endcsname{\let\PY@it=\textit}
\expandafter\def\csname PY@tok@gs\endcsname{\let\PY@bf=\textbf}
\expandafter\def\csname PY@tok@gp\endcsname{\let\PY@bf=\textbf\def\PY@tc##1{\textcolor[rgb]{0.00,0.00,0.50}{##1}}}
\expandafter\def\csname PY@tok@go\endcsname{\def\PY@tc##1{\textcolor[rgb]{0.53,0.53,0.53}{##1}}}
\expandafter\def\csname PY@tok@gt\endcsname{\def\PY@tc##1{\textcolor[rgb]{0.00,0.27,0.87}{##1}}}
\expandafter\def\csname PY@tok@err\endcsname{\def\PY@bc##1{\setlength{\fboxsep}{0pt}\fcolorbox[rgb]{1.00,0.00,0.00}{1,1,1}{\strut ##1}}}
\expandafter\def\csname PY@tok@kc\endcsname{\let\PY@bf=\textbf\def\PY@tc##1{\textcolor[rgb]{0.00,0.50,0.00}{##1}}}
\expandafter\def\csname PY@tok@kd\endcsname{\let\PY@bf=\textbf\def\PY@tc##1{\textcolor[rgb]{0.00,0.50,0.00}{##1}}}
\expandafter\def\csname PY@tok@kn\endcsname{\let\PY@bf=\textbf\def\PY@tc##1{\textcolor[rgb]{0.00,0.50,0.00}{##1}}}
\expandafter\def\csname PY@tok@kr\endcsname{\let\PY@bf=\textbf\def\PY@tc##1{\textcolor[rgb]{0.00,0.50,0.00}{##1}}}
\expandafter\def\csname PY@tok@bp\endcsname{\def\PY@tc##1{\textcolor[rgb]{0.00,0.50,0.00}{##1}}}
\expandafter\def\csname PY@tok@fm\endcsname{\def\PY@tc##1{\textcolor[rgb]{0.00,0.00,1.00}{##1}}}
\expandafter\def\csname PY@tok@vc\endcsname{\def\PY@tc##1{\textcolor[rgb]{0.10,0.09,0.49}{##1}}}
\expandafter\def\csname PY@tok@vg\endcsname{\def\PY@tc##1{\textcolor[rgb]{0.10,0.09,0.49}{##1}}}
\expandafter\def\csname PY@tok@vi\endcsname{\def\PY@tc##1{\textcolor[rgb]{0.10,0.09,0.49}{##1}}}
\expandafter\def\csname PY@tok@vm\endcsname{\def\PY@tc##1{\textcolor[rgb]{0.10,0.09,0.49}{##1}}}
\expandafter\def\csname PY@tok@sa\endcsname{\def\PY@tc##1{\textcolor[rgb]{0.73,0.13,0.13}{##1}}}
\expandafter\def\csname PY@tok@sb\endcsname{\def\PY@tc##1{\textcolor[rgb]{0.73,0.13,0.13}{##1}}}
\expandafter\def\csname PY@tok@sc\endcsname{\def\PY@tc##1{\textcolor[rgb]{0.73,0.13,0.13}{##1}}}
\expandafter\def\csname PY@tok@dl\endcsname{\def\PY@tc##1{\textcolor[rgb]{0.73,0.13,0.13}{##1}}}
\expandafter\def\csname PY@tok@s2\endcsname{\def\PY@tc##1{\textcolor[rgb]{0.73,0.13,0.13}{##1}}}
\expandafter\def\csname PY@tok@sh\endcsname{\def\PY@tc##1{\textcolor[rgb]{0.73,0.13,0.13}{##1}}}
\expandafter\def\csname PY@tok@s1\endcsname{\def\PY@tc##1{\textcolor[rgb]{0.73,0.13,0.13}{##1}}}
\expandafter\def\csname PY@tok@mb\endcsname{\def\PY@tc##1{\textcolor[rgb]{0.40,0.40,0.40}{##1}}}
\expandafter\def\csname PY@tok@mf\endcsname{\def\PY@tc##1{\textcolor[rgb]{0.40,0.40,0.40}{##1}}}
\expandafter\def\csname PY@tok@mh\endcsname{\def\PY@tc##1{\textcolor[rgb]{0.40,0.40,0.40}{##1}}}
\expandafter\def\csname PY@tok@mi\endcsname{\def\PY@tc##1{\textcolor[rgb]{0.40,0.40,0.40}{##1}}}
\expandafter\def\csname PY@tok@il\endcsname{\def\PY@tc##1{\textcolor[rgb]{0.40,0.40,0.40}{##1}}}
\expandafter\def\csname PY@tok@mo\endcsname{\def\PY@tc##1{\textcolor[rgb]{0.40,0.40,0.40}{##1}}}
\expandafter\def\csname PY@tok@ch\endcsname{\let\PY@it=\textit\def\PY@tc##1{\textcolor[rgb]{0.25,0.50,0.50}{##1}}}
\expandafter\def\csname PY@tok@cm\endcsname{\let\PY@it=\textit\def\PY@tc##1{\textcolor[rgb]{0.25,0.50,0.50}{##1}}}
\expandafter\def\csname PY@tok@cpf\endcsname{\let\PY@it=\textit\def\PY@tc##1{\textcolor[rgb]{0.25,0.50,0.50}{##1}}}
\expandafter\def\csname PY@tok@c1\endcsname{\let\PY@it=\textit\def\PY@tc##1{\textcolor[rgb]{0.25,0.50,0.50}{##1}}}
\expandafter\def\csname PY@tok@cs\endcsname{\let\PY@it=\textit\def\PY@tc##1{\textcolor[rgb]{0.25,0.50,0.50}{##1}}}

\def\PYZbs{\char`\\}
\def\PYZus{\char`\_}
\def\PYZob{\char`\{}
\def\PYZcb{\char`\}}
\def\PYZca{\char`\^}
\def\PYZam{\char`\&}
\def\PYZlt{\char`\<}
\def\PYZgt{\char`\>}
\def\PYZsh{\char`\#}
\def\PYZpc{\char`\%}
\def\PYZdl{\char`\$}
\def\PYZhy{\char`\-}
\def\PYZsq{\char`\'}
\def\PYZdq{\char`\"}
\def\PYZti{\char`\~}
% for compatibility with earlier versions
\def\PYZat{@}
\def\PYZlb{[}
\def\PYZrb{]}
\makeatother


    % Exact colors from NB
    \definecolor{incolor}{rgb}{0.0, 0.0, 0.5}
    \definecolor{outcolor}{rgb}{0.545, 0.0, 0.0}



    
    % Prevent overflowing lines due to hard-to-break entities
    \sloppy 
    % Setup hyperref package
    \hypersetup{
      breaklinks=true,  % so long urls are correctly broken across lines
      colorlinks=true,
      urlcolor=urlcolor,
      linkcolor=linkcolor,
      citecolor=citecolor,
      }
    % Slightly bigger margins than the latex defaults
    
    \geometry{verbose,tmargin=1in,bmargin=1in,lmargin=1in,rmargin=1in}
    
    

    \begin{document}
    
    
    \maketitle
    
    

    
    \subsection{Supervised Learning}\label{supervised-learning}

\subsection{\texorpdfstring{Project: Finding Donors for
\emph{CharityML}}{Project: Finding Donors for CharityML}}\label{project-finding-donors-for-charityml}

    \subsection{Getting Started}\label{getting-started}

In this project, we will employ several supervised algorithms of our
choice to accurately model individuals' income using data collected from
the 1994 U.S. Census. We will then choose the best candidate algorithm
from preliminary results and further optimize this algorithm to best
model the data. Your goal with this implementation is to construct a
model that accurately predicts whether an individual makes more than
\$50,000. This sort of task can arise in a non-profit setting, where
organizations survive on donations. Understanding an individual's income
can help a non-profit better understand how large of a donation to
request, or whether or not they should reach out to begin with. While it
can be difficult to determine an individual's general income bracket
directly from public sources, we can (as we will see) infer this value
from other publically available features.

The dataset for this project originates from the
\href{https://archive.ics.uci.edu/ml/datasets/Census+Income}{UCI Machine
Learning Repository}. The datset was donated by Ron Kohavi and Barry
Becker, after being published in the article \emph{"Scaling Up the
Accuracy of Naive-Bayes Classifiers: A Decision-Tree Hybrid"}. You can
find the article by Ron Kohavi
\href{https://www.aaai.org/Papers/KDD/1996/KDD96-033.pdf}{online}. The
data we investigate here consists of small changes to the original
dataset, such as removing the
\texttt{\textquotesingle{}fnlwgt\textquotesingle{}} feature and records
with missing or ill-formatted entries.

    \begin{center}\rule{0.5\linewidth}{\linethickness}\end{center}

\subsection{Exploring the Data}\label{exploring-the-data}

Run the code cell below to load necessary Python libraries and load the
census data. Note that the last column from this dataset,
\texttt{\textquotesingle{}income\textquotesingle{}}, will be our target
label (whether an individual makes more than, or at most, \$50,000
annually). All other columns are features about each individual in the
census database.

    \begin{Verbatim}[commandchars=\\\{\}]
{\color{incolor}In [{\color{incolor}1}]:} \PY{c+c1}{\PYZsh{} Import libraries necessary for this project}
        \PY{k+kn}{import} \PY{n+nn}{numpy} \PY{k}{as} \PY{n+nn}{np}
        \PY{k+kn}{import} \PY{n+nn}{pandas} \PY{k}{as} \PY{n+nn}{pd}
        \PY{k+kn}{from} \PY{n+nn}{time} \PY{k}{import} \PY{n}{time}
        \PY{k+kn}{from} \PY{n+nn}{IPython}\PY{n+nn}{.}\PY{n+nn}{display} \PY{k}{import} \PY{n}{display} \PY{c+c1}{\PYZsh{} Allows the use of display() for DataFrames}
        
        \PY{c+c1}{\PYZsh{} Import supplementary visualization code visuals.py}
        \PY{k+kn}{import} \PY{n+nn}{visuals} \PY{k}{as} \PY{n+nn}{vs}
        
        \PY{c+c1}{\PYZsh{} Pretty display for notebooks}
        \PY{o}{\PYZpc{}}\PY{k}{matplotlib} inline
        
        \PY{c+c1}{\PYZsh{} Load the Census dataset}
        \PY{n}{data} \PY{o}{=} \PY{n}{pd}\PY{o}{.}\PY{n}{read\PYZus{}csv}\PY{p}{(}\PY{l+s+s2}{\PYZdq{}}\PY{l+s+s2}{census.csv}\PY{l+s+s2}{\PYZdq{}}\PY{p}{)}
        
        \PY{c+c1}{\PYZsh{} Success \PYZhy{} Display the first record}
        \PY{n}{display}\PY{p}{(}\PY{n}{data}\PY{o}{.}\PY{n}{head}\PY{p}{(}\PY{n}{n}\PY{o}{=}\PY{l+m+mi}{1}\PY{p}{)}\PY{p}{)}
\end{Verbatim}


    
    \begin{verbatim}
   age   workclass education_level  education-num  marital-status  \
0   39   State-gov       Bachelors           13.0   Never-married   

      occupation    relationship    race    sex  capital-gain  capital-loss  \
0   Adm-clerical   Not-in-family   White   Male        2174.0           0.0   

   hours-per-week  native-country income  
0            40.0   United-States  <=50K  
    \end{verbatim}

    
    \subsubsection{Implementation: Data
Exploration}\label{implementation-data-exploration}

A cursory investigation of the dataset will determine how many
individuals fit into either group, and will tell us about the percentage
of these individuals making more than \$50,000.

    \begin{Verbatim}[commandchars=\\\{\}]
{\color{incolor}In [{\color{incolor}2}]:} \PY{c+c1}{\PYZsh{} TODO: Total number of records}
        \PY{n}{n\PYZus{}records} \PY{o}{=} \PY{n}{data}\PY{o}{.}\PY{n}{shape}\PY{p}{[}\PY{l+m+mi}{0}\PY{p}{]}
        
        \PY{c+c1}{\PYZsh{} TODO: Number of records where individual\PYZsq{}s income is more than \PYZdl{}50,000}
        \PY{n}{n\PYZus{}greater\PYZus{}50k} \PY{o}{=} \PY{n}{np}\PY{o}{.}\PY{n}{sum}\PY{p}{(}\PY{n}{data}\PY{p}{[}\PY{l+s+s1}{\PYZsq{}}\PY{l+s+s1}{income}\PY{l+s+s1}{\PYZsq{}}\PY{p}{]} \PY{o}{==} \PY{l+s+s1}{\PYZsq{}}\PY{l+s+s1}{\PYZgt{}50K}\PY{l+s+s1}{\PYZsq{}}\PY{p}{)}
        
        \PY{c+c1}{\PYZsh{} TODO: Number of records where individual\PYZsq{}s income is at most \PYZdl{}50,000}
        \PY{n}{n\PYZus{}at\PYZus{}most\PYZus{}50k} \PY{o}{=} \PY{n}{np}\PY{o}{.}\PY{n}{sum}\PY{p}{(}\PY{n}{data}\PY{p}{[}\PY{l+s+s1}{\PYZsq{}}\PY{l+s+s1}{income}\PY{l+s+s1}{\PYZsq{}}\PY{p}{]} \PY{o}{==} \PY{l+s+s1}{\PYZsq{}}\PY{l+s+s1}{\PYZlt{}=50K}\PY{l+s+s1}{\PYZsq{}}\PY{p}{)}
        
        \PY{c+c1}{\PYZsh{} TODO: Percentage of individuals whose income is more than \PYZdl{}50,000}
        \PY{n}{greater\PYZus{}percent} \PY{o}{=} \PY{l+m+mi}{100} \PY{o}{*} \PY{p}{(}\PY{n}{n\PYZus{}greater\PYZus{}50k}\PY{o}{/}\PY{p}{(}\PY{n}{n\PYZus{}greater\PYZus{}50k}\PY{o}{+}\PY{n}{n\PYZus{}at\PYZus{}most\PYZus{}50k}\PY{p}{)}\PY{p}{)}
        
        \PY{c+c1}{\PYZsh{} Print the results}
        \PY{n+nb}{print}\PY{p}{(}\PY{l+s+s2}{\PYZdq{}}\PY{l+s+s2}{Total number of records: }\PY{l+s+si}{\PYZob{}\PYZcb{}}\PY{l+s+s2}{\PYZdq{}}\PY{o}{.}\PY{n}{format}\PY{p}{(}\PY{n}{n\PYZus{}records}\PY{p}{)}\PY{p}{)}
        \PY{n+nb}{print}\PY{p}{(}\PY{l+s+s2}{\PYZdq{}}\PY{l+s+s2}{Individuals making more than \PYZdl{}50,000: }\PY{l+s+si}{\PYZob{}\PYZcb{}}\PY{l+s+s2}{\PYZdq{}}\PY{o}{.}\PY{n}{format}\PY{p}{(}\PY{n}{n\PYZus{}greater\PYZus{}50k}\PY{p}{)}\PY{p}{)}
        \PY{n+nb}{print}\PY{p}{(}\PY{l+s+s2}{\PYZdq{}}\PY{l+s+s2}{Individuals making at most \PYZdl{}50,000: }\PY{l+s+si}{\PYZob{}\PYZcb{}}\PY{l+s+s2}{\PYZdq{}}\PY{o}{.}\PY{n}{format}\PY{p}{(}\PY{n}{n\PYZus{}at\PYZus{}most\PYZus{}50k}\PY{p}{)}\PY{p}{)}
        \PY{n+nb}{print}\PY{p}{(}\PY{l+s+s2}{\PYZdq{}}\PY{l+s+s2}{Percentage of individuals making more than \PYZdl{}50,000: }\PY{l+s+si}{\PYZob{}\PYZcb{}}\PY{l+s+s2}{\PYZpc{}}\PY{l+s+s2}{\PYZdq{}}\PY{o}{.}\PY{n}{format}\PY{p}{(}\PY{n}{greater\PYZus{}percent}\PY{p}{)}\PY{p}{)}
\end{Verbatim}


    \begin{Verbatim}[commandchars=\\\{\}]
Total number of records: 45222
Individuals making more than \$50,000: 11208
Individuals making at most \$50,000: 34014
Percentage of individuals making more than \$50,000: 24.78439697492371\%

    \end{Verbatim}

    ** Featureset Exploration **

\begin{itemize}
\tightlist
\item
  \textbf{age}: continuous.
\item
  \textbf{workclass}: Private, Self-emp-not-inc, Self-emp-inc,
  Federal-gov, Local-gov, State-gov, Without-pay, Never-worked.
\item
  \textbf{education}: Bachelors, Some-college, 11th, HS-grad,
  Prof-school, Assoc-acdm, Assoc-voc, 9th, 7th-8th, 12th, Masters,
  1st-4th, 10th, Doctorate, 5th-6th, Preschool.
\item
  \textbf{education-num}: continuous.
\item
  \textbf{marital-status}: Married-civ-spouse, Divorced, Never-married,
  Separated, Widowed, Married-spouse-absent, Married-AF-spouse.
\item
  \textbf{occupation}: Tech-support, Craft-repair, Other-service, Sales,
  Exec-managerial, Prof-specialty, Handlers-cleaners, Machine-op-inspct,
  Adm-clerical, Farming-fishing, Transport-moving, Priv-house-serv,
  Protective-serv, Armed-Forces.
\item
  \textbf{relationship}: Wife, Own-child, Husband, Not-in-family,
  Other-relative, Unmarried.
\item
  \textbf{race}: Black, White, Asian-Pac-Islander, Amer-Indian-Eskimo,
  Other.
\item
  \textbf{sex}: Female, Male.
\item
  \textbf{capital-gain}: continuous.
\item
  \textbf{capital-loss}: continuous.
\item
  \textbf{hours-per-week}: continuous.
\item
  \textbf{native-country}: United-States, Cambodia, England,
  Puerto-Rico, Canada, Germany, Outlying-US(Guam-USVI-etc), India,
  Japan, Greece, South, China, Cuba, Iran, Honduras, Philippines, Italy,
  Poland, Jamaica, Vietnam, Mexico, Portugal, Ireland, France,
  Dominican-Republic, Laos, Ecuador, Taiwan, Haiti, Columbia, Hungary,
  Guatemala, Nicaragua, Scotland, Thailand, Yugoslavia, El-Salvador,
  Trinadad\&Tobago, Peru, Hong, Holand-Netherlands.
\end{itemize}

    \begin{center}\rule{0.5\linewidth}{\linethickness}\end{center}

\subsection{Preparing the Data}\label{preparing-the-data}

Before data can be used as input for machine learning algorithms, it
often must be cleaned, formatted, and restructured --- this is typically
known as \textbf{preprocessing}. Fortunately, for this dataset, there
are no invalid or missing entries we must deal with, however, there are
some qualities about certain features that must be adjusted. This
preprocessing can help tremendously with the outcome and predictive
power of nearly all learning algorithms.

    \subsubsection{Transforming Skewed Continuous
Features}\label{transforming-skewed-continuous-features}

A dataset may sometimes contain at least one feature whose values tend
to lie near a single number, but will also have a non-trivial number of
vastly larger or smaller values than that single number. Algorithms can
be sensitive to such distributions of values and can underperform if the
range is not properly normalized. With the census dataset two features
fit this description: '\texttt{capital-gain\textquotesingle{}} and
\texttt{\textquotesingle{}capital-loss\textquotesingle{}}.

    \begin{Verbatim}[commandchars=\\\{\}]
{\color{incolor}In [{\color{incolor}3}]:} \PY{c+c1}{\PYZsh{} Split the data into features and target label}
        \PY{n}{income\PYZus{}raw} \PY{o}{=} \PY{n}{data}\PY{p}{[}\PY{l+s+s1}{\PYZsq{}}\PY{l+s+s1}{income}\PY{l+s+s1}{\PYZsq{}}\PY{p}{]}
        \PY{n}{features\PYZus{}raw} \PY{o}{=} \PY{n}{data}\PY{o}{.}\PY{n}{drop}\PY{p}{(}\PY{l+s+s1}{\PYZsq{}}\PY{l+s+s1}{income}\PY{l+s+s1}{\PYZsq{}}\PY{p}{,} \PY{n}{axis} \PY{o}{=} \PY{l+m+mi}{1}\PY{p}{)}
        
        \PY{c+c1}{\PYZsh{} Visualize skewed continuous features of original data}
        \PY{n}{vs}\PY{o}{.}\PY{n}{distribution}\PY{p}{(}\PY{n}{data}\PY{p}{)}
\end{Verbatim}


    \begin{center}
    \adjustimage{max size={0.9\linewidth}{0.9\paperheight}}{output_9_0.png}
    \end{center}
    { \hspace*{\fill} \\}
    
    For highly-skewed feature distributions such as
\texttt{\textquotesingle{}capital-gain\textquotesingle{}} and
\texttt{\textquotesingle{}capital-loss\textquotesingle{}}, it is common
practice to apply a logarithmic transformation on the data so that the
very large and very small values do not negatively affect the
performance of a learning algorithm. Using a logarithmic transformation
significantly reduces the range of values caused by outliers. Care must
be taken when applying this transformation however: The logarithm of
\texttt{0} is undefined, so we must translate the values by a small
amount above \texttt{0} to apply the the logarithm successfully.

    \begin{Verbatim}[commandchars=\\\{\}]
{\color{incolor}In [{\color{incolor}4}]:} \PY{c+c1}{\PYZsh{} Log\PYZhy{}transform the skewed features}
        \PY{n}{skewed} \PY{o}{=} \PY{p}{[}\PY{l+s+s1}{\PYZsq{}}\PY{l+s+s1}{capital\PYZhy{}gain}\PY{l+s+s1}{\PYZsq{}}\PY{p}{,} \PY{l+s+s1}{\PYZsq{}}\PY{l+s+s1}{capital\PYZhy{}loss}\PY{l+s+s1}{\PYZsq{}}\PY{p}{]}
        \PY{n}{features\PYZus{}log\PYZus{}transformed} \PY{o}{=} \PY{n}{pd}\PY{o}{.}\PY{n}{DataFrame}\PY{p}{(}\PY{n}{data} \PY{o}{=} \PY{n}{features\PYZus{}raw}\PY{p}{)}
        \PY{n}{features\PYZus{}log\PYZus{}transformed}\PY{p}{[}\PY{n}{skewed}\PY{p}{]} \PY{o}{=} \PY{n}{features\PYZus{}raw}\PY{p}{[}\PY{n}{skewed}\PY{p}{]}\PY{o}{.}\PY{n}{apply}\PY{p}{(}\PY{k}{lambda} \PY{n}{x}\PY{p}{:} \PY{n}{np}\PY{o}{.}\PY{n}{log}\PY{p}{(}\PY{n}{x} \PY{o}{+} \PY{l+m+mi}{1}\PY{p}{)}\PY{p}{)}
        
        \PY{c+c1}{\PYZsh{} Visualize the new log distributions}
        \PY{n}{vs}\PY{o}{.}\PY{n}{distribution}\PY{p}{(}\PY{n}{features\PYZus{}log\PYZus{}transformed}\PY{p}{,} \PY{n}{transformed} \PY{o}{=} \PY{k+kc}{True}\PY{p}{)}
\end{Verbatim}


    \begin{center}
    \adjustimage{max size={0.9\linewidth}{0.9\paperheight}}{output_11_0.png}
    \end{center}
    { \hspace*{\fill} \\}
    
    \subsubsection{Normalizing Numerical
Features}\label{normalizing-numerical-features}

In addition to performing transformations on features that are highly
skewed, it is often good practice to perform some type of scaling on
numerical features. Applying a scaling to the data does not change the
shape of each feature's distribution (such as
\texttt{\textquotesingle{}capital-gain\textquotesingle{}} or
\texttt{\textquotesingle{}capital-loss\textquotesingle{}} above);
however, normalization ensures that each feature is treated equally when
applying supervised learners. Note that once scaling is applied,
observing the data in its raw form will no longer have the same original
meaning, as exampled below.

    \begin{Verbatim}[commandchars=\\\{\}]
{\color{incolor}In [{\color{incolor}5}]:} \PY{c+c1}{\PYZsh{} Import sklearn.preprocessing.StandardScaler}
        \PY{k+kn}{from} \PY{n+nn}{sklearn}\PY{n+nn}{.}\PY{n+nn}{preprocessing} \PY{k}{import} \PY{n}{MinMaxScaler}
        
        \PY{c+c1}{\PYZsh{} Initialize a scaler, then apply it to the features}
        \PY{n}{scaler} \PY{o}{=} \PY{n}{MinMaxScaler}\PY{p}{(}\PY{p}{)} \PY{c+c1}{\PYZsh{} default=(0, 1)}
        \PY{n}{numerical} \PY{o}{=} \PY{p}{[}\PY{l+s+s1}{\PYZsq{}}\PY{l+s+s1}{age}\PY{l+s+s1}{\PYZsq{}}\PY{p}{,} \PY{l+s+s1}{\PYZsq{}}\PY{l+s+s1}{education\PYZhy{}num}\PY{l+s+s1}{\PYZsq{}}\PY{p}{,} \PY{l+s+s1}{\PYZsq{}}\PY{l+s+s1}{capital\PYZhy{}gain}\PY{l+s+s1}{\PYZsq{}}\PY{p}{,} \PY{l+s+s1}{\PYZsq{}}\PY{l+s+s1}{capital\PYZhy{}loss}\PY{l+s+s1}{\PYZsq{}}\PY{p}{,} \PY{l+s+s1}{\PYZsq{}}\PY{l+s+s1}{hours\PYZhy{}per\PYZhy{}week}\PY{l+s+s1}{\PYZsq{}}\PY{p}{]}
        
        \PY{n}{features\PYZus{}log\PYZus{}minmax\PYZus{}transform} \PY{o}{=} \PY{n}{pd}\PY{o}{.}\PY{n}{DataFrame}\PY{p}{(}\PY{n}{data} \PY{o}{=} \PY{n}{features\PYZus{}log\PYZus{}transformed}\PY{p}{)}
        \PY{n}{features\PYZus{}log\PYZus{}minmax\PYZus{}transform}\PY{p}{[}\PY{n}{numerical}\PY{p}{]} \PY{o}{=} \PY{n}{scaler}\PY{o}{.}\PY{n}{fit\PYZus{}transform}\PY{p}{(}\PY{n}{features\PYZus{}log\PYZus{}transformed}\PY{p}{[}\PY{n}{numerical}\PY{p}{]}\PY{p}{)}
        
        \PY{c+c1}{\PYZsh{} Show an example of a record with scaling applied}
        \PY{n}{display}\PY{p}{(}\PY{n}{features\PYZus{}log\PYZus{}minmax\PYZus{}transform}\PY{o}{.}\PY{n}{head}\PY{p}{(}\PY{n}{n} \PY{o}{=} \PY{l+m+mi}{5}\PY{p}{)}\PY{p}{)}
\end{Verbatim}


    
    \begin{verbatim}
        age          workclass education_level  education-num  \
0  0.301370          State-gov       Bachelors       0.800000   
1  0.452055   Self-emp-not-inc       Bachelors       0.800000   
2  0.287671            Private         HS-grad       0.533333   
3  0.493151            Private            11th       0.400000   
4  0.150685            Private       Bachelors       0.800000   

        marital-status          occupation    relationship    race      sex  \
0        Never-married        Adm-clerical   Not-in-family   White     Male   
1   Married-civ-spouse     Exec-managerial         Husband   White     Male   
2             Divorced   Handlers-cleaners   Not-in-family   White     Male   
3   Married-civ-spouse   Handlers-cleaners         Husband   Black     Male   
4   Married-civ-spouse      Prof-specialty            Wife   Black   Female   

   capital-gain  capital-loss  hours-per-week  native-country  
0      0.667492           0.0        0.397959   United-States  
1      0.000000           0.0        0.122449   United-States  
2      0.000000           0.0        0.397959   United-States  
3      0.000000           0.0        0.397959   United-States  
4      0.000000           0.0        0.397959            Cuba  
    \end{verbatim}

    
    \subsubsection{Implementation: Data
Preprocessing}\label{implementation-data-preprocessing}

From the table in \textbf{Exploring the Data} above, we can see there
are several features for each record that are non-numeric. Typically,
learning algorithms expect input to be numeric, which requires that
non-numeric features (called \emph{categorical variables}) be converted.
One popular way to convert categorical variables is by using the
\textbf{one-hot encoding} scheme. One-hot encoding creates a
\emph{"dummy"} variable for each possible category of each non-numeric
feature. For example, assume \texttt{someFeature} has three possible
entries: \texttt{A}, \texttt{B}, or \texttt{C}. We then encode this
feature into \texttt{someFeature\_A}, \texttt{someFeature\_B} and
\texttt{someFeature\_C}.

~~\textbar{} someFeature \textbar{} \textbar{} someFeature\_A \textbar{}
someFeature\_B \textbar{} someFeature\_C \textbar{}\\
:-: \textbar{} :-: \textbar{} \textbar{} :-: \textbar{} :-: \textbar{}
:-: \textbar{}\\
0 \textbar{} B \textbar{} \textbar{} 0 \textbar{} 1 \textbar{} 0
\textbar{}\\
1 \textbar{} C \textbar{} -\/-\/-\/-\textgreater{} one-hot encode
-\/-\/-\/-\textgreater{} \textbar{} 0 \textbar{} 0 \textbar{} 1
\textbar{}\\
2 \textbar{} A \textbar{} \textbar{} 1 \textbar{} 0 \textbar{} 0
\textbar{}

Additionally, as with the non-numeric features, we need to convert the
non-numeric target label,
\texttt{\textquotesingle{}income\textquotesingle{}} to numerical values
for the learning algorithm to work. Since there are only two possible
categories for this label ("\textless{}=50K" and "\textgreater{}50K"),
we can avoid using one-hot encoding and simply encode these two
categories as \texttt{0} and \texttt{1}, respectively.

    \begin{Verbatim}[commandchars=\\\{\}]
{\color{incolor}In [{\color{incolor}6}]:} \PY{c+c1}{\PYZsh{} TODO: One\PYZhy{}hot encode the \PYZsq{}features\PYZus{}log\PYZus{}minmax\PYZus{}transform\PYZsq{} data using pandas.get\PYZus{}dummies()}
        \PY{c+c1}{\PYZsh{}non\PYZus{}numerical = [\PYZsq{}workclass\PYZsq{},\PYZsq{}education\PYZus{}level\PYZsq{},\PYZsq{}marital\PYZhy{}status\PYZsq{},\PYZsq{}occupation\PYZsq{},\PYZsq{}relationship\PYZsq{},\PYZsq{}race\PYZsq{},\PYZsq{}sex\PYZsq{},\PYZsq{}native\PYZhy{}country\PYZsq{}]}
        \PY{n}{features\PYZus{}final} \PY{o}{=} \PY{n}{pd}\PY{o}{.}\PY{n}{get\PYZus{}dummies}\PY{p}{(}\PY{n}{features\PYZus{}log\PYZus{}minmax\PYZus{}transform}\PY{p}{)}
        
        \PY{c+c1}{\PYZsh{} TODO: Encode the \PYZsq{}income\PYZus{}raw\PYZsq{} data to numerical values}
        \PY{n}{income} \PY{o}{=} \PY{p}{(}\PY{n}{income\PYZus{}raw} \PY{o}{==} \PY{l+s+s1}{\PYZsq{}}\PY{l+s+s1}{\PYZgt{}50K}\PY{l+s+s1}{\PYZsq{}}\PY{p}{)}
        \PY{n}{income} \PY{o}{=} \PY{n}{income}\PY{o}{.}\PY{n}{astype}\PY{p}{(}\PY{n}{dtype} \PY{o}{=} \PY{n+nb}{int}\PY{p}{)}
        
        \PY{c+c1}{\PYZsh{} Print the number of features after one\PYZhy{}hot encoding}
        \PY{n}{encoded} \PY{o}{=} \PY{n+nb}{list}\PY{p}{(}\PY{n}{features\PYZus{}final}\PY{o}{.}\PY{n}{columns}\PY{p}{)}
        \PY{n+nb}{print}\PY{p}{(}\PY{l+s+s2}{\PYZdq{}}\PY{l+s+si}{\PYZob{}\PYZcb{}}\PY{l+s+s2}{ total features after one\PYZhy{}hot encoding.}\PY{l+s+s2}{\PYZdq{}}\PY{o}{.}\PY{n}{format}\PY{p}{(}\PY{n+nb}{len}\PY{p}{(}\PY{n}{encoded}\PY{p}{)}\PY{p}{)}\PY{p}{)}
        
        \PY{c+c1}{\PYZsh{} Uncomment the following line to see the encoded feature names}
        \PY{c+c1}{\PYZsh{} print(encoded)}
\end{Verbatim}


    \begin{Verbatim}[commandchars=\\\{\}]
103 total features after one-hot encoding.

    \end{Verbatim}

    \subsubsection{Shuffle and Split Data}\label{shuffle-and-split-data}

Now all \emph{categorical variables} have been converted into numerical
features, and all numerical features have been normalized. As always, we
will now split the data (both features and their labels) into training
and test sets. 80\% of the data will be used for training and 20\% for
testing.

    \begin{Verbatim}[commandchars=\\\{\}]
{\color{incolor}In [{\color{incolor}7}]:} \PY{c+c1}{\PYZsh{} Import train\PYZus{}test\PYZus{}split}
        \PY{k+kn}{from} \PY{n+nn}{sklearn}\PY{n+nn}{.}\PY{n+nn}{model\PYZus{}selection} \PY{k}{import} \PY{n}{train\PYZus{}test\PYZus{}split}
        
        \PY{c+c1}{\PYZsh{} Split the \PYZsq{}features\PYZsq{} and \PYZsq{}income\PYZsq{} data into training and testing sets}
        \PY{n}{X\PYZus{}train}\PY{p}{,} \PY{n}{X\PYZus{}test}\PY{p}{,} \PY{n}{y\PYZus{}train}\PY{p}{,} \PY{n}{y\PYZus{}test} \PY{o}{=} \PY{n}{train\PYZus{}test\PYZus{}split}\PY{p}{(}\PY{n}{features\PYZus{}final}\PY{p}{,} 
                                                            \PY{n}{income}\PY{p}{,} 
                                                            \PY{n}{test\PYZus{}size} \PY{o}{=} \PY{l+m+mf}{0.2}\PY{p}{,} 
                                                            \PY{n}{random\PYZus{}state} \PY{o}{=} \PY{l+m+mi}{0}\PY{p}{)}
        
        \PY{c+c1}{\PYZsh{} Show the results of the split}
        \PY{n+nb}{print}\PY{p}{(}\PY{l+s+s2}{\PYZdq{}}\PY{l+s+s2}{Training set has }\PY{l+s+si}{\PYZob{}\PYZcb{}}\PY{l+s+s2}{ samples.}\PY{l+s+s2}{\PYZdq{}}\PY{o}{.}\PY{n}{format}\PY{p}{(}\PY{n}{X\PYZus{}train}\PY{o}{.}\PY{n}{shape}\PY{p}{[}\PY{l+m+mi}{0}\PY{p}{]}\PY{p}{)}\PY{p}{)}
        \PY{n+nb}{print}\PY{p}{(}\PY{l+s+s2}{\PYZdq{}}\PY{l+s+s2}{Testing set has }\PY{l+s+si}{\PYZob{}\PYZcb{}}\PY{l+s+s2}{ samples.}\PY{l+s+s2}{\PYZdq{}}\PY{o}{.}\PY{n}{format}\PY{p}{(}\PY{n}{X\PYZus{}test}\PY{o}{.}\PY{n}{shape}\PY{p}{[}\PY{l+m+mi}{0}\PY{p}{]}\PY{p}{)}\PY{p}{)}
\end{Verbatim}


    \begin{Verbatim}[commandchars=\\\{\}]
Training set has 36177 samples.
Testing set has 9045 samples.

    \end{Verbatim}

    \begin{center}\rule{0.5\linewidth}{\linethickness}\end{center}

\subsection{Evaluating Model
Performance}\label{evaluating-model-performance}

In this section, we will investigate four different algorithms, and
determine which is best at modeling the data. Three of these algorithms
will be supervised learners of your choice, and the fourth algorithm is
known as a \emph{naive predictor}.

    \subsubsection{Metrics and the Naive
Predictor}\label{metrics-and-the-naive-predictor}

\emph{CharityML}, equipped with their research, knows individuals that
make more than \$50,000 are most likely to donate to their charity.
Because of this, \emph{CharityML} is particularly interested in
predicting who makes more than \$50,000 accurately. It would seem that
using \textbf{accuracy} as a metric for evaluating a particular model's
performace would be appropriate. Additionally, identifying someone that
\emph{does not} make more than \$50,000 as someone who does would be
detrimental to \emph{CharityML}, since they are looking to find
individuals willing to donate. Therefore, a model's ability to precisely
predict those that make more than \$50,000 is \emph{more important} than
the model's ability to \textbf{recall} those individuals. We can use
\textbf{F-beta score} as a metric that considers both precision and
recall:

\[ F_{\beta} = (1 + \beta^2) \cdot \frac{precision \cdot recall}{\left( \beta^2 \cdot precision \right) + recall} \]

In particular, when \(\beta = 0.5\), more emphasis is placed on
precision. This is called the \textbf{F\(_{0.5}\) score} (or F-score for
simplicity).

Looking at the distribution of classes (those who make at most \$50,000,
and those who make more), it's clear most individuals do not make more
than \$50,000. This can greatly affect \textbf{accuracy}, since we could
simply say \emph{"this person does not make more than \$50,000"} and
generally be right, without ever looking at the data! Making such a
statement would be called \textbf{naive}, since we have not considered
any information to substantiate the claim. It is always important to
consider the \emph{naive prediction} for your data, to help establish a
benchmark for whether a model is performing well. That been said, using
that prediction would be pointless: If we predicted all people made less
than \$50,000, \emph{CharityML} would identify no one as donors.

** Accuracy ** measures how often the classifier makes the correct
prediction. It's the ratio of the number of correct predictions to the
total number of predictions (the number of test data points).

** Precision ** tells us what proportion of messages we classified as
spam, actually were spam. It is a ratio of true positives(words
classified as spam, and which are actually spam) to all positives(all
words classified as spam, irrespective of whether that was the correct
classificatio), in other words it is the ratio of

\texttt{{[}True\ Positives/(True\ Positives\ +\ False\ Positives){]}}

** Recall(sensitivity)** tells us what proportion of messages that
actually were spam were classified by us as spam. It is a ratio of true
positives(words classified as spam, and which are actually spam) to all
the words that were actually spam, in other words it is the ratio of

\texttt{{[}True\ Positives/(True\ Positives\ +\ False\ Negatives){]}}

For classification problems that are skewed in their classification
distributions like in our case, for example if we had a 100 text
messages and only 2 were spam and the rest 98 weren't, accuracy by
itself is not a very good metric. We could classify 90 messages as not
spam(including the 2 that were spam but we classify them as not spam,
hence they would be false negatives) and 10 as spam(all 10 false
positives) and still get a reasonably good accuracy score. For such
cases, precision and recall come in very handy. These two metrics can be
combined to get the F1 score, which is weighted average(harmonic mean)
of the precision and recall scores. This score can range from 0 to 1,
with 1 being the best possible F1 score(we take the harmonic mean as we
are dealing with ratios).

    \subsubsection{Naive Predictor
Performace}\label{naive-predictor-performace}

\begin{itemize}
\tightlist
\item
  If we chose a model that always predicted an individual made more than
  \$50,000, what would that model's accuracy and F-score be on this
  dataset? You must use the code cell below and assign your results to
  \texttt{\textquotesingle{}accuracy\textquotesingle{}} and
  \texttt{\textquotesingle{}fscore\textquotesingle{}} to be used later.
\end{itemize}

** Please note ** that the the purpose of generating a naive predictor
is simply to show what a base model without any intelligence would look
like. In the real world, ideally your base model would be either the
results of a previous model or could be based on a research paper upon
which you are looking to improve. When there is no benchmark model set,
getting a result better than random choice is a place you could start
from.

\begin{itemize}
\tightlist
\item
  When we have a model that always predicts '1' (i.e. the individual
  makes more than 50k) then our model will have no True Negatives(TN) or
  False Negatives(FN) as we are not making any negative('0' value)
  predictions. Therefore our Accuracy in this case becomes the same as
  our Precision(True Positives/(True Positives + False Positives)) as
  every prediction that we have made with value '1' that should have '0'
  becomes a False Positive; therefore our denominator in this case is
  the total number of records we have in total.
\item
  Our Recall score(True Positives/(True Positives + False Negatives)) in
  this setting becomes 1 as we have no False Negatives.
\end{itemize}

    \begin{Verbatim}[commandchars=\\\{\}]
{\color{incolor}In [{\color{incolor}8}]:} \PY{l+s+sd}{\PYZsq{}\PYZsq{}\PYZsq{}}
        \PY{l+s+sd}{TP = np.sum(income) \PYZsh{} Counting the ones as this is the naive case. Note that \PYZsq{}income\PYZsq{} is the \PYZsq{}income\PYZus{}raw\PYZsq{} data }
        \PY{l+s+sd}{encoded to numerical values done in the data preprocessing step.}
        \PY{l+s+sd}{FP = income.count() \PYZhy{} TP \PYZsh{} Specific to the naive case}
        
        \PY{l+s+sd}{TN = 0 \PYZsh{} No predicted negatives in the naive case}
        \PY{l+s+sd}{FN = 0 \PYZsh{} No predicted negatives in the naive case}
        \PY{l+s+sd}{\PYZsq{}\PYZsq{}\PYZsq{}}
        \PY{k+kn}{from} \PY{n+nn}{sklearn}\PY{n+nn}{.}\PY{n+nn}{metrics} \PY{k}{import} \PY{n}{accuracy\PYZus{}score}
        \PY{k+kn}{from} \PY{n+nn}{sklearn}\PY{n+nn}{.}\PY{n+nn}{metrics} \PY{k}{import} \PY{n}{precision\PYZus{}score}
        \PY{k+kn}{from} \PY{n+nn}{sklearn}\PY{n+nn}{.}\PY{n+nn}{metrics} \PY{k}{import} \PY{n}{recall\PYZus{}score}
        
        \PY{c+c1}{\PYZsh{} TODO: Calculate accuracy, precision and recall}
        \PY{n}{income\PYZus{}pred} \PY{o}{=} \PY{n}{np}\PY{o}{.}\PY{n}{ones}\PY{p}{(}\PY{n}{shape}\PY{o}{=}\PY{n}{income}\PY{o}{.}\PY{n}{shape}\PY{p}{)}
        \PY{n}{accuracy} \PY{o}{=} \PY{n}{accuracy\PYZus{}score}\PY{p}{(}\PY{n}{income}\PY{p}{,}\PY{n}{income\PYZus{}pred}\PY{p}{)}
        \PY{n}{precision} \PY{o}{=} \PY{n}{precision\PYZus{}score}\PY{p}{(}\PY{n}{income}\PY{p}{,}\PY{n}{income\PYZus{}pred}\PY{p}{)}
        \PY{n}{recall} \PY{o}{=} \PY{n}{recall\PYZus{}score}\PY{p}{(}\PY{n}{income}\PY{p}{,}\PY{n}{income\PYZus{}pred}\PY{p}{)}
        
        \PY{c+c1}{\PYZsh{} TODO: Calculate F\PYZhy{}score using the formula above for beta = 0.5 and correct values for precision and recall.}
        \PY{n}{beta}\PY{o}{=}\PY{l+m+mf}{0.5}
        \PY{n}{fscore} \PY{o}{=} \PY{p}{(}\PY{p}{(}\PY{l+m+mi}{1}\PY{o}{+}\PY{n}{beta}\PY{o}{*}\PY{n}{beta}\PY{p}{)}\PY{o}{*}\PY{p}{(}\PY{n}{precision}\PY{o}{*}\PY{n}{recall}\PY{p}{)}\PY{p}{)}\PY{o}{/}\PY{p}{(}\PY{p}{(}\PY{n}{beta}\PY{o}{*}\PY{n}{beta}\PY{o}{*}\PY{n}{precision}\PY{p}{)}\PY{o}{+}\PY{n}{recall}\PY{p}{)}
        
        \PY{c+c1}{\PYZsh{} Print the results }
        \PY{n+nb}{print}\PY{p}{(}\PY{l+s+s2}{\PYZdq{}}\PY{l+s+s2}{Naive Predictor: [Accuracy score: }\PY{l+s+si}{\PYZob{}:.4f\PYZcb{}}\PY{l+s+s2}{, F\PYZhy{}score: }\PY{l+s+si}{\PYZob{}:.4f\PYZcb{}}\PY{l+s+s2}{]}\PY{l+s+s2}{\PYZdq{}}\PY{o}{.}\PY{n}{format}\PY{p}{(}\PY{n}{accuracy}\PY{p}{,} \PY{n}{fscore}\PY{p}{)}\PY{p}{)}
\end{Verbatim}


    \begin{Verbatim}[commandchars=\\\{\}]
Naive Predictor: [Accuracy score: 0.2478, F-score: 0.2917]

    \end{Verbatim}

    \subsubsection{Model Application}\label{model-application}

\begin{itemize}
\tightlist
\item
  Which supervised learning modelis suitable for sucha task ?
\end{itemize}

    \section{Answer:}\label{answer}

\subsection{Ensemble Methods(AdaBoost):}\label{ensemble-methodsadaboost}

\begin{itemize}
\tightlist
\item
  Real-world application: AdaBoost is a technique of fitting weak
  learners repeatedly on the data , finally to get an overall better
  classifier. Adaboost algorithm is used commercially where a the data
  is rich. One such example of AdaBoost algorithm is face-detection. The
  adaboost algorithm using weak learners can identify certain features
  of the pixels of the image and detect if that particular area is the
  face or the background of the image.
\item
  Strengths: The main strength of AdaBoost is that it has very less
  hyper parameters to tune(namely 'base\_estimator', 'n\_estimator' and
  'learning\_rate'. Due to this the AdaBoost algorithm is pretty fast as
  compared to other algorithms like neural networks and SVM.
\item
  Weakness: AdaBoost algorithm is highly affected by the quality of data
  that we are working with. If our data has quite some number of
  outliers, then such outliers would affect the performance of the
  adaboos classifier
\item
  Why a good Candidate: Our dataset is large but we have cleaned it to
  quite some extent, and as a result I think that using the adaboost
  algorithm with decision stump would be quick in training and would
  also generate good results on the test data
\end{itemize}

\subsection{Stochastic Gradient Descent Classifier
(SGDC):}\label{stochastic-gradient-descent-classifier-sgdc}

\begin{itemize}
\tightlist
\item
  Real-world application: A perceptron classifier is an example of SGDC
  family of classifiers. A perceptron algorithm is used as a binary
  classifier when the data is linearly separable. When there are more
  than 2 classes to classify we can still use the perceptron classifier
  to classify this multi-class data through the use of one vs rest or
  the one vs one classification technique. Commercially Perceptron
  algorithm is generally used in classification task such as a good or a
  bad review, ratings of a book depending on certain features etc.
\item
  Strengths: The theory of Perceptron algorithm states that if the data
  is linearly seperable the algorithm will find a decision boundary that
  will always separate the clusters of data points. SGDC algorithms are
  quite fast as compared to other algorithms.
\item
  Weakness: Even if the data is linearly separable, but if our learning
  rate is too small, the gradient descent will take very small leaps
  towards the global minimum and the algorithm will take a very long
  time to converge i.e to find a decision boundry. On the other hand if
  the data is not linearly separable and we keep running the perceptron
  algorithm , it will never converge and will keep on going
  indefinitely. In such a case we will have to specify the maximum
  number of iterations after which the perceptron algorithm should stop.
  Also needless to say the main drawback of SGDC algorithms like
  perceptron is that it is inherently a linear classifier and when it
  comes across a non-linear data it will perform very poorly and in such
  a circumstance we should use non-linear or probabilistic classifiers
  SVM, Neural networks or Naive Bayes respectively.
\item
  Why a good Candidate: Being a binary classification problem we can
  think of a perceptron algorithm to classify the data well by finding a
  linear decision boundary, obviously considering that the data is
  linearly separable.
\end{itemize}

\subsection{SVM:}\label{svm}

\begin{itemize}
\tightlist
\item
  Real-world application: SVM are a class of algorithms that work on the
  principle of Langrange optimization. A simple SVM with a linear kernel
  is a linear classifier, however as we change the kernel from linear to
  polynomial with n degrees or rbf, the SVM classifier becomes a
  nonlinear classifier. SVM algorithms are used in a plethora of
  applications but is most commonly used for text and hyper-text
  categorization(For example: Classification of news articles as
  "sports" or "science", or classification of web-pages as personal or
  others, or to find potential clients for a long term deposit account
  at a bank depending on various features like salary, loans, education
  etc.\\
\item
  Strengths: An SVM algorithm with a non-linear kernel almost always
  performs better or atleast the same than its linear counter-parts and
  also the linear SGDC algorithms. of the There are two main strengths
  SVM algorithm and they are as follows:

  \begin{itemize}
  \tightlist
  \item
    The SVM algorithm decides its spearating boundry by maximizing the
    separating boundary between the points closest to the boundary(These
    points are also known as the support vectors). Due to the fact that
    the SVM algorithm considers only a few points(support vectors) from
    the vast numbers of possible data points to generate its decision
    boundary, the algorithm works very well and provides good results
    even when the amount of data available is less. Thus SVM algorithm
    works well with less data.
  \item
    Due to the kernel trick the limitation of linear classifiers is
    overcome as we can now model SVM as a non-linear classifier. And
    this makes sense because the data that we see in the real world is
    almost never linearly spearable and during such a time a nonliner
    classifier like SVM comes in handy and gives better results as
    compared to its linear counter-parts.
  \end{itemize}
\item
  Weakness: The SVM algorithm has the following weaknesses:

  \begin{itemize}
  \tightlist
  \item
    When the data set is large the SVM algorithm takes a conclusively
    longer time than other algorithms like SGDC or AdaBoost with
    decision trees.
  \item
    Also if we are not careful in tuning the hyper-parameters or even
    the parameters SVM tends to overfit the data when the data is highly
    overlapping.
  \end{itemize}
\item
  Why a good Candidate: Since this is a classification problem with only
  two class, a nonliner SVM with properly tuned hyper parameters may
  perform well on the classification.
\end{itemize}

    \subsubsection{Implementation - Creating a Training and Predicting
Pipeline}\label{implementation---creating-a-training-and-predicting-pipeline}

To properly evaluate the performance of each model you've chosen, it's
important that we create a training and predicting pipeline that allows
us to quickly and effectively train models using various sizes of
training data and perform predictions on the testing data. Our
implementation here will be used in the following section.

    \begin{Verbatim}[commandchars=\\\{\}]
{\color{incolor}In [{\color{incolor}9}]:} \PY{c+c1}{\PYZsh{} TODO: Import two metrics from sklearn \PYZhy{} fbeta\PYZus{}score and accuracy\PYZus{}score}
        \PY{k+kn}{from} \PY{n+nn}{sklearn}\PY{n+nn}{.}\PY{n+nn}{metrics} \PY{k}{import} \PY{n}{fbeta\PYZus{}score}\PY{p}{,} \PY{n}{accuracy\PYZus{}score}
        
        \PY{k}{def} \PY{n+nf}{train\PYZus{}predict}\PY{p}{(}\PY{n}{learner}\PY{p}{,} \PY{n}{sample\PYZus{}size}\PY{p}{,} \PY{n}{X\PYZus{}train}\PY{p}{,} \PY{n}{y\PYZus{}train}\PY{p}{,} \PY{n}{X\PYZus{}test}\PY{p}{,} \PY{n}{y\PYZus{}test}\PY{p}{)}\PY{p}{:} 
            \PY{l+s+sd}{\PYZsq{}\PYZsq{}\PYZsq{}}
        \PY{l+s+sd}{    inputs:}
        \PY{l+s+sd}{       \PYZhy{} learner: the learning algorithm to be trained and predicted on}
        \PY{l+s+sd}{       \PYZhy{} sample\PYZus{}size: the size of samples (number) to be drawn from training set}
        \PY{l+s+sd}{       \PYZhy{} X\PYZus{}train: features training set}
        \PY{l+s+sd}{       \PYZhy{} y\PYZus{}train: income training set}
        \PY{l+s+sd}{       \PYZhy{} X\PYZus{}test: features testing set}
        \PY{l+s+sd}{       \PYZhy{} y\PYZus{}test: income testing set}
        \PY{l+s+sd}{    \PYZsq{}\PYZsq{}\PYZsq{}}
            
            \PY{n}{results} \PY{o}{=} \PY{p}{\PYZob{}}\PY{p}{\PYZcb{}}
            
            \PY{c+c1}{\PYZsh{} TODO: Fit the learner to the training data using slicing with \PYZsq{}sample\PYZus{}size\PYZsq{} using .fit(training\PYZus{}features[:], training\PYZus{}labels[:])}
            \PY{n}{start} \PY{o}{=} \PY{n}{time}\PY{p}{(}\PY{p}{)} \PY{c+c1}{\PYZsh{} Get start time}
            \PY{n}{learner} \PY{o}{=} \PY{n}{learner}\PY{o}{.}\PY{n}{fit}\PY{p}{(}\PY{n}{X\PYZus{}train}\PY{p}{[}\PY{l+m+mi}{0}\PY{p}{:}\PY{n}{sample\PYZus{}size}\PY{p}{]}\PY{p}{,}\PY{n}{y\PYZus{}train}\PY{p}{[}\PY{l+m+mi}{0}\PY{p}{:}\PY{n}{sample\PYZus{}size}\PY{p}{]}\PY{p}{)}
            \PY{n}{end} \PY{o}{=} \PY{n}{time}\PY{p}{(}\PY{p}{)} \PY{c+c1}{\PYZsh{} Get end time}
            
            \PY{c+c1}{\PYZsh{} TODO: Calculate the training time}
            \PY{n}{results}\PY{p}{[}\PY{l+s+s1}{\PYZsq{}}\PY{l+s+s1}{train\PYZus{}time}\PY{l+s+s1}{\PYZsq{}}\PY{p}{]} \PY{o}{=} \PY{n}{end}\PY{o}{\PYZhy{}}\PY{n}{start}
                
            \PY{c+c1}{\PYZsh{} TODO: Get the predictions on the test set(X\PYZus{}test),}
            \PY{c+c1}{\PYZsh{}       then get predictions on the first 300 training samples(X\PYZus{}train) using .predict()}
            \PY{n}{start} \PY{o}{=} \PY{n}{time}\PY{p}{(}\PY{p}{)} \PY{c+c1}{\PYZsh{} Get start time}
            \PY{n}{predictions\PYZus{}test} \PY{o}{=} \PY{n}{learner}\PY{o}{.}\PY{n}{predict}\PY{p}{(}\PY{n}{X\PYZus{}test}\PY{p}{)}
            \PY{n}{predictions\PYZus{}train} \PY{o}{=} \PY{n}{learner}\PY{o}{.}\PY{n}{predict}\PY{p}{(}\PY{n}{X\PYZus{}train}\PY{p}{[}\PY{l+m+mi}{0}\PY{p}{:}\PY{n}{sample\PYZus{}size}\PY{p}{]}\PY{p}{)}
            \PY{n}{end} \PY{o}{=} \PY{n}{time}\PY{p}{(}\PY{p}{)} \PY{c+c1}{\PYZsh{} Get end time}
            
            \PY{c+c1}{\PYZsh{} TODO: Calculate the total prediction time}
            \PY{n}{results}\PY{p}{[}\PY{l+s+s1}{\PYZsq{}}\PY{l+s+s1}{pred\PYZus{}time}\PY{l+s+s1}{\PYZsq{}}\PY{p}{]} \PY{o}{=} \PY{n}{end} \PY{o}{\PYZhy{}} \PY{n}{start}
                    
            \PY{c+c1}{\PYZsh{} TODO: Compute accuracy on the first 300 training samples which is y\PYZus{}train[:300]}
            \PY{n}{results}\PY{p}{[}\PY{l+s+s1}{\PYZsq{}}\PY{l+s+s1}{acc\PYZus{}train}\PY{l+s+s1}{\PYZsq{}}\PY{p}{]} \PY{o}{=} \PY{n}{accuracy\PYZus{}score}\PY{p}{(}\PY{n}{y\PYZus{}train}\PY{p}{[}\PY{l+m+mi}{0}\PY{p}{:}\PY{n}{sample\PYZus{}size}\PY{p}{]}\PY{p}{,}\PY{n}{predictions\PYZus{}train}\PY{p}{)}
                
            \PY{c+c1}{\PYZsh{} TODO: Compute accuracy on test set using accuracy\PYZus{}score()}
            \PY{n}{results}\PY{p}{[}\PY{l+s+s1}{\PYZsq{}}\PY{l+s+s1}{acc\PYZus{}test}\PY{l+s+s1}{\PYZsq{}}\PY{p}{]} \PY{o}{=} \PY{n}{accuracy\PYZus{}score}\PY{p}{(}\PY{n}{y\PYZus{}test}\PY{p}{,}\PY{n}{predictions\PYZus{}test}\PY{p}{)}
            
            \PY{c+c1}{\PYZsh{} TODO: Compute F\PYZhy{}score on the the first 300 training samples using fbeta\PYZus{}score()}
            \PY{n}{results}\PY{p}{[}\PY{l+s+s1}{\PYZsq{}}\PY{l+s+s1}{f\PYZus{}train}\PY{l+s+s1}{\PYZsq{}}\PY{p}{]} \PY{o}{=} \PY{n}{fbeta\PYZus{}score}\PY{p}{(}\PY{n}{y\PYZus{}train}\PY{p}{[}\PY{l+m+mi}{0}\PY{p}{:}\PY{n}{sample\PYZus{}size}\PY{p}{]}\PY{p}{,}\PY{n}{predictions\PYZus{}train}\PY{p}{,}\PY{l+m+mf}{0.5}\PY{p}{)}
                
            \PY{c+c1}{\PYZsh{} TODO: Compute F\PYZhy{}score on the test set which is y\PYZus{}test}
            \PY{n}{results}\PY{p}{[}\PY{l+s+s1}{\PYZsq{}}\PY{l+s+s1}{f\PYZus{}test}\PY{l+s+s1}{\PYZsq{}}\PY{p}{]} \PY{o}{=} \PY{n}{fbeta\PYZus{}score}\PY{p}{(}\PY{n}{y\PYZus{}test}\PY{p}{,}\PY{n}{predictions\PYZus{}test}\PY{p}{,}\PY{l+m+mf}{0.5}\PY{p}{)}
               
            \PY{c+c1}{\PYZsh{} Success}
            \PY{n+nb}{print}\PY{p}{(}\PY{l+s+s2}{\PYZdq{}}\PY{l+s+si}{\PYZob{}\PYZcb{}}\PY{l+s+s2}{ trained on }\PY{l+s+si}{\PYZob{}\PYZcb{}}\PY{l+s+s2}{ samples.}\PY{l+s+s2}{\PYZdq{}}\PY{o}{.}\PY{n}{format}\PY{p}{(}\PY{n}{learner}\PY{o}{.}\PY{n+nv+vm}{\PYZus{}\PYZus{}class\PYZus{}\PYZus{}}\PY{o}{.}\PY{n+nv+vm}{\PYZus{}\PYZus{}name\PYZus{}\PYZus{}}\PY{p}{,} \PY{n}{sample\PYZus{}size}\PY{p}{)}\PY{p}{)}
                
            \PY{c+c1}{\PYZsh{} Return the results}
            \PY{k}{return} \PY{n}{results}
\end{Verbatim}


    \subsubsection{Implementation: Initial Model
Evaluation}\label{implementation-initial-model-evaluation}

In the code cell, you will need to implement the following:

\textbf{Note:} Depending on which algorithms we chose, the following
implementation may take some time to run!

    \begin{Verbatim}[commandchars=\\\{\}]
{\color{incolor}In [{\color{incolor}10}]:} \PY{c+c1}{\PYZsh{} TODO: Import the three supervised learning models from sklearn}
         \PY{k+kn}{from} \PY{n+nn}{sklearn}\PY{n+nn}{.}\PY{n+nn}{ensemble} \PY{k}{import} \PY{n}{AdaBoostClassifier}
         \PY{k+kn}{from} \PY{n+nn}{sklearn}\PY{n+nn}{.}\PY{n+nn}{linear\PYZus{}model} \PY{k}{import} \PY{n}{SGDClassifier}
         \PY{k+kn}{from} \PY{n+nn}{sklearn}\PY{n+nn}{.}\PY{n+nn}{svm} \PY{k}{import} \PY{n}{SVC}
         
         \PY{c+c1}{\PYZsh{} TODO: Initialize the three models}
         \PY{n}{clf\PYZus{}A} \PY{o}{=} \PY{n}{AdaBoostClassifier}\PY{p}{(}\PY{n}{random\PYZus{}state}\PY{o}{=}\PY{l+m+mi}{42}\PY{p}{)}
         \PY{n}{clf\PYZus{}B} \PY{o}{=} \PY{n}{SGDClassifier}\PY{p}{(}\PY{n}{loss} \PY{o}{=}\PY{l+s+s1}{\PYZsq{}}\PY{l+s+s1}{perceptron}\PY{l+s+s1}{\PYZsq{}}\PY{p}{,}\PY{n}{random\PYZus{}state} \PY{o}{=} \PY{l+m+mi}{42}\PY{p}{)}
         \PY{n}{clf\PYZus{}C} \PY{o}{=} \PY{n}{SVC}\PY{p}{(}\PY{n}{random\PYZus{}state} \PY{o}{=} \PY{l+m+mi}{42}\PY{p}{)}
         
         \PY{c+c1}{\PYZsh{} TODO: Calculate the number of samples for 1\PYZpc{}, 10\PYZpc{}, and 100\PYZpc{} of the training data}
         \PY{c+c1}{\PYZsh{} HINT: samples\PYZus{}100 is the entire training set i.e. len(y\PYZus{}train)}
         \PY{c+c1}{\PYZsh{} HINT: samples\PYZus{}10 is 10\PYZpc{} of samples\PYZus{}100 (ensure to set the count of the values to be `int` and not `float`)}
         \PY{c+c1}{\PYZsh{} HINT: samples\PYZus{}1 is 1\PYZpc{} of samples\PYZus{}100 (ensure to set the count of the values to be `int` and not `float`)}
         \PY{n}{samples\PYZus{}100} \PY{o}{=} \PY{n+nb}{int}\PY{p}{(}\PY{n+nb}{len}\PY{p}{(}\PY{n}{y\PYZus{}train}\PY{p}{)}\PY{p}{)}
         \PY{n}{samples\PYZus{}10} \PY{o}{=} \PY{n+nb}{int}\PY{p}{(}\PY{l+m+mf}{0.1} \PY{o}{*} \PY{n+nb}{len}\PY{p}{(}\PY{n}{y\PYZus{}train}\PY{p}{)}\PY{p}{)}
         \PY{n}{samples\PYZus{}1} \PY{o}{=} \PY{n+nb}{int}\PY{p}{(}\PY{l+m+mf}{0.01} \PY{o}{*} \PY{n+nb}{len}\PY{p}{(}\PY{n}{y\PYZus{}train}\PY{p}{)}\PY{p}{)}
         
         \PY{c+c1}{\PYZsh{} Collect results on the learners}
         \PY{n}{results} \PY{o}{=} \PY{p}{\PYZob{}}\PY{p}{\PYZcb{}}
         \PY{k}{for} \PY{n}{clf} \PY{o+ow}{in} \PY{p}{[}\PY{n}{clf\PYZus{}A}\PY{p}{,} \PY{n}{clf\PYZus{}B}\PY{p}{,} \PY{n}{clf\PYZus{}C}\PY{p}{]}\PY{p}{:}
             \PY{n}{clf\PYZus{}name} \PY{o}{=} \PY{n}{clf}\PY{o}{.}\PY{n+nv+vm}{\PYZus{}\PYZus{}class\PYZus{}\PYZus{}}\PY{o}{.}\PY{n+nv+vm}{\PYZus{}\PYZus{}name\PYZus{}\PYZus{}}
             \PY{n}{results}\PY{p}{[}\PY{n}{clf\PYZus{}name}\PY{p}{]} \PY{o}{=} \PY{p}{\PYZob{}}\PY{p}{\PYZcb{}}
             \PY{k}{for} \PY{n}{i}\PY{p}{,} \PY{n}{samples} \PY{o+ow}{in} \PY{n+nb}{enumerate}\PY{p}{(}\PY{p}{[}\PY{n}{samples\PYZus{}1}\PY{p}{,} \PY{n}{samples\PYZus{}10}\PY{p}{,} \PY{n}{samples\PYZus{}100}\PY{p}{]}\PY{p}{)}\PY{p}{:}
                 \PY{n}{results}\PY{p}{[}\PY{n}{clf\PYZus{}name}\PY{p}{]}\PY{p}{[}\PY{n}{i}\PY{p}{]} \PY{o}{=} \PYZbs{}
                 \PY{n}{train\PYZus{}predict}\PY{p}{(}\PY{n}{clf}\PY{p}{,} \PY{n}{samples}\PY{p}{,} \PY{n}{X\PYZus{}train}\PY{p}{,} \PY{n}{y\PYZus{}train}\PY{p}{,} \PY{n}{X\PYZus{}test}\PY{p}{,} \PY{n}{y\PYZus{}test}\PY{p}{)}
         
         \PY{c+c1}{\PYZsh{} Run metrics visualization for the three supervised learning models chosen}
         \PY{n}{vs}\PY{o}{.}\PY{n}{evaluate}\PY{p}{(}\PY{n}{results}\PY{p}{,} \PY{n}{accuracy}\PY{p}{,} \PY{n}{fscore}\PY{p}{)}
\end{Verbatim}


    \begin{Verbatim}[commandchars=\\\{\}]
AdaBoostClassifier trained on 361 samples.
AdaBoostClassifier trained on 3617 samples.
AdaBoostClassifier trained on 36177 samples.
SGDClassifier trained on 361 samples.
SGDClassifier trained on 3617 samples.

    \end{Verbatim}

    \begin{Verbatim}[commandchars=\\\{\}]
/opt/conda/lib/python3.6/site-packages/sklearn/linear\_model/stochastic\_gradient.py:128: FutureWarning: max\_iter and tol parameters have been added in <class 'sklearn.linear\_model.stochastic\_gradient.SGDClassifier'> in 0.19. If both are left unset, they default to max\_iter=5 and tol=None. If tol is not None, max\_iter defaults to max\_iter=1000. From 0.21, default max\_iter will be 1000, and default tol will be 1e-3.
  "and default tol will be 1e-3." \% type(self), FutureWarning)

    \end{Verbatim}

    \begin{Verbatim}[commandchars=\\\{\}]
SGDClassifier trained on 36177 samples.

    \end{Verbatim}

    \begin{Verbatim}[commandchars=\\\{\}]
/opt/conda/lib/python3.6/site-packages/sklearn/metrics/classification.py:1135: UndefinedMetricWarning: F-score is ill-defined and being set to 0.0 due to no predicted samples.
  'precision', 'predicted', average, warn\_for)

    \end{Verbatim}

    \begin{Verbatim}[commandchars=\\\{\}]
SVC trained on 361 samples.
SVC trained on 3617 samples.
SVC trained on 36177 samples.

    \end{Verbatim}

    \begin{center}
    \adjustimage{max size={0.9\linewidth}{0.9\paperheight}}{output_27_5.png}
    \end{center}
    { \hspace*{\fill} \\}
    
    \begin{center}\rule{0.5\linewidth}{\linethickness}\end{center}

\subsection{Improving Results}\label{improving-results}

In this final section, we will choose from the three supervised learning
models the \emph{best} model to use on the student data. We will then
perform a grid search optimization for the model over the entire
training set (\texttt{X\_train} and \texttt{y\_train}) by tuning at
least one parameter to improve upon the untuned model's F-score.

    \subsubsection{Choosing the Best Model}\label{choosing-the-best-model}

\begin{itemize}
\tightlist
\item
  Based on the evaluation we performed earlier, we will explain to
  \emph{CharityML} which of the three models we believe to be most
  appropriate for the task of identifying individuals that make more
  than \$50,000.
\end{itemize}

    \textbf{Answer: } * The above graph gives us a great intuition about the
working of the three models that we chose namely, the perceptron
classifier, the SVM classifier and the Adaboost classifier using the
decision stump. As mentioned earlier the SVM classifier takes the
maximum time when 100\% of the training data is used.The green bar in
the training and predicting time figure is manifolds greater than both
the AdaBoost and the SGDC classifiers. Also if we look at the f-1 score
for training and testing data when 100\% of the dataset is used the
AdaBoost and the SVM classifiers produce almost the same result. However
if we take into consideration the training and prediction time we would
definitely rule out the SVM classifier mainly due to the time is take
for the algorithm to run and also owing to the fact that AdaBoost
produces similar results. Also if we consider all the training and
testing data sizes( 1\%, 10\% and 100\%) the AdaBoost algorithm produces
higher or almost the same f-1 scores as the SVM classifier and as a
result of this I think that the AdaBoost classifier would be the best
suited classifier for identifying individuals that make more than
\$50,000.

\begin{itemize}
\tightlist
\item
  The above mentioned reasons also throw light to the fact that AdaBoost
  algorithm which gave us the highest f-1 scores on all sizes of dataset
  would be better at extracting the hidden attribute importances and
  therby would generalize well on the unseen data. Thus in conclusion
  AdaBoost is the best classifier that we can use in our urrent
  situation.
\end{itemize}

    \subsubsection{Describing the Model in Layman's
Terms}\label{describing-the-model-in-laymans-terms}

\begin{itemize}
\tightlist
\item
  We will explain to \emph{CharityML}, in layman's terms, how the final
  model chosen is supposed to work.
\end{itemize}

    \textbf{Answer: }

\begin{itemize}
\item
  The AdaBoost algorithm works by fitting a sequence of weak learners on
  data that is being modified at every iteration of those fittings of
  weak learners. At the end of the algorithm the predictions are made by
  combining all those weak models through a weighted sum. The process
  takes place as follows:

  \begin{itemize}
  \item
    TRAINING: At the start of the training a weak decision tree
    classifier with depths as low as 1 is used to fit the data in such a
    way that minimum error is produced. After that all the correctly and
    incorrectly classified points are counted and the weights of the
    incorrectly classified points are modified in such a way that the
    total weight of correctly and incorrectly classified points become
    equal(This increase in weight is done to make sure that the next
    weak classifier classifies those incorrectly classified points
    perfectly when the fitting is done again.) The above mentioned
    process continues till there are no misclassified points or till
    maximum iteration is reached. Thus at the end of the training
    session we have a certain number of weak classifiers which we
    combine in the prediction step.
  \item
    COMBINING THE MODELS: As mentioned in the training step above we
    created a certain number of weak classifiers. There is a weight
    associated with every such model. This weight can be any real
    number. For every such model the area classified as positive is
    given a positive value of the corresponding weight and the areas
    classified as negative are given negative value of the corresponding
    weight. After doing this all the models are combined and a weighted
    sum of all the regions is taken. In the end certain areas of the
    final model will have values of weights greater than zero and
    certain areas wile have values of weights less than zero.
  \item
    PREDICTION: Finally any point that belongs to a positive area of the
    final model is classified as positive and any model belonging to the
    negative areas is classified as negative.
  \end{itemize}
\item
  Thus by using many weak models and combining them we made a complex
  model that can classify data points well enough than not only a Naive
  classifier but also better and more efficiently than non-linear
  classifiers like SVM.
\end{itemize}

    \subsubsection{Implementation: Model
Tuning}\label{implementation-model-tuning}

    \begin{Verbatim}[commandchars=\\\{\}]
{\color{incolor}In [{\color{incolor}11}]:} \PY{c+c1}{\PYZsh{} TODO: Import \PYZsq{}GridSearchCV\PYZsq{}, \PYZsq{}make\PYZus{}scorer\PYZsq{}, and any other necessary libraries}
         \PY{k+kn}{from} \PY{n+nn}{sklearn}\PY{n+nn}{.}\PY{n+nn}{metrics} \PY{k}{import} \PY{n}{make\PYZus{}scorer}
         \PY{k+kn}{from} \PY{n+nn}{sklearn}\PY{n+nn}{.}\PY{n+nn}{model\PYZus{}selection} \PY{k}{import} \PY{n}{GridSearchCV}
         \PY{k+kn}{from} \PY{n+nn}{sklearn}\PY{n+nn}{.}\PY{n+nn}{model\PYZus{}selection} \PY{k}{import} \PY{n}{ShuffleSplit}
         
         \PY{c+c1}{\PYZsh{} TODO: Initialize the classifier}
         \PY{n}{clf} \PY{o}{=} \PY{n}{AdaBoostClassifier}\PY{p}{(}\PY{n}{random\PYZus{}state}\PY{o}{=}\PY{l+m+mi}{42}\PY{p}{)}
         
         \PY{c+c1}{\PYZsh{} TODO: Create the parameters list you wish to tune, using a dictionary if needed.}
         \PY{c+c1}{\PYZsh{} HINT: parameters = \PYZob{}\PYZsq{}parameter\PYZus{}1\PYZsq{}: [value1, value2], \PYZsq{}parameter\PYZus{}2\PYZsq{}: [value1, value2]\PYZcb{}}
         \PY{n}{n\PYZus{}estimators} \PY{o}{=} \PY{p}{[}\PY{n+nb}{int}\PY{p}{(}\PY{n}{x}\PY{p}{)} \PY{k}{for} \PY{n}{x} \PY{o+ow}{in} \PY{n}{np}\PY{o}{.}\PY{n}{linspace}\PY{p}{(}\PY{n}{start} \PY{o}{=} \PY{l+m+mi}{100}\PY{p}{,} \PY{n}{stop} \PY{o}{=} \PY{l+m+mi}{500}\PY{p}{,} \PY{n}{num} \PY{o}{=} \PY{l+m+mi}{10}\PY{p}{)}\PY{p}{]}
         \PY{n}{learning\PYZus{}rate}  \PY{o}{=} \PY{n+nb}{list}\PY{p}{(}\PY{n}{np}\PY{o}{.}\PY{n}{arange}\PY{p}{(}\PY{l+m+mf}{0.5}\PY{p}{,}\PY{l+m+mi}{2}\PY{p}{,}\PY{l+m+mf}{0.2}\PY{p}{)}\PY{p}{)}
         
         
         
         \PY{n}{parameters} \PY{o}{=} \PY{p}{\PYZob{}}\PY{l+s+s1}{\PYZsq{}}\PY{l+s+s1}{n\PYZus{}estimators}\PY{l+s+s1}{\PYZsq{}}\PY{p}{:} \PY{n}{n\PYZus{}estimators}\PY{p}{,}\PY{l+s+s1}{\PYZsq{}}\PY{l+s+s1}{learning\PYZus{}rate}\PY{l+s+s1}{\PYZsq{}}\PY{p}{:} \PY{n}{learning\PYZus{}rate}\PY{p}{\PYZcb{}}
         
         
         \PY{c+c1}{\PYZsh{} TODO: Make an fbeta\PYZus{}score scoring object using make\PYZus{}scorer()}
         \PY{n}{scorer} \PY{o}{=} \PY{n}{make\PYZus{}scorer}\PY{p}{(}\PY{n}{fbeta\PYZus{}score}\PY{p}{,}\PY{n}{beta} \PY{o}{=} \PY{l+m+mf}{0.5}\PY{p}{)}
         
         \PY{c+c1}{\PYZsh{} TODO: Perform grid search on the classifier using \PYZsq{}scorer\PYZsq{} as the scoring method using GridSearchCV()}
         \PY{n}{grid\PYZus{}obj} \PY{o}{=} \PY{n}{GridSearchCV}\PY{p}{(}\PY{n}{clf}\PY{p}{,}\PY{n}{parameters}\PY{p}{,}\PY{n}{scoring}\PY{o}{=}\PY{n}{scorer}\PY{p}{)}
         
         \PY{c+c1}{\PYZsh{} TODO: Fit the grid search object to the training data and find the optimal parameters using fit()}
         \PY{n}{grid\PYZus{}fit} \PY{o}{=} \PY{n}{grid\PYZus{}obj}\PY{o}{.}\PY{n}{fit}\PY{p}{(}\PY{n}{X\PYZus{}train}\PY{p}{,}\PY{n}{y\PYZus{}train}\PY{p}{)}
         
         \PY{c+c1}{\PYZsh{} Get the estimator}
         \PY{n}{best\PYZus{}clf} \PY{o}{=} \PY{n}{grid\PYZus{}fit}\PY{o}{.}\PY{n}{best\PYZus{}estimator\PYZus{}}
         
         \PY{c+c1}{\PYZsh{} Make predictions using the unoptimized and model}
         \PY{n}{predictions} \PY{o}{=} \PY{p}{(}\PY{n}{clf}\PY{o}{.}\PY{n}{fit}\PY{p}{(}\PY{n}{X\PYZus{}train}\PY{p}{,} \PY{n}{y\PYZus{}train}\PY{p}{)}\PY{p}{)}\PY{o}{.}\PY{n}{predict}\PY{p}{(}\PY{n}{X\PYZus{}test}\PY{p}{)}
         \PY{n}{best\PYZus{}predictions} \PY{o}{=} \PY{n}{best\PYZus{}clf}\PY{o}{.}\PY{n}{predict}\PY{p}{(}\PY{n}{X\PYZus{}test}\PY{p}{)}
         
         \PY{c+c1}{\PYZsh{} Report the before\PYZhy{}and\PYZhy{}afterscores}
         \PY{n+nb}{print}\PY{p}{(}\PY{l+s+s2}{\PYZdq{}}\PY{l+s+s2}{Unoptimized model}\PY{l+s+se}{\PYZbs{}n}\PY{l+s+s2}{\PYZhy{}\PYZhy{}\PYZhy{}\PYZhy{}\PYZhy{}\PYZhy{}}\PY{l+s+s2}{\PYZdq{}}\PY{p}{)}
         \PY{n+nb}{print}\PY{p}{(}\PY{l+s+s2}{\PYZdq{}}\PY{l+s+s2}{Accuracy score on testing data: }\PY{l+s+si}{\PYZob{}:.4f\PYZcb{}}\PY{l+s+s2}{\PYZdq{}}\PY{o}{.}\PY{n}{format}\PY{p}{(}\PY{n}{accuracy\PYZus{}score}\PY{p}{(}\PY{n}{y\PYZus{}test}\PY{p}{,} \PY{n}{predictions}\PY{p}{)}\PY{p}{)}\PY{p}{)}
         \PY{n+nb}{print}\PY{p}{(}\PY{l+s+s2}{\PYZdq{}}\PY{l+s+s2}{F\PYZhy{}score on testing data: }\PY{l+s+si}{\PYZob{}:.4f\PYZcb{}}\PY{l+s+s2}{\PYZdq{}}\PY{o}{.}\PY{n}{format}\PY{p}{(}\PY{n}{fbeta\PYZus{}score}\PY{p}{(}\PY{n}{y\PYZus{}test}\PY{p}{,} \PY{n}{predictions}\PY{p}{,} \PY{n}{beta} \PY{o}{=} \PY{l+m+mf}{0.5}\PY{p}{)}\PY{p}{)}\PY{p}{)}
         \PY{n+nb}{print}\PY{p}{(}\PY{l+s+s2}{\PYZdq{}}\PY{l+s+se}{\PYZbs{}n}\PY{l+s+s2}{Optimized Model}\PY{l+s+se}{\PYZbs{}n}\PY{l+s+s2}{\PYZhy{}\PYZhy{}\PYZhy{}\PYZhy{}\PYZhy{}\PYZhy{}}\PY{l+s+s2}{\PYZdq{}}\PY{p}{)}
         \PY{n+nb}{print}\PY{p}{(}\PY{l+s+s2}{\PYZdq{}}\PY{l+s+s2}{Final accuracy score on the testing data: }\PY{l+s+si}{\PYZob{}:.4f\PYZcb{}}\PY{l+s+s2}{\PYZdq{}}\PY{o}{.}\PY{n}{format}\PY{p}{(}\PY{n}{accuracy\PYZus{}score}\PY{p}{(}\PY{n}{y\PYZus{}test}\PY{p}{,} \PY{n}{best\PYZus{}predictions}\PY{p}{)}\PY{p}{)}\PY{p}{)}
         \PY{n+nb}{print}\PY{p}{(}\PY{l+s+s2}{\PYZdq{}}\PY{l+s+s2}{Final F\PYZhy{}score on the testing data: }\PY{l+s+si}{\PYZob{}:.4f\PYZcb{}}\PY{l+s+s2}{\PYZdq{}}\PY{o}{.}\PY{n}{format}\PY{p}{(}\PY{n}{fbeta\PYZus{}score}\PY{p}{(}\PY{n}{y\PYZus{}test}\PY{p}{,} \PY{n}{best\PYZus{}predictions}\PY{p}{,} \PY{n}{beta} \PY{o}{=} \PY{l+m+mf}{0.5}\PY{p}{)}\PY{p}{)}\PY{p}{)}
\end{Verbatim}


    \begin{Verbatim}[commandchars=\\\{\}]
Unoptimized model
------
Accuracy score on testing data: 0.8576
F-score on testing data: 0.7246

Optimized Model
------
Final accuracy score on the testing data: 0.8677
Final F-score on the testing data: 0.7457

    \end{Verbatim}

    \subsubsection{Final Model Evaluation}\label{final-model-evaluation}

\begin{itemize}
\tightlist
\item
  What is our optimized model's accuracy and F-score on the testing
  data?
\item
  Are these scores better or worse than the unoptimized model?
\item
  How do the results from our optimized model compare to the naive
  predictor benchmarks we found earlier in ?
\end{itemize}

    \paragraph{Results:}\label{results}

\begin{longtable}[]{@{}cccc@{}}
\toprule
Metric & Naive Model & Unoptimized Model & Optimized
Model\tabularnewline
\midrule
\endhead
Accuracy Score & 0.2478 & 0.8576 & 0.8677\tabularnewline
F-score & 0.2917 & 0.7246 & 0.7457\tabularnewline
\bottomrule
\end{longtable}

    \textbf{Answer: }

\begin{itemize}
\tightlist
\item
  As seen from the table above the f1-scores of the optimized model are
  better than the unoptimized model. Also the naive model is no match
  for the optimized model. The f1-score is quite high than the naive
  model and that is due to the fact that in the naive model we predicted
  everyone to be the doner whereas in the optimized model we created
  complex decision boundaries on the basis of the features with the help
  of AdaBoost classifier which in turn helped us in predicting better on
  the unseen data.
\end{itemize}

    \begin{center}\rule{0.5\linewidth}{\linethickness}\end{center}

\subsection{Feature Importance}\label{feature-importance}

An important task when performing supervised learning on a dataset like
the census data we study here is determining which features provide the
most predictive power. By focusing on the relationship between only a
few crucial features and the target label we simplify our understanding
of the phenomenon, which is most always a useful thing to do. In the
case of this project, that means we wish to identify a small number of
features that most strongly predict whether an individual makes at most
or more than \$50,000.

We will choose a scikit-learn classifier (e.g., adaboost, random
forests) that has a \texttt{feature\_importance\_} attribute, which is a
function that ranks the importance of features according to the chosen
classifier. In the next python cell we will fit this classifier to
training set and use this attribute to determine the top 5 most
important features for the census dataset.

    \subsubsection{Feature Relevance
Observation}\label{feature-relevance-observation}

When \textbf{Exploring the Data}, it was shown there are thirteen
available features for each individual on record in the census data. Of
these thirteen records, the 5 features that we believe to be the most
important for prediction are in the given order:

    \textbf{Answer:} * The following are the top five features that I
consider most important:

\begin{verbatim}
- OCCUPATION(1)    : Occupation comes at the topmost of the list because the income of a person is directly related to what a person does and certain occupations like doctor, lawyers etc. tend to make more money than their counterparts. Also since we are targeting people with salaries greater than 50K the occupation of a person gives us good insights. For example(A person at managerial role will have salaries higher than a person who is in a clerical position) 

- WORK_CLASS(2)    : Work class is another important feature which can give us good insights about a person's salary. For example. A person working for a private sector company is definitely going to earn more than a person who works for a local government. I have onsidered the second most important feature because whatever the occupation of a person is, his salary greatly depends on the work_class he works in.(For eg. A doctor in private hospital is going to earn more than a doctor in public hospital)

- EDUCATION (3)    : Generally people who tend to have a good education also tend to make more money. A person with a Phd is definitely going to make more money than a high-school graduate. Thus education is an important feature in determining a person's salary. This feature is third in the list as I think its not always necessary that a person with good education is successful. We have good good examples of dropouts who have made it better than their graduate counter parts.

- HOURS_PER_WEEK(4): The more a person works the more is his or hers' income and hece the number of hours a person works in  a week gives us a meaningful insight of his or her's salary. However this feature is fourth in the list because sometimes a person may be working 40 hours a week on minimum wages while on the other hand we may have a person working 20 hours part time with thrice the hourly minimum wage and in such a case the salary of the person working part-time is more than his full time counterpart. Hence the hours_per_week feature is subjective and it depends more on the occupation of the person.

- CAPITAL_GAIN  (5): Capital gain indicates a profit from a sale of property or an investment. Thus we can reasonably say that higer the capital gain higher is the income of that person and he is more likely to make an investment. I have ranked this feature at the bottom of the list because I think that a person other than a reasl estate businessman is less likely to sell his property or investments and hence capital gain is not an event that occurs as usually as other events.
\end{verbatim}

    \subsubsection{Implementation - Extracting Feature
Importance}\label{implementation---extracting-feature-importance}

\begin{itemize}
\tightlist
\item
  We will choose AdaBoostClassifier as it has a feature importance
  attribute available for it. This attribute is a function that ranks
  the importance of each feature when making predictions based on the
  chosen algorithm.
\end{itemize}

    \begin{Verbatim}[commandchars=\\\{\}]
{\color{incolor}In [{\color{incolor}13}]:} \PY{c+c1}{\PYZsh{} TODO: Import a supervised learning model that has \PYZsq{}feature\PYZus{}importances\PYZus{}\PYZsq{}}
         
         \PY{c+c1}{\PYZsh{} TODO: Train the supervised model on the training set }
         \PY{c+c1}{\PYZsh{}AdaBoostClassifier(random\PYZus{}state=0)}
         \PY{n}{model} \PY{o}{=} \PY{n}{AdaBoostClassifier}\PY{p}{(}\PY{n}{random\PYZus{}state}\PY{o}{=}\PY{l+m+mi}{0}\PY{p}{,}\PY{n}{n\PYZus{}estimators}\PY{o}{=}\PY{l+m+mi}{500}\PY{p}{)}\PY{o}{.}\PY{n}{fit}\PY{p}{(}\PY{n}{X\PYZus{}train}\PY{p}{,} \PY{n}{y\PYZus{}train}\PY{p}{)}
         
         \PY{c+c1}{\PYZsh{} TODO: Extract the feature importances}
         \PY{n}{importances} \PY{o}{=} \PY{n}{model}\PY{o}{.}\PY{n}{feature\PYZus{}importances\PYZus{}}
         
         \PY{c+c1}{\PYZsh{} Plot}
         \PY{n}{vs}\PY{o}{.}\PY{n}{feature\PYZus{}plot}\PY{p}{(}\PY{n}{importances}\PY{p}{,} \PY{n}{X\PYZus{}train}\PY{p}{,} \PY{n}{y\PYZus{}train}\PY{p}{)}
\end{Verbatim}


    \begin{center}
    \adjustimage{max size={0.9\linewidth}{0.9\paperheight}}{output_42_0.png}
    \end{center}
    { \hspace*{\fill} \\}
    
    \subsubsection{Extracting Feature
Importance}\label{extracting-feature-importance}

Observing the visualization created above which displays the five most
relevant features for predicting if an individual makes at most or above
\$50,000, we will try to answer the following questions.\\
* How do these five features compare to the five features you discussed
in \textbf{Question 6}? * If you were close to the same answer, how does
this visualization confirm your thoughts? * If you were not close, why
do you think these features are more relevant?

    \textbf{Answer:}

\begin{itemize}
\item
  The visualizations above are quite different than the ones predicted
  by me. My prediction was only partially right. Out of the five
  features predicted by me , 3 of them seem to pop up in the
  visualization above, though not in the same order.
\item
  I was not that close to the answer as my prediction's order were far
  off from the order presented in the visualization.
\item
  The reason I think I could not capture the truly important features is
  because I did not have a good insight of the data. Since the algorithm
  that I have performed above takes the insight of the data with
  mathematical manipulations, I think these machine learning algorithms
  better capture the true underlying importance of the individual
  features. Also as mentioned earlier I think capital-gain and
  capital-loss are events that do not occur often for a normal person
  and as a result they are given the least cumulative feature weight
  from the top five features selected. A younger person will definitely
  have lesser experience than a person who is older and we have seen in
  job market that people with more experience tend to make more money
  than people with less experience. After this no matter what the
  experience of the person, if a person with higher experience works
  less hours per week while a person with less experience works more
  hours, there are chances that a less experienced person will make more
  money than his counterpart and hence 'hours\_per\_week' has higher
  feature importance than 'age'. Finally the education number has
  highest cumulaive feature weight from the top five selected feature
  and it makes sense because a person with a higher education number
  will be highly qualified than person with lower education number and
  no matter how many hours per week he works or what his age is, he is
  qoing to make more money than his counterparts.
\end{itemize}

    \subsubsection{Feature Selection}\label{feature-selection}

How does a model perform if we only use a subset of all the available
features in the data? With less features required to train, the
expectation is that training and prediction time is much lower --- at
the cost of performance metrics. From the visualization above, we see
that the top five most important features contribute more than half of
the importance of \textbf{all} features present in the data. This hints
that we can attempt to \emph{reduce the feature space} and simplify the
information required for the model to learn. The code cell below will
use the same optimized model we found earlier, and train it on the same
training set \emph{with only the top five important features}.

    \begin{Verbatim}[commandchars=\\\{\}]
{\color{incolor}In [{\color{incolor}14}]:} \PY{c+c1}{\PYZsh{} Import functionality for cloning a model}
         \PY{k+kn}{from} \PY{n+nn}{sklearn}\PY{n+nn}{.}\PY{n+nn}{base} \PY{k}{import} \PY{n}{clone}
         
         \PY{c+c1}{\PYZsh{} Reduce the feature space}
         \PY{n}{X\PYZus{}train\PYZus{}reduced} \PY{o}{=} \PY{n}{X\PYZus{}train}\PY{p}{[}\PY{n}{X\PYZus{}train}\PY{o}{.}\PY{n}{columns}\PY{o}{.}\PY{n}{values}\PY{p}{[}\PY{p}{(}\PY{n}{np}\PY{o}{.}\PY{n}{argsort}\PY{p}{(}\PY{n}{importances}\PY{p}{)}\PY{p}{[}\PY{p}{:}\PY{p}{:}\PY{o}{\PYZhy{}}\PY{l+m+mi}{1}\PY{p}{]}\PY{p}{)}\PY{p}{[}\PY{p}{:}\PY{l+m+mi}{5}\PY{p}{]}\PY{p}{]}\PY{p}{]}
         \PY{n}{X\PYZus{}test\PYZus{}reduced} \PY{o}{=} \PY{n}{X\PYZus{}test}\PY{p}{[}\PY{n}{X\PYZus{}test}\PY{o}{.}\PY{n}{columns}\PY{o}{.}\PY{n}{values}\PY{p}{[}\PY{p}{(}\PY{n}{np}\PY{o}{.}\PY{n}{argsort}\PY{p}{(}\PY{n}{importances}\PY{p}{)}\PY{p}{[}\PY{p}{:}\PY{p}{:}\PY{o}{\PYZhy{}}\PY{l+m+mi}{1}\PY{p}{]}\PY{p}{)}\PY{p}{[}\PY{p}{:}\PY{l+m+mi}{5}\PY{p}{]}\PY{p}{]}\PY{p}{]}
         
         \PY{c+c1}{\PYZsh{} Train on the \PYZdq{}best\PYZdq{} model found from grid search earlier}
         \PY{n}{clf} \PY{o}{=} \PY{p}{(}\PY{n}{clone}\PY{p}{(}\PY{n}{best\PYZus{}clf}\PY{p}{)}\PY{p}{)}\PY{o}{.}\PY{n}{fit}\PY{p}{(}\PY{n}{X\PYZus{}train\PYZus{}reduced}\PY{p}{,} \PY{n}{y\PYZus{}train}\PY{p}{)}
         
         \PY{c+c1}{\PYZsh{} Make new predictions}
         \PY{n}{reduced\PYZus{}predictions} \PY{o}{=} \PY{n}{clf}\PY{o}{.}\PY{n}{predict}\PY{p}{(}\PY{n}{X\PYZus{}test\PYZus{}reduced}\PY{p}{)}
         
         \PY{c+c1}{\PYZsh{} Report scores from the final model using both versions of data}
         \PY{n+nb}{print}\PY{p}{(}\PY{l+s+s2}{\PYZdq{}}\PY{l+s+s2}{Final Model trained on full data}\PY{l+s+se}{\PYZbs{}n}\PY{l+s+s2}{\PYZhy{}\PYZhy{}\PYZhy{}\PYZhy{}\PYZhy{}\PYZhy{}}\PY{l+s+s2}{\PYZdq{}}\PY{p}{)}
         \PY{n+nb}{print}\PY{p}{(}\PY{l+s+s2}{\PYZdq{}}\PY{l+s+s2}{Accuracy on testing data: }\PY{l+s+si}{\PYZob{}:.4f\PYZcb{}}\PY{l+s+s2}{\PYZdq{}}\PY{o}{.}\PY{n}{format}\PY{p}{(}\PY{n}{accuracy\PYZus{}score}\PY{p}{(}\PY{n}{y\PYZus{}test}\PY{p}{,} \PY{n}{best\PYZus{}predictions}\PY{p}{)}\PY{p}{)}\PY{p}{)}
         \PY{n+nb}{print}\PY{p}{(}\PY{l+s+s2}{\PYZdq{}}\PY{l+s+s2}{F\PYZhy{}score on testing data: }\PY{l+s+si}{\PYZob{}:.4f\PYZcb{}}\PY{l+s+s2}{\PYZdq{}}\PY{o}{.}\PY{n}{format}\PY{p}{(}\PY{n}{fbeta\PYZus{}score}\PY{p}{(}\PY{n}{y\PYZus{}test}\PY{p}{,} \PY{n}{best\PYZus{}predictions}\PY{p}{,} \PY{n}{beta} \PY{o}{=} \PY{l+m+mf}{0.5}\PY{p}{)}\PY{p}{)}\PY{p}{)}
         \PY{n+nb}{print}\PY{p}{(}\PY{l+s+s2}{\PYZdq{}}\PY{l+s+se}{\PYZbs{}n}\PY{l+s+s2}{Final Model trained on reduced data}\PY{l+s+se}{\PYZbs{}n}\PY{l+s+s2}{\PYZhy{}\PYZhy{}\PYZhy{}\PYZhy{}\PYZhy{}\PYZhy{}}\PY{l+s+s2}{\PYZdq{}}\PY{p}{)}
         \PY{n+nb}{print}\PY{p}{(}\PY{l+s+s2}{\PYZdq{}}\PY{l+s+s2}{Accuracy on testing data: }\PY{l+s+si}{\PYZob{}:.4f\PYZcb{}}\PY{l+s+s2}{\PYZdq{}}\PY{o}{.}\PY{n}{format}\PY{p}{(}\PY{n}{accuracy\PYZus{}score}\PY{p}{(}\PY{n}{y\PYZus{}test}\PY{p}{,} \PY{n}{reduced\PYZus{}predictions}\PY{p}{)}\PY{p}{)}\PY{p}{)}
         \PY{n+nb}{print}\PY{p}{(}\PY{l+s+s2}{\PYZdq{}}\PY{l+s+s2}{F\PYZhy{}score on testing data: }\PY{l+s+si}{\PYZob{}:.4f\PYZcb{}}\PY{l+s+s2}{\PYZdq{}}\PY{o}{.}\PY{n}{format}\PY{p}{(}\PY{n}{fbeta\PYZus{}score}\PY{p}{(}\PY{n}{y\PYZus{}test}\PY{p}{,} \PY{n}{reduced\PYZus{}predictions}\PY{p}{,} \PY{n}{beta} \PY{o}{=} \PY{l+m+mf}{0.5}\PY{p}{)}\PY{p}{)}\PY{p}{)}
\end{Verbatim}


    \begin{Verbatim}[commandchars=\\\{\}]
Final Model trained on full data
------
Accuracy on testing data: 0.8677
F-score on testing data: 0.7457

Final Model trained on reduced data
------
Accuracy on testing data: 0.8421
F-score on testing data: 0.7017

    \end{Verbatim}

    \subsubsection{Effects of Feature
Selection}\label{effects-of-feature-selection}

\begin{itemize}
\tightlist
\item
  How does the final model's F-score and accuracy score on the reduced
  data using only five features compare to those same scores when all
  features are used?
\item
  If training time was a factor, would you consider using the reduced
  data as your training set?
\end{itemize}

    \textbf{Answer:}

\paragraph{Results:}\label{results}

\begin{longtable}[]{@{}ccc@{}}
\toprule
Metric & Final Model(Full data) & Final Model(Reduced
data)\tabularnewline
\midrule
\endhead
Accuracy Score & 0.8677 & 0.8421\tabularnewline
F-score & 0.7457 & 0.7017\tabularnewline
\bottomrule
\end{longtable}

\begin{itemize}
\item
  As we can see from the table above the accuracy of the final model
  with reduced data is almost 3\% lesser than the model with full data
  whereas the F-score of reduced model is approximately 4\% less than
  the F-score of model with full data. However the accuracy and f-score
  of the reduced data model is still very high as compared to the naive
  model. This reduction in model performance should not come as a shock,
  because what we are doing with reduced data model is removing or
  reducing information. Intuitively anything which has less information
  is always going to perform poorly than anything that has comparitivly
  more information. Thus no matter how less important the other features
  are, if we remove those features the model performance will be
  affected negatively to some level.
\item
  Feature removal should never be done when training the data does not
  take more time because removing information without any reason makes
  no sense. However in real world some datasets are huge and as we saw
  above there are certain algorithms like SVM which take a lot of time
  to train on large datasets and if we are working with real time
  applications we cannot afford such high training time, under such
  circumstances we should trade performance in favour of training time,
  however if the applications are life altering(For eg. Medical
  applications) then we should never compromise the performance of the
  model. Thus feature reduction is subjective and it depends highly on
  the application
\end{itemize}

    \begin{Verbatim}[commandchars=\\\{\}]
{\color{incolor}In [{\color{incolor}3}]:} \PY{o}{!!}jupyter nbconvert *.ipynb
\end{Verbatim}


\begin{Verbatim}[commandchars=\\\{\}]
{\color{outcolor}Out[{\color{outcolor}3}]:} ['[NbConvertApp] Converting notebook Untitled.ipynb to html',
         '[NbConvertApp] Writing 248821 bytes to Untitled.html',
         '[NbConvertApp] Converting notebook finding\_donors.ipynb to html',
         '[NbConvertApp] Writing 472983 bytes to finding\_donors.html']
\end{Verbatim}
            

    % Add a bibliography block to the postdoc
    
    
    
    \end{document}
