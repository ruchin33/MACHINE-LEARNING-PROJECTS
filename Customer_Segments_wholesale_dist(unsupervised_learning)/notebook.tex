
% Default to the notebook output style

    


% Inherit from the specified cell style.




    
\documentclass[11pt]{article}

    
    
    \usepackage[T1]{fontenc}
    % Nicer default font (+ math font) than Computer Modern for most use cases
    \usepackage{mathpazo}

    % Basic figure setup, for now with no caption control since it's done
    % automatically by Pandoc (which extracts ![](path) syntax from Markdown).
    \usepackage{graphicx}
    % We will generate all images so they have a width \maxwidth. This means
    % that they will get their normal width if they fit onto the page, but
    % are scaled down if they would overflow the margins.
    \makeatletter
    \def\maxwidth{\ifdim\Gin@nat@width>\linewidth\linewidth
    \else\Gin@nat@width\fi}
    \makeatother
    \let\Oldincludegraphics\includegraphics
    % Set max figure width to be 80% of text width, for now hardcoded.
    \renewcommand{\includegraphics}[1]{\Oldincludegraphics[width=.8\maxwidth]{#1}}
    % Ensure that by default, figures have no caption (until we provide a
    % proper Figure object with a Caption API and a way to capture that
    % in the conversion process - todo).
    \usepackage{caption}
    \DeclareCaptionLabelFormat{nolabel}{}
    \captionsetup{labelformat=nolabel}

    \usepackage{adjustbox} % Used to constrain images to a maximum size 
    \usepackage{xcolor} % Allow colors to be defined
    \usepackage{enumerate} % Needed for markdown enumerations to work
    \usepackage{geometry} % Used to adjust the document margins
    \usepackage{amsmath} % Equations
    \usepackage{amssymb} % Equations
    \usepackage{textcomp} % defines textquotesingle
    % Hack from http://tex.stackexchange.com/a/47451/13684:
    \AtBeginDocument{%
        \def\PYZsq{\textquotesingle}% Upright quotes in Pygmentized code
    }
    \usepackage{upquote} % Upright quotes for verbatim code
    \usepackage{eurosym} % defines \euro
    \usepackage[mathletters]{ucs} % Extended unicode (utf-8) support
    \usepackage[utf8x]{inputenc} % Allow utf-8 characters in the tex document
    \usepackage{fancyvrb} % verbatim replacement that allows latex
    \usepackage{grffile} % extends the file name processing of package graphics 
                         % to support a larger range 
    % The hyperref package gives us a pdf with properly built
    % internal navigation ('pdf bookmarks' for the table of contents,
    % internal cross-reference links, web links for URLs, etc.)
    \usepackage{hyperref}
    \usepackage{longtable} % longtable support required by pandoc >1.10
    \usepackage{booktabs}  % table support for pandoc > 1.12.2
    \usepackage[inline]{enumitem} % IRkernel/repr support (it uses the enumerate* environment)
    \usepackage[normalem]{ulem} % ulem is needed to support strikethroughs (\sout)
                                % normalem makes italics be italics, not underlines
    

    
    
    % Colors for the hyperref package
    \definecolor{urlcolor}{rgb}{0,.145,.698}
    \definecolor{linkcolor}{rgb}{.71,0.21,0.01}
    \definecolor{citecolor}{rgb}{.12,.54,.11}

    % ANSI colors
    \definecolor{ansi-black}{HTML}{3E424D}
    \definecolor{ansi-black-intense}{HTML}{282C36}
    \definecolor{ansi-red}{HTML}{E75C58}
    \definecolor{ansi-red-intense}{HTML}{B22B31}
    \definecolor{ansi-green}{HTML}{00A250}
    \definecolor{ansi-green-intense}{HTML}{007427}
    \definecolor{ansi-yellow}{HTML}{DDB62B}
    \definecolor{ansi-yellow-intense}{HTML}{B27D12}
    \definecolor{ansi-blue}{HTML}{208FFB}
    \definecolor{ansi-blue-intense}{HTML}{0065CA}
    \definecolor{ansi-magenta}{HTML}{D160C4}
    \definecolor{ansi-magenta-intense}{HTML}{A03196}
    \definecolor{ansi-cyan}{HTML}{60C6C8}
    \definecolor{ansi-cyan-intense}{HTML}{258F8F}
    \definecolor{ansi-white}{HTML}{C5C1B4}
    \definecolor{ansi-white-intense}{HTML}{A1A6B2}

    % commands and environments needed by pandoc snippets
    % extracted from the output of `pandoc -s`
    \providecommand{\tightlist}{%
      \setlength{\itemsep}{0pt}\setlength{\parskip}{0pt}}
    \DefineVerbatimEnvironment{Highlighting}{Verbatim}{commandchars=\\\{\}}
    % Add ',fontsize=\small' for more characters per line
    \newenvironment{Shaded}{}{}
    \newcommand{\KeywordTok}[1]{\textcolor[rgb]{0.00,0.44,0.13}{\textbf{{#1}}}}
    \newcommand{\DataTypeTok}[1]{\textcolor[rgb]{0.56,0.13,0.00}{{#1}}}
    \newcommand{\DecValTok}[1]{\textcolor[rgb]{0.25,0.63,0.44}{{#1}}}
    \newcommand{\BaseNTok}[1]{\textcolor[rgb]{0.25,0.63,0.44}{{#1}}}
    \newcommand{\FloatTok}[1]{\textcolor[rgb]{0.25,0.63,0.44}{{#1}}}
    \newcommand{\CharTok}[1]{\textcolor[rgb]{0.25,0.44,0.63}{{#1}}}
    \newcommand{\StringTok}[1]{\textcolor[rgb]{0.25,0.44,0.63}{{#1}}}
    \newcommand{\CommentTok}[1]{\textcolor[rgb]{0.38,0.63,0.69}{\textit{{#1}}}}
    \newcommand{\OtherTok}[1]{\textcolor[rgb]{0.00,0.44,0.13}{{#1}}}
    \newcommand{\AlertTok}[1]{\textcolor[rgb]{1.00,0.00,0.00}{\textbf{{#1}}}}
    \newcommand{\FunctionTok}[1]{\textcolor[rgb]{0.02,0.16,0.49}{{#1}}}
    \newcommand{\RegionMarkerTok}[1]{{#1}}
    \newcommand{\ErrorTok}[1]{\textcolor[rgb]{1.00,0.00,0.00}{\textbf{{#1}}}}
    \newcommand{\NormalTok}[1]{{#1}}
    
    % Additional commands for more recent versions of Pandoc
    \newcommand{\ConstantTok}[1]{\textcolor[rgb]{0.53,0.00,0.00}{{#1}}}
    \newcommand{\SpecialCharTok}[1]{\textcolor[rgb]{0.25,0.44,0.63}{{#1}}}
    \newcommand{\VerbatimStringTok}[1]{\textcolor[rgb]{0.25,0.44,0.63}{{#1}}}
    \newcommand{\SpecialStringTok}[1]{\textcolor[rgb]{0.73,0.40,0.53}{{#1}}}
    \newcommand{\ImportTok}[1]{{#1}}
    \newcommand{\DocumentationTok}[1]{\textcolor[rgb]{0.73,0.13,0.13}{\textit{{#1}}}}
    \newcommand{\AnnotationTok}[1]{\textcolor[rgb]{0.38,0.63,0.69}{\textbf{\textit{{#1}}}}}
    \newcommand{\CommentVarTok}[1]{\textcolor[rgb]{0.38,0.63,0.69}{\textbf{\textit{{#1}}}}}
    \newcommand{\VariableTok}[1]{\textcolor[rgb]{0.10,0.09,0.49}{{#1}}}
    \newcommand{\ControlFlowTok}[1]{\textcolor[rgb]{0.00,0.44,0.13}{\textbf{{#1}}}}
    \newcommand{\OperatorTok}[1]{\textcolor[rgb]{0.40,0.40,0.40}{{#1}}}
    \newcommand{\BuiltInTok}[1]{{#1}}
    \newcommand{\ExtensionTok}[1]{{#1}}
    \newcommand{\PreprocessorTok}[1]{\textcolor[rgb]{0.74,0.48,0.00}{{#1}}}
    \newcommand{\AttributeTok}[1]{\textcolor[rgb]{0.49,0.56,0.16}{{#1}}}
    \newcommand{\InformationTok}[1]{\textcolor[rgb]{0.38,0.63,0.69}{\textbf{\textit{{#1}}}}}
    \newcommand{\WarningTok}[1]{\textcolor[rgb]{0.38,0.63,0.69}{\textbf{\textit{{#1}}}}}
    
    
    % Define a nice break command that doesn't care if a line doesn't already
    % exist.
    \def\br{\hspace*{\fill} \\* }
    % Math Jax compatability definitions
    \def\gt{>}
    \def\lt{<}
    % Document parameters
    \title{Machine\_Learning\_Customer\_Segments\_for\_wholesale\_distributor}
    
    
    

    % Pygments definitions
    
\makeatletter
\def\PY@reset{\let\PY@it=\relax \let\PY@bf=\relax%
    \let\PY@ul=\relax \let\PY@tc=\relax%
    \let\PY@bc=\relax \let\PY@ff=\relax}
\def\PY@tok#1{\csname PY@tok@#1\endcsname}
\def\PY@toks#1+{\ifx\relax#1\empty\else%
    \PY@tok{#1}\expandafter\PY@toks\fi}
\def\PY@do#1{\PY@bc{\PY@tc{\PY@ul{%
    \PY@it{\PY@bf{\PY@ff{#1}}}}}}}
\def\PY#1#2{\PY@reset\PY@toks#1+\relax+\PY@do{#2}}

\expandafter\def\csname PY@tok@w\endcsname{\def\PY@tc##1{\textcolor[rgb]{0.73,0.73,0.73}{##1}}}
\expandafter\def\csname PY@tok@c\endcsname{\let\PY@it=\textit\def\PY@tc##1{\textcolor[rgb]{0.25,0.50,0.50}{##1}}}
\expandafter\def\csname PY@tok@cp\endcsname{\def\PY@tc##1{\textcolor[rgb]{0.74,0.48,0.00}{##1}}}
\expandafter\def\csname PY@tok@k\endcsname{\let\PY@bf=\textbf\def\PY@tc##1{\textcolor[rgb]{0.00,0.50,0.00}{##1}}}
\expandafter\def\csname PY@tok@kp\endcsname{\def\PY@tc##1{\textcolor[rgb]{0.00,0.50,0.00}{##1}}}
\expandafter\def\csname PY@tok@kt\endcsname{\def\PY@tc##1{\textcolor[rgb]{0.69,0.00,0.25}{##1}}}
\expandafter\def\csname PY@tok@o\endcsname{\def\PY@tc##1{\textcolor[rgb]{0.40,0.40,0.40}{##1}}}
\expandafter\def\csname PY@tok@ow\endcsname{\let\PY@bf=\textbf\def\PY@tc##1{\textcolor[rgb]{0.67,0.13,1.00}{##1}}}
\expandafter\def\csname PY@tok@nb\endcsname{\def\PY@tc##1{\textcolor[rgb]{0.00,0.50,0.00}{##1}}}
\expandafter\def\csname PY@tok@nf\endcsname{\def\PY@tc##1{\textcolor[rgb]{0.00,0.00,1.00}{##1}}}
\expandafter\def\csname PY@tok@nc\endcsname{\let\PY@bf=\textbf\def\PY@tc##1{\textcolor[rgb]{0.00,0.00,1.00}{##1}}}
\expandafter\def\csname PY@tok@nn\endcsname{\let\PY@bf=\textbf\def\PY@tc##1{\textcolor[rgb]{0.00,0.00,1.00}{##1}}}
\expandafter\def\csname PY@tok@ne\endcsname{\let\PY@bf=\textbf\def\PY@tc##1{\textcolor[rgb]{0.82,0.25,0.23}{##1}}}
\expandafter\def\csname PY@tok@nv\endcsname{\def\PY@tc##1{\textcolor[rgb]{0.10,0.09,0.49}{##1}}}
\expandafter\def\csname PY@tok@no\endcsname{\def\PY@tc##1{\textcolor[rgb]{0.53,0.00,0.00}{##1}}}
\expandafter\def\csname PY@tok@nl\endcsname{\def\PY@tc##1{\textcolor[rgb]{0.63,0.63,0.00}{##1}}}
\expandafter\def\csname PY@tok@ni\endcsname{\let\PY@bf=\textbf\def\PY@tc##1{\textcolor[rgb]{0.60,0.60,0.60}{##1}}}
\expandafter\def\csname PY@tok@na\endcsname{\def\PY@tc##1{\textcolor[rgb]{0.49,0.56,0.16}{##1}}}
\expandafter\def\csname PY@tok@nt\endcsname{\let\PY@bf=\textbf\def\PY@tc##1{\textcolor[rgb]{0.00,0.50,0.00}{##1}}}
\expandafter\def\csname PY@tok@nd\endcsname{\def\PY@tc##1{\textcolor[rgb]{0.67,0.13,1.00}{##1}}}
\expandafter\def\csname PY@tok@s\endcsname{\def\PY@tc##1{\textcolor[rgb]{0.73,0.13,0.13}{##1}}}
\expandafter\def\csname PY@tok@sd\endcsname{\let\PY@it=\textit\def\PY@tc##1{\textcolor[rgb]{0.73,0.13,0.13}{##1}}}
\expandafter\def\csname PY@tok@si\endcsname{\let\PY@bf=\textbf\def\PY@tc##1{\textcolor[rgb]{0.73,0.40,0.53}{##1}}}
\expandafter\def\csname PY@tok@se\endcsname{\let\PY@bf=\textbf\def\PY@tc##1{\textcolor[rgb]{0.73,0.40,0.13}{##1}}}
\expandafter\def\csname PY@tok@sr\endcsname{\def\PY@tc##1{\textcolor[rgb]{0.73,0.40,0.53}{##1}}}
\expandafter\def\csname PY@tok@ss\endcsname{\def\PY@tc##1{\textcolor[rgb]{0.10,0.09,0.49}{##1}}}
\expandafter\def\csname PY@tok@sx\endcsname{\def\PY@tc##1{\textcolor[rgb]{0.00,0.50,0.00}{##1}}}
\expandafter\def\csname PY@tok@m\endcsname{\def\PY@tc##1{\textcolor[rgb]{0.40,0.40,0.40}{##1}}}
\expandafter\def\csname PY@tok@gh\endcsname{\let\PY@bf=\textbf\def\PY@tc##1{\textcolor[rgb]{0.00,0.00,0.50}{##1}}}
\expandafter\def\csname PY@tok@gu\endcsname{\let\PY@bf=\textbf\def\PY@tc##1{\textcolor[rgb]{0.50,0.00,0.50}{##1}}}
\expandafter\def\csname PY@tok@gd\endcsname{\def\PY@tc##1{\textcolor[rgb]{0.63,0.00,0.00}{##1}}}
\expandafter\def\csname PY@tok@gi\endcsname{\def\PY@tc##1{\textcolor[rgb]{0.00,0.63,0.00}{##1}}}
\expandafter\def\csname PY@tok@gr\endcsname{\def\PY@tc##1{\textcolor[rgb]{1.00,0.00,0.00}{##1}}}
\expandafter\def\csname PY@tok@ge\endcsname{\let\PY@it=\textit}
\expandafter\def\csname PY@tok@gs\endcsname{\let\PY@bf=\textbf}
\expandafter\def\csname PY@tok@gp\endcsname{\let\PY@bf=\textbf\def\PY@tc##1{\textcolor[rgb]{0.00,0.00,0.50}{##1}}}
\expandafter\def\csname PY@tok@go\endcsname{\def\PY@tc##1{\textcolor[rgb]{0.53,0.53,0.53}{##1}}}
\expandafter\def\csname PY@tok@gt\endcsname{\def\PY@tc##1{\textcolor[rgb]{0.00,0.27,0.87}{##1}}}
\expandafter\def\csname PY@tok@err\endcsname{\def\PY@bc##1{\setlength{\fboxsep}{0pt}\fcolorbox[rgb]{1.00,0.00,0.00}{1,1,1}{\strut ##1}}}
\expandafter\def\csname PY@tok@kc\endcsname{\let\PY@bf=\textbf\def\PY@tc##1{\textcolor[rgb]{0.00,0.50,0.00}{##1}}}
\expandafter\def\csname PY@tok@kd\endcsname{\let\PY@bf=\textbf\def\PY@tc##1{\textcolor[rgb]{0.00,0.50,0.00}{##1}}}
\expandafter\def\csname PY@tok@kn\endcsname{\let\PY@bf=\textbf\def\PY@tc##1{\textcolor[rgb]{0.00,0.50,0.00}{##1}}}
\expandafter\def\csname PY@tok@kr\endcsname{\let\PY@bf=\textbf\def\PY@tc##1{\textcolor[rgb]{0.00,0.50,0.00}{##1}}}
\expandafter\def\csname PY@tok@bp\endcsname{\def\PY@tc##1{\textcolor[rgb]{0.00,0.50,0.00}{##1}}}
\expandafter\def\csname PY@tok@fm\endcsname{\def\PY@tc##1{\textcolor[rgb]{0.00,0.00,1.00}{##1}}}
\expandafter\def\csname PY@tok@vc\endcsname{\def\PY@tc##1{\textcolor[rgb]{0.10,0.09,0.49}{##1}}}
\expandafter\def\csname PY@tok@vg\endcsname{\def\PY@tc##1{\textcolor[rgb]{0.10,0.09,0.49}{##1}}}
\expandafter\def\csname PY@tok@vi\endcsname{\def\PY@tc##1{\textcolor[rgb]{0.10,0.09,0.49}{##1}}}
\expandafter\def\csname PY@tok@vm\endcsname{\def\PY@tc##1{\textcolor[rgb]{0.10,0.09,0.49}{##1}}}
\expandafter\def\csname PY@tok@sa\endcsname{\def\PY@tc##1{\textcolor[rgb]{0.73,0.13,0.13}{##1}}}
\expandafter\def\csname PY@tok@sb\endcsname{\def\PY@tc##1{\textcolor[rgb]{0.73,0.13,0.13}{##1}}}
\expandafter\def\csname PY@tok@sc\endcsname{\def\PY@tc##1{\textcolor[rgb]{0.73,0.13,0.13}{##1}}}
\expandafter\def\csname PY@tok@dl\endcsname{\def\PY@tc##1{\textcolor[rgb]{0.73,0.13,0.13}{##1}}}
\expandafter\def\csname PY@tok@s2\endcsname{\def\PY@tc##1{\textcolor[rgb]{0.73,0.13,0.13}{##1}}}
\expandafter\def\csname PY@tok@sh\endcsname{\def\PY@tc##1{\textcolor[rgb]{0.73,0.13,0.13}{##1}}}
\expandafter\def\csname PY@tok@s1\endcsname{\def\PY@tc##1{\textcolor[rgb]{0.73,0.13,0.13}{##1}}}
\expandafter\def\csname PY@tok@mb\endcsname{\def\PY@tc##1{\textcolor[rgb]{0.40,0.40,0.40}{##1}}}
\expandafter\def\csname PY@tok@mf\endcsname{\def\PY@tc##1{\textcolor[rgb]{0.40,0.40,0.40}{##1}}}
\expandafter\def\csname PY@tok@mh\endcsname{\def\PY@tc##1{\textcolor[rgb]{0.40,0.40,0.40}{##1}}}
\expandafter\def\csname PY@tok@mi\endcsname{\def\PY@tc##1{\textcolor[rgb]{0.40,0.40,0.40}{##1}}}
\expandafter\def\csname PY@tok@il\endcsname{\def\PY@tc##1{\textcolor[rgb]{0.40,0.40,0.40}{##1}}}
\expandafter\def\csname PY@tok@mo\endcsname{\def\PY@tc##1{\textcolor[rgb]{0.40,0.40,0.40}{##1}}}
\expandafter\def\csname PY@tok@ch\endcsname{\let\PY@it=\textit\def\PY@tc##1{\textcolor[rgb]{0.25,0.50,0.50}{##1}}}
\expandafter\def\csname PY@tok@cm\endcsname{\let\PY@it=\textit\def\PY@tc##1{\textcolor[rgb]{0.25,0.50,0.50}{##1}}}
\expandafter\def\csname PY@tok@cpf\endcsname{\let\PY@it=\textit\def\PY@tc##1{\textcolor[rgb]{0.25,0.50,0.50}{##1}}}
\expandafter\def\csname PY@tok@c1\endcsname{\let\PY@it=\textit\def\PY@tc##1{\textcolor[rgb]{0.25,0.50,0.50}{##1}}}
\expandafter\def\csname PY@tok@cs\endcsname{\let\PY@it=\textit\def\PY@tc##1{\textcolor[rgb]{0.25,0.50,0.50}{##1}}}

\def\PYZbs{\char`\\}
\def\PYZus{\char`\_}
\def\PYZob{\char`\{}
\def\PYZcb{\char`\}}
\def\PYZca{\char`\^}
\def\PYZam{\char`\&}
\def\PYZlt{\char`\<}
\def\PYZgt{\char`\>}
\def\PYZsh{\char`\#}
\def\PYZpc{\char`\%}
\def\PYZdl{\char`\$}
\def\PYZhy{\char`\-}
\def\PYZsq{\char`\'}
\def\PYZdq{\char`\"}
\def\PYZti{\char`\~}
% for compatibility with earlier versions
\def\PYZat{@}
\def\PYZlb{[}
\def\PYZrb{]}
\makeatother


    % Exact colors from NB
    \definecolor{incolor}{rgb}{0.0, 0.0, 0.5}
    \definecolor{outcolor}{rgb}{0.545, 0.0, 0.0}



    
    % Prevent overflowing lines due to hard-to-break entities
    \sloppy 
    % Setup hyperref package
    \hypersetup{
      breaklinks=true,  % so long urls are correctly broken across lines
      colorlinks=true,
      urlcolor=urlcolor,
      linkcolor=linkcolor,
      citecolor=citecolor,
      }
    % Slightly bigger margins than the latex defaults
    
    \geometry{verbose,tmargin=1in,bmargin=1in,lmargin=1in,rmargin=1in}
    
    

    \begin{document}
    
    
    \maketitle
    
    

    
    \section{Project: Creating Customer
Segments}\label{project-creating-customer-segments}

    \subsection{Getting Started}\label{getting-started}

In this project, we will analyze a dataset containing data on various
customers' annual spending amounts (reported in \emph{monetary units})
of diverse product categories for internal structure. One goal of this
project is to best describe the variation in the different types of
customers that a wholesale distributor interacts with. Doing so would
equip the distributor with insight into how to best structure their
delivery service to meet the needs of each customer.

The dataset for this project can be found on the
\href{https://archive.ics.uci.edu/ml/datasets/Wholesale+customers}{UCI
Machine Learning Repository}. For the purposes of this project, the
features \texttt{\textquotesingle{}Channel\textquotesingle{}} and
\texttt{\textquotesingle{}Region\textquotesingle{}} will be excluded in
the analysis --- with focus instead on the six product categories
recorded for customers.

    \begin{Verbatim}[commandchars=\\\{\}]
{\color{incolor}In [{\color{incolor}1}]:} \PY{c+c1}{\PYZsh{} Import libraries necessary for this project}
        \PY{k+kn}{import} \PY{n+nn}{numpy} \PY{k}{as} \PY{n+nn}{np}
        \PY{k+kn}{import} \PY{n+nn}{pandas} \PY{k}{as} \PY{n+nn}{pd}
        \PY{k+kn}{from} \PY{n+nn}{IPython}\PY{n+nn}{.}\PY{n+nn}{display} \PY{k}{import} \PY{n}{display} \PY{c+c1}{\PYZsh{} Allows the use of display() for DataFrames}
        
        \PY{c+c1}{\PYZsh{} Import supplementary visualizations code visuals.py}
        \PY{k+kn}{import} \PY{n+nn}{visuals} \PY{k}{as} \PY{n+nn}{vs}
        
        \PY{c+c1}{\PYZsh{} Pretty display for notebooks}
        \PY{o}{\PYZpc{}}\PY{k}{matplotlib} inline
        
        \PY{c+c1}{\PYZsh{} Load the wholesale customers dataset}
        \PY{k}{try}\PY{p}{:}
            \PY{n}{data} \PY{o}{=} \PY{n}{pd}\PY{o}{.}\PY{n}{read\PYZus{}csv}\PY{p}{(}\PY{l+s+s2}{\PYZdq{}}\PY{l+s+s2}{customers.csv}\PY{l+s+s2}{\PYZdq{}}\PY{p}{)}
            \PY{n}{data}\PY{o}{.}\PY{n}{drop}\PY{p}{(}\PY{p}{[}\PY{l+s+s1}{\PYZsq{}}\PY{l+s+s1}{Region}\PY{l+s+s1}{\PYZsq{}}\PY{p}{,} \PY{l+s+s1}{\PYZsq{}}\PY{l+s+s1}{Channel}\PY{l+s+s1}{\PYZsq{}}\PY{p}{]}\PY{p}{,} \PY{n}{axis} \PY{o}{=} \PY{l+m+mi}{1}\PY{p}{,} \PY{n}{inplace} \PY{o}{=} \PY{k+kc}{True}\PY{p}{)}
            \PY{n+nb}{print}\PY{p}{(}\PY{l+s+s2}{\PYZdq{}}\PY{l+s+s2}{Wholesale customers dataset has }\PY{l+s+si}{\PYZob{}\PYZcb{}}\PY{l+s+s2}{ samples with }\PY{l+s+si}{\PYZob{}\PYZcb{}}\PY{l+s+s2}{ features each.}\PY{l+s+s2}{\PYZdq{}}\PY{o}{.}\PY{n}{format}\PY{p}{(}\PY{o}{*}\PY{n}{data}\PY{o}{.}\PY{n}{shape}\PY{p}{)}\PY{p}{)}
        \PY{k}{except}\PY{p}{:}
            \PY{n+nb}{print}\PY{p}{(}\PY{l+s+s2}{\PYZdq{}}\PY{l+s+s2}{Dataset could not be loaded. Is the dataset missing?}\PY{l+s+s2}{\PYZdq{}}\PY{p}{)}
\end{Verbatim}


    \begin{Verbatim}[commandchars=\\\{\}]
Wholesale customers dataset has 440 samples with 6 features each.

    \end{Verbatim}

    \subsection{Data Exploration}\label{data-exploration}

In this section, we will begin exploring the data through visualizations
and code to understand how each feature is related to the others. You
will observe a statistical description of the dataset, consider the
relevance of each feature, and select a few sample data points from the
dataset which we will track through the course of this project.

Note that the dataset is composed of six important product categories:
\textbf{'Fresh'}, \textbf{'Milk'}, \textbf{'Grocery'},
\textbf{'Frozen'}, \textbf{'Detergents\_Paper'}, and
\textbf{'Delicatessen'}.

    \begin{Verbatim}[commandchars=\\\{\}]
{\color{incolor}In [{\color{incolor}2}]:} \PY{c+c1}{\PYZsh{} Display a description of the dataset}
        \PY{n}{display}\PY{p}{(}\PY{n}{data}\PY{o}{.}\PY{n}{describe}\PY{p}{(}\PY{p}{)}\PY{p}{)}
\end{Verbatim}


    
    \begin{verbatim}
               Fresh          Milk       Grocery        Frozen  \
count     440.000000    440.000000    440.000000    440.000000   
mean    12000.297727   5796.265909   7951.277273   3071.931818   
std     12647.328865   7380.377175   9503.162829   4854.673333   
min         3.000000     55.000000      3.000000     25.000000   
25%      3127.750000   1533.000000   2153.000000    742.250000   
50%      8504.000000   3627.000000   4755.500000   1526.000000   
75%     16933.750000   7190.250000  10655.750000   3554.250000   
max    112151.000000  73498.000000  92780.000000  60869.000000   

       Detergents_Paper  Delicatessen  
count        440.000000    440.000000  
mean        2881.493182   1524.870455  
std         4767.854448   2820.105937  
min            3.000000      3.000000  
25%          256.750000    408.250000  
50%          816.500000    965.500000  
75%         3922.000000   1820.250000  
max        40827.000000  47943.000000  
    \end{verbatim}

    
    \subsubsection{Implementation: Selecting
Samples}\label{implementation-selecting-samples}

To get a better understanding of the customers and how their data will
transform through the analysis, it would be best to select a few sample
data points and explore them in more detail. In the code block below, we
add \textbf{three} indices of our choice to the \texttt{indices} list
which will represent the customers to track. It is suggested to try
different sets of samples until we obtain customers that vary
significantly from one another.

    \begin{Verbatim}[commandchars=\\\{\}]
{\color{incolor}In [{\color{incolor}3}]:} \PY{c+c1}{\PYZsh{} TODO: Select three indices of your choice you wish to sample from the dataset}
        \PY{n}{indices} \PY{o}{=} \PY{p}{[}\PY{l+m+mi}{3}\PY{p}{,}\PY{l+m+mi}{87}\PY{p}{,}\PY{l+m+mi}{200}\PY{p}{]}
        
        \PY{c+c1}{\PYZsh{} Create a DataFrame of the chosen samples}
        \PY{n}{samples} \PY{o}{=} \PY{n}{pd}\PY{o}{.}\PY{n}{DataFrame}\PY{p}{(}\PY{n}{data}\PY{o}{.}\PY{n}{loc}\PY{p}{[}\PY{n}{indices}\PY{p}{]}\PY{p}{,} \PY{n}{columns} \PY{o}{=} \PY{n}{data}\PY{o}{.}\PY{n}{keys}\PY{p}{(}\PY{p}{)}\PY{p}{)}\PY{o}{.}\PY{n}{reset\PYZus{}index}\PY{p}{(}\PY{n}{drop} \PY{o}{=} \PY{k+kc}{True}\PY{p}{)}
\end{Verbatim}


    \subsubsection{Question}\label{question}

Consider the total purchase cost of each product category and the
statistical description of the dataset above for our sample customers.

\begin{itemize}
\tightlist
\item
  What kind of establishment (customer) could each of the three samples
  we've chosen represent?
\end{itemize}

The mean values are as follows:

\begin{itemize}
\tightlist
\item
  Fresh: 12000.2977
\item
  Milk: 5796.2
\item
  Grocery: 7951.277273
\item
  Frozen: 3071.9
\item
  Detergents\_paper: 2881.4
\item
  Delicatessen: 1524.8
\end{itemize}

Knowing this, how do our samples compare? Does that help in driving our
insight into what kind of establishments they might be?

    \begin{Verbatim}[commandchars=\\\{\}]
{\color{incolor}In [{\color{incolor}6}]:} \PY{k+kn}{import} \PY{n+nn}{seaborn} \PY{k}{as} \PY{n+nn}{sns}
        \PY{k+kn}{import} \PY{n+nn}{matplotlib}\PY{n+nn}{.}\PY{n+nn}{pyplot} \PY{k}{as} \PY{n+nn}{plt}
        
        \PY{c+c1}{\PYZsh{}Percentile values of the the sampled data}
        \PY{n}{percentile\PYZus{}values} \PY{o}{=} \PY{l+m+mf}{100.} \PY{o}{*}\PY{n}{data}\PY{o}{.}\PY{n}{rank}\PY{p}{(}\PY{n}{axis}\PY{o}{=}\PY{l+m+mi}{0}\PY{p}{,} \PY{n}{pct}\PY{o}{=}\PY{k+kc}{True}\PY{p}{)}\PY{o}{.}\PY{n}{iloc}\PY{p}{[}\PY{n}{indices}\PY{p}{]}\PY{o}{.}\PY{n}{round}\PY{p}{(}\PY{n}{decimals}\PY{o}{=}\PY{l+m+mi}{3}\PY{p}{)}
        \PY{c+c1}{\PYZsh{}heatmap of percentiled value}
        \PY{n}{sns}\PY{o}{.}\PY{n}{heatmap}\PY{p}{(}\PY{n}{data}\PY{o}{=}\PY{n}{percentile\PYZus{}values}\PY{p}{,}\PY{n}{annot}\PY{o}{=}\PY{k+kc}{True}\PY{p}{,}\PY{n}{fmt}\PY{o}{=}\PY{l+s+s1}{\PYZsq{}}\PY{l+s+s1}{.1f}\PY{l+s+s1}{\PYZsq{}}\PY{p}{)}
        \PY{n}{plt}\PY{o}{.}\PY{n}{yticks}\PY{p}{(}\PY{p}{[}\PY{l+m+mf}{0.5}\PY{p}{,}\PY{l+m+mf}{1.5}\PY{p}{,}\PY{l+m+mf}{2.5}\PY{p}{]}\PY{p}{,}\PY{p}{[}\PY{l+s+s1}{\PYZsq{}}\PY{l+s+s1}{Customer 0, Index }\PY{l+s+s1}{\PYZsq{}}\PY{o}{+}\PY{n+nb}{str}\PY{p}{(}\PY{n}{indices}\PY{p}{[}\PY{l+m+mi}{0}\PY{p}{]}\PY{p}{)}\PY{p}{,}\PY{l+s+s1}{\PYZsq{}}\PY{l+s+s1}{Customer 1, Index }\PY{l+s+s1}{\PYZsq{}}\PY{o}{+}\PY{n+nb}{str}\PY{p}{(}\PY{n}{indices}\PY{p}{[}\PY{l+m+mi}{1}\PY{p}{]}\PY{p}{)}\PY{p}{,}\PY{l+s+s1}{\PYZsq{}}\PY{l+s+s1}{Customer 2, Index }\PY{l+s+s1}{\PYZsq{}}\PY{o}{+}\PY{n+nb}{str}\PY{p}{(}\PY{n}{indices}\PY{p}{[}\PY{l+m+mi}{2}\PY{p}{]}\PY{p}{)}\PY{p}{]}\PY{p}{,}\PY{n}{rotation}\PY{o}{=}\PY{l+s+s1}{\PYZsq{}}\PY{l+s+s1}{horizontal}\PY{l+s+s1}{\PYZsq{}}\PY{p}{)}
        \PY{n}{plt}\PY{o}{.}\PY{n}{title}\PY{p}{(}\PY{l+s+s1}{\PYZsq{}}\PY{l+s+s1}{Percentile scores of every value in the sampled data frame}\PY{l+s+s1}{\PYZsq{}}\PY{p}{)}
        
        \PY{n+nb}{print}\PY{p}{(}\PY{l+s+s2}{\PYZdq{}}\PY{l+s+s2}{Chosen samples of wholesale customers dataset:}\PY{l+s+s2}{\PYZdq{}}\PY{p}{)}
        \PY{n}{display}\PY{p}{(}\PY{n}{samples}\PY{p}{)}
\end{Verbatim}


    \begin{Verbatim}[commandchars=\\\{\}]
Chosen samples of wholesale customers dataset:

    \end{Verbatim}

    
    \begin{verbatim}
   Fresh   Milk  Grocery  Frozen  Detergents_Paper  Delicatessen
0  13265   1196     4221    6404               507          1788
1  43265   5025     8117    6312              1579         14351
2   3067  13240    23127    3941              9959           731
    \end{verbatim}

    
    \begin{center}
    \adjustimage{max size={0.9\linewidth}{0.9\paperheight}}{output_8_2.png}
    \end{center}
    { \hspace*{\fill} \\}
    
    \paragraph{Answer:}\label{answer}

\begin{itemize}
\item
  CUSTOMER 0:

  \begin{itemize}
  \tightlist
  \item
    As we can see from the above heatmap, the maximum spending of this
    customer is for Fresh products with a spending of 13265 and score of
    67.7th percentile, then comes the Frozen products with a spending of
    6404 and score of 74.8th percentile and finally delicatessen
    products comes after this with a spending of 1788 and score of
    67.7th percentile. Now even though we have a high percentile value
    for the delicatessen products, the spending in actual money is quite
    less, whereas the spending in actual currency for fresh and frozen
    products is substantially high.
  \item
    One the other hand the spending for other categories like
    milk,grocery and detergents\_papers is below 50th percentile. This
    points points to the fact that 'CUSTOMER 0' might be a small retail
    owner who sells fresh and frozen products with a small side shop
    which provides limited amount of delicatessen products like
    sandwiches etc.
  \end{itemize}
\item
  CUSTOMER 1:

  \begin{itemize}
  \tightlist
  \item
    The above heatmap shows that the spending of 'CUSTOMER 1' is more
    than 60th percentile in all of the categories.
  \item
    Infact 'CUSTOMER 1' spends the maximum amount of money in fresh
    products with spending and percentile score of 43265 and 97.5
    respectively. The second highest spending is in Delicatessen
    products with spendinng and percentile score of 14351 and 99.3
    respectively. The third is frozen products with 6312 money spent
    anually and a percentile score of 87.0. Also the other categories
    like milk, grocery and detergent\_paper have a subtantially high
    percentile score with the scores being 62.3, 67 and 61.4
    respectively.
  \item
    Such a trend can bee seen in big resturants who provide both
    breakfast, lunch and dinner.

    \begin{itemize}
    \tightlist
    \item
      This is because people tend to have coffe, milk products, egg
      dishes etc in breakfast which accounts for the above average
      spending in milk products and grocery.
    \item
      Also since we have predicted it to be a big resturant it will have
      an above average spending in detergents to wash the dishes and
      also above average spending in tissue papers which accounts for
      the affore mentioned percentile score of category
      detergent\_papar.
    \item
      In lunch and dinner people tend to have more things made up of
      vegetables and meats(which are usually frozen), and this is the
      time when people usually tend to come in large numbers. People
      have all sorts of things during lunch and dinner, ranging from all
      vegan meal, salads, meat dishes, fruit dishes etc. This accounts
      for the very high spending in fresh and frozen products.
    \item
      The spending in delicatessen is also high(at a 99.3th percentile)
      which points to the fact that this might be a continental
      resturant which serves a variety of food.
    \end{itemize}
  \end{itemize}
\item
  These facts mentioned above makes us conclude that 'CUSTOMER 1' is
  probably a big famous continental resturant.
\item
  CUSTOMER 2:

  \begin{itemize}
  \tightlist
  \item
    The heat map and table above indicates that the maximum spending is
    done in categories groceries, milk and detergent\_paper with
    spending and percentile score being (23127,94.5\%), (13240,91.8\%),
    (9959,94.1\%) respectively.
  \item
    The spending and percentile score in frozen is above average with
    values (3941,77.7\%) and the spending for fresh and delicatessen
    products is relatively low.
  \item
    Such a trend can be seen in coffee shops or break fast resturants.
    But its more likely that it is a coffee shop.

    \begin{itemize}
    \tightlist
    \item
      The maximum use of milk is done in a cafe. Also we hardly see a
      cafe that sells coffee exclusively. It always or most of the times
      comes with a section where they sell food. This two statistics
      accounts for the high spending in milk and groceries.
    \item
      The other high spending is in detergent and papers. This suggests
      that 'CUSTOMER 2' owns a coffee house that works the entire day or
      maybe 24X7, only then can we justify such high spending in
      detergents and papers. This fact also provides a concrete fact
      that 'CUSTOMER 2' does not own a break-fast resturant as such
      resturants a open only in the morning.
    \item
      The relatively low spending in other categories makes our
      assumption even more concrete that this customer is to a very high
      degree an owner of a cafe.
    \end{itemize}
  \end{itemize}
\end{itemize}

    \subsubsection{Implementation: Feature
Relevance}\label{implementation-feature-relevance}

One interesting thought to consider is if one (or more) of the six
product categories is actually relevant for understanding customer
purchasing. That is to say, is it possible to determine whether
customers purchasing some amount of one category of products will
necessarily purchase some proportional amount of another category of
products? We can make this determination quite easily by training a
supervised regression learner on a subset of the data with one feature
removed, and then score how well that model can predict the removed
feature.

    \begin{Verbatim}[commandchars=\\\{\}]
{\color{incolor}In [{\color{incolor}7}]:} \PY{c+c1}{\PYZsh{} TODO: Make a copy of the DataFrame, using the \PYZsq{}drop\PYZsq{} function to drop the given feature}
        \PY{k+kn}{from} \PY{n+nn}{sklearn}\PY{n+nn}{.}\PY{n+nn}{model\PYZus{}selection} \PY{k}{import} \PY{n}{train\PYZus{}test\PYZus{}split}
        \PY{k+kn}{from} \PY{n+nn}{sklearn}\PY{n+nn}{.}\PY{n+nn}{tree} \PY{k}{import} \PY{n}{DecisionTreeRegressor}
        \PY{n}{target\PYZus{}features} \PY{o}{=} \PY{p}{[}\PY{l+s+s1}{\PYZsq{}}\PY{l+s+s1}{Fresh}\PY{l+s+s1}{\PYZsq{}}\PY{p}{,} \PY{l+s+s1}{\PYZsq{}}\PY{l+s+s1}{Milk}\PY{l+s+s1}{\PYZsq{}}\PY{p}{,} \PY{l+s+s1}{\PYZsq{}}\PY{l+s+s1}{Grocery}\PY{l+s+s1}{\PYZsq{}}\PY{p}{,} \PY{l+s+s1}{\PYZsq{}}\PY{l+s+s1}{Frozen}\PY{l+s+s1}{\PYZsq{}}\PY{p}{,} \PY{l+s+s1}{\PYZsq{}}\PY{l+s+s1}{Detergents\PYZus{}Paper}\PY{l+s+s1}{\PYZsq{}}\PY{p}{,} \PY{l+s+s1}{\PYZsq{}}\PY{l+s+s1}{Delicatessen}\PY{l+s+s1}{\PYZsq{}}\PY{p}{]}
        
        \PY{k}{for} \PY{n}{target} \PY{o+ow}{in} \PY{n}{target\PYZus{}features}\PY{p}{:}
            \PY{n}{y\PYZus{}target} \PY{o}{=} \PY{n}{data}\PY{p}{[}\PY{n}{target}\PY{p}{]}
            \PY{n}{new\PYZus{}data} \PY{o}{=} \PY{n}{data}\PY{o}{.}\PY{n}{drop}\PY{p}{(}\PY{p}{[}\PY{n}{target}\PY{p}{]}\PY{p}{,} \PY{n}{axis} \PY{o}{=} \PY{l+m+mi}{1}\PY{p}{,} \PY{n}{inplace} \PY{o}{=} \PY{k+kc}{False}\PY{p}{)}
        
            \PY{c+c1}{\PYZsh{} TODO: Split the data into training and testing sets using the given feature as the target}
            \PY{n}{X\PYZus{}train}\PY{p}{,} \PY{n}{X\PYZus{}test}\PY{p}{,} \PY{n}{y\PYZus{}train}\PY{p}{,} \PY{n}{y\PYZus{}test} \PY{o}{=} \PY{n}{train\PYZus{}test\PYZus{}split}\PY{p}{(}\PY{n}{new\PYZus{}data}\PY{p}{,} \PY{n}{y\PYZus{}target}\PY{p}{,} \PY{n}{test\PYZus{}size}\PY{o}{=}\PY{l+m+mf}{0.25}\PY{p}{,} \PY{n}{random\PYZus{}state}\PY{o}{=}\PY{l+m+mi}{0}\PY{p}{)}
        
            \PY{c+c1}{\PYZsh{} TODO: Create a decision tree regressor and fit it to the training set}
            
            \PY{n}{regressor} \PY{o}{=} \PY{n}{DecisionTreeRegressor}\PY{p}{(}\PY{n}{random\PYZus{}state}\PY{o}{=}\PY{l+m+mi}{0}\PY{p}{)}
            \PY{n}{regressor}\PY{o}{.}\PY{n}{fit}\PY{p}{(}\PY{n}{X\PYZus{}train}\PY{p}{,} \PY{n}{y\PYZus{}train}\PY{p}{)}
        
            \PY{c+c1}{\PYZsh{} TODO: Report the score of the prediction using the testing set}
            \PY{n}{score} \PY{o}{=} \PY{n}{regressor}\PY{o}{.}\PY{n}{score}\PY{p}{(}\PY{n}{X\PYZus{}test}\PY{p}{,} \PY{n}{y\PYZus{}test}\PY{p}{)}
            \PY{n+nb}{print} \PY{p}{(}\PY{l+s+s2}{\PYZdq{}}\PY{l+s+s2}{score for target feature }\PY{l+s+s2}{\PYZdq{}}\PY{p}{,}\PY{n}{target}\PY{p}{,}\PY{l+s+s2}{\PYZdq{}}\PY{l+s+s2}{ is }\PY{l+s+s2}{\PYZdq{}}\PY{p}{,}\PY{n+nb}{str}\PY{p}{(}\PY{n}{score}\PY{p}{)}\PY{p}{)}
\end{Verbatim}


    \begin{Verbatim}[commandchars=\\\{\}]
score for target feature  Fresh  is  -0.252469807688
score for target feature  Milk  is  0.365725292736
score for target feature  Grocery  is  0.602801978878
score for target feature  Frozen  is  0.253973446697
score for target feature  Detergents\_Paper  is  0.728655181254
score for target feature  Delicatessen  is  -11.6636871594

    \end{Verbatim}

    \subsubsection{Question}\label{question}

\begin{itemize}
\tightlist
\item
  Which feature did we attempt to predict?
\item
  What was the reported prediction score?
\item
  Is this feature necessary for identifying customers' spending habits?
\end{itemize}

The coefficient of determination, \texttt{R\^{}2}, is scored between 0
and 1, with 1 being a perfect fit. A negative \texttt{R\^{}2} implies
the model fails to fit the data. If we get a low score for a particular
feature, that lends us to beleive that that feature point is hard to
predict using the other features, thereby making it an important feature
to consider when considering relevance.

    \textbf{Answer:} * Since there are just features to consider, we
selected all the features as target variables one by one and their
respective scores are as mentioned above. * The reported prediction
scores are:

\begin{longtable}[]{@{}cc@{}}
\toprule
target feature & score\tabularnewline
\midrule
\endhead
Fresh & -0.2525\tabularnewline
Milk & 0.3657\tabularnewline
Grocery & 0.6028\tabularnewline
Frozen & 0.2539\tabularnewline
Detergent\_paper & 0.7287\tabularnewline
Delicatessen & -11.6637\tabularnewline
\bottomrule
\end{longtable}

\begin{itemize}
\tightlist
\item
  The feature necessary fro identifying customer segments are selected
  as follows:

  \begin{itemize}
  \tightlist
  \item
    As shown in the above table there are two target variables 'Fresh'
    and 'Delicatessen' that have an \texttt{R\^{}2} score which is less
    than zero. This means that the model fails to fit the data with the
    when 'Fresh' and 'Delicatessen' are taken as target variables.
  \item
    On the other hand the two features 'Grocery' and 'Detergent\_paper'
    have high \texttt{R\^{}2} score which means that there is a very
    high correlation between 'Grocery' and 'Detergent\_paper' and other
    features. Hence 'Grocery' can be predicted using other features.
    Thus these features do not provide good insight for identifying
    customers' spending segment.
  \item
    However there are two features 'Milk' and 'Frozen' which when used
    as target variable have very little \texttt{R\^{}2} which points to
    the fact that these feature do not have much correlation with other
    features and as a result of this, these features can be used for
    identifying customers' spending habits.
  \end{itemize}
\end{itemize}

    \subsubsection{Visualize Feature
Distributions}\label{visualize-feature-distributions}

To get a better understanding of the dataset, we can construct a scatter
matrix of each of the six product features present in the data. If you
found that the feature you attempted to predict above is relevant for
identifying a specific customer, then the scatter matrix below may not
show any correlation between that feature and the others. Conversely, if
you believe that feature is not relevant for identifying a specific
customer, the scatter matrix might show a correlation between that
feature and another feature in the data.

    \begin{Verbatim}[commandchars=\\\{\}]
{\color{incolor}In [{\color{incolor}8}]:} \PY{c+c1}{\PYZsh{} Produce a scatter matrix for each pair of features in the data}
        \PY{n}{pd}\PY{o}{.}\PY{n}{plotting}\PY{o}{.}\PY{n}{scatter\PYZus{}matrix}\PY{p}{(}\PY{n}{data}\PY{p}{,} \PY{n}{alpha} \PY{o}{=} \PY{l+m+mf}{0.3}\PY{p}{,} \PY{n}{figsize} \PY{o}{=} \PY{p}{(}\PY{l+m+mi}{14}\PY{p}{,}\PY{l+m+mi}{8}\PY{p}{)}\PY{p}{,} \PY{n}{diagonal} \PY{o}{=} \PY{l+s+s1}{\PYZsq{}}\PY{l+s+s1}{kde}\PY{l+s+s1}{\PYZsq{}}\PY{p}{)}\PY{p}{;}
\end{Verbatim}


    \begin{center}
    \adjustimage{max size={0.9\linewidth}{0.9\paperheight}}{output_15_0.png}
    \end{center}
    { \hspace*{\fill} \\}
    
    \subsubsection{Question}\label{question}

\begin{itemize}
\tightlist
\item
  Using the scatter matrix as a reference, discuss the distribution of
  the dataset, specifically talk about the normality, outliers, large
  number of data points near 0 among others. If you need to sepearate
  out some of the plots individually to further accentuate your point,
  you may do so as well.
\item
  Are there any pairs of features which exhibit some degree of
  correlation?
\item
  Does this confirm or deny your suspicions about the relevance of the
  feature you attempted to predict?
\item
  How is the data for those features distributed?
\end{itemize}

    \begin{Verbatim}[commandchars=\\\{\}]
{\color{incolor}In [{\color{incolor}9}]:} \PY{n}{sns}\PY{o}{.}\PY{n}{heatmap}\PY{p}{(}\PY{n}{data}\PY{o}{=}\PY{n}{data}\PY{o}{.}\PY{n}{corr}\PY{p}{(}\PY{p}{)}\PY{p}{,}\PY{n}{annot}\PY{o}{=}\PY{k+kc}{True}\PY{p}{,}\PY{n}{fmt}\PY{o}{=}\PY{l+s+s1}{\PYZsq{}}\PY{l+s+s1}{.1f}\PY{l+s+s1}{\PYZsq{}}\PY{p}{)}
        \PY{n}{plt}\PY{o}{.}\PY{n}{title}\PY{p}{(}\PY{l+s+s1}{\PYZsq{}}\PY{l+s+s1}{Correlation heatmap of features}\PY{l+s+s1}{\PYZsq{}}\PY{p}{)}
        \PY{n}{plt}\PY{o}{.}\PY{n}{show}\PY{p}{(}\PY{p}{)}
\end{Verbatim}


    \begin{center}
    \adjustimage{max size={0.9\linewidth}{0.9\paperheight}}{output_17_0.png}
    \end{center}
    { \hspace*{\fill} \\}
    
    \textbf{Answer:} * From the graphs above we can say that most of the
data points are concenterated near zero i.e below the mean. And as a
result of this almost all the distributions are right skewed. Also since
the distributions are right skewed there are outliers and they will
always be toward the right side i.e the customers who spend extrems
amounts of money will be very less. * We are going to neglect the
features 'Fresh' and 'Delicatessen' as the \texttt{R\^{}2} score was
negative which points to the fact that those two features do not fit the
model and are useless. - Thus we concenterate on the other 4
features('Milk','Frozen','Grocery' and 'Detergents\_Paper'). - Take a
look at the heat-map shown above in whcih each cell represents the
correlation of the feature corresponding to the particular row and
column. (We won't be considering the auto-correlation of features when
we look at the heat map above, we will only be considering the
cross-correlations.).

\begin{verbatim}
 - As we can see, the feature 'Milk' has highest correlation with features 'Grocery' and 'Detergent_Papers' and has the lowest correlation with 'Frozen'.
 - The feature 'Grocery' has the maximum correlation with feature 'detergents_Paper' and minimum correlation with 'Frozen'.
 - 'Frozen' has minimum correlation with 'Milk'.
 
 - Thus we included all the other four features we are interested in. Now we will take a look at the scatter plots of the above mentioned four features in more detail.
\end{verbatim}

    \begin{Verbatim}[commandchars=\\\{\}]
{\color{incolor}In [{\color{incolor}10}]:} \PY{n}{f}\PY{p}{,} \PY{p}{(}\PY{p}{(}\PY{n}{ax1}\PY{p}{,} \PY{n}{ax2}\PY{p}{)}\PY{p}{,} \PY{p}{(}\PY{n}{ax3}\PY{p}{,} \PY{n}{ax4}\PY{p}{)}\PY{p}{)} \PY{o}{=} \PY{n}{plt}\PY{o}{.}\PY{n}{subplots}\PY{p}{(}\PY{l+m+mi}{2}\PY{p}{,} \PY{l+m+mi}{2}\PY{p}{,}\PY{n}{figsize}\PY{o}{=}\PY{p}{(}\PY{l+m+mi}{12}\PY{p}{,}\PY{l+m+mi}{12}\PY{p}{)}\PY{p}{)}
         \PY{n}{f}\PY{o}{.}\PY{n}{suptitle}\PY{p}{(}\PY{l+s+s1}{\PYZsq{}}\PY{l+s+s1}{The magnified subplots of useful variables}\PY{l+s+s1}{\PYZsq{}}\PY{p}{)}
         \PY{n}{ax1}\PY{o}{.}\PY{n}{scatter}\PY{p}{(}\PY{n}{np}\PY{o}{.}\PY{n}{array}\PY{p}{(}\PY{n}{data}\PY{p}{[}\PY{l+s+s1}{\PYZsq{}}\PY{l+s+s1}{Grocery}\PY{l+s+s1}{\PYZsq{}}\PY{p}{]}\PY{p}{)}\PY{p}{,}\PY{n}{np}\PY{o}{.}\PY{n}{array}\PY{p}{(}\PY{n}{data}\PY{p}{[}\PY{l+s+s1}{\PYZsq{}}\PY{l+s+s1}{Milk}\PY{l+s+s1}{\PYZsq{}}\PY{p}{]}\PY{p}{)}\PY{p}{,}\PY{n}{alpha}\PY{o}{=}\PY{l+m+mf}{0.2}\PY{p}{,}\PY{n}{edgecolors}\PY{o}{=}\PY{l+s+s1}{\PYZsq{}}\PY{l+s+s1}{purple}\PY{l+s+s1}{\PYZsq{}}\PY{p}{)}
         \PY{n}{ax1}\PY{o}{.}\PY{n}{set}\PY{p}{(}\PY{n}{xlabel}\PY{o}{=}\PY{l+s+s1}{\PYZsq{}}\PY{l+s+s1}{Grocery}\PY{l+s+s1}{\PYZsq{}}\PY{p}{,} \PY{n}{ylabel}\PY{o}{=}\PY{l+s+s1}{\PYZsq{}}\PY{l+s+s1}{Milk}\PY{l+s+s1}{\PYZsq{}}\PY{p}{)}
         \PY{n}{ax2}\PY{o}{.}\PY{n}{scatter}\PY{p}{(}\PY{n}{np}\PY{o}{.}\PY{n}{array}\PY{p}{(}\PY{n}{data}\PY{p}{[}\PY{l+s+s1}{\PYZsq{}}\PY{l+s+s1}{Frozen}\PY{l+s+s1}{\PYZsq{}}\PY{p}{]}\PY{p}{)}\PY{p}{,}\PY{n}{np}\PY{o}{.}\PY{n}{array}\PY{p}{(}\PY{n}{data}\PY{p}{[}\PY{l+s+s1}{\PYZsq{}}\PY{l+s+s1}{Milk}\PY{l+s+s1}{\PYZsq{}}\PY{p}{]}\PY{p}{)}\PY{p}{,}\PY{n}{c}\PY{o}{=}\PY{l+s+s1}{\PYZsq{}}\PY{l+s+s1}{orange}\PY{l+s+s1}{\PYZsq{}}\PY{p}{,}\PY{n}{alpha}\PY{o}{=}\PY{l+m+mf}{0.2}\PY{p}{,}\PY{n}{edgecolors}\PY{o}{=}\PY{l+s+s1}{\PYZsq{}}\PY{l+s+s1}{red}\PY{l+s+s1}{\PYZsq{}}\PY{p}{)}
         \PY{n}{ax2}\PY{o}{.}\PY{n}{set}\PY{p}{(}\PY{n}{xlabel}\PY{o}{=}\PY{l+s+s1}{\PYZsq{}}\PY{l+s+s1}{Frozen}\PY{l+s+s1}{\PYZsq{}}\PY{p}{,} \PY{n}{ylabel}\PY{o}{=}\PY{l+s+s1}{\PYZsq{}}\PY{l+s+s1}{Milk}\PY{l+s+s1}{\PYZsq{}}\PY{p}{)}
         \PY{n}{ax3}\PY{o}{.}\PY{n}{scatter}\PY{p}{(}\PY{n}{np}\PY{o}{.}\PY{n}{array}\PY{p}{(}\PY{n}{data}\PY{p}{[}\PY{l+s+s1}{\PYZsq{}}\PY{l+s+s1}{Detergents\PYZus{}Paper}\PY{l+s+s1}{\PYZsq{}}\PY{p}{]}\PY{p}{)}\PY{p}{,}\PY{n}{np}\PY{o}{.}\PY{n}{array}\PY{p}{(}\PY{n}{data}\PY{p}{[}\PY{l+s+s1}{\PYZsq{}}\PY{l+s+s1}{Milk}\PY{l+s+s1}{\PYZsq{}}\PY{p}{]}\PY{p}{)}\PY{p}{,}\PY{n}{c}\PY{o}{=}\PY{l+s+s1}{\PYZsq{}}\PY{l+s+s1}{red}\PY{l+s+s1}{\PYZsq{}}\PY{p}{,}\PY{n}{alpha}\PY{o}{=}\PY{l+m+mf}{0.2}\PY{p}{,}\PY{n}{edgecolors}\PY{o}{=}\PY{l+s+s1}{\PYZsq{}}\PY{l+s+s1}{green}\PY{l+s+s1}{\PYZsq{}}\PY{p}{)}
         \PY{n}{ax3}\PY{o}{.}\PY{n}{set}\PY{p}{(}\PY{n}{xlabel}\PY{o}{=}\PY{l+s+s1}{\PYZsq{}}\PY{l+s+s1}{Detergents\PYZus{}Paper}\PY{l+s+s1}{\PYZsq{}}\PY{p}{,} \PY{n}{ylabel}\PY{o}{=}\PY{l+s+s1}{\PYZsq{}}\PY{l+s+s1}{Milk}\PY{l+s+s1}{\PYZsq{}}\PY{p}{)}
         \PY{n}{ax4}\PY{o}{.}\PY{n}{scatter}\PY{p}{(}\PY{n}{np}\PY{o}{.}\PY{n}{array}\PY{p}{(}\PY{n}{data}\PY{p}{[}\PY{l+s+s1}{\PYZsq{}}\PY{l+s+s1}{Detergents\PYZus{}Paper}\PY{l+s+s1}{\PYZsq{}}\PY{p}{]}\PY{p}{)}\PY{p}{,}\PY{n}{np}\PY{o}{.}\PY{n}{array}\PY{p}{(}\PY{n}{data}\PY{p}{[}\PY{l+s+s1}{\PYZsq{}}\PY{l+s+s1}{Grocery}\PY{l+s+s1}{\PYZsq{}}\PY{p}{]}\PY{p}{)}\PY{p}{,}\PY{n}{c}\PY{o}{=}\PY{l+s+s1}{\PYZsq{}}\PY{l+s+s1}{purple}\PY{l+s+s1}{\PYZsq{}}\PY{p}{,}\PY{n}{alpha}\PY{o}{=}\PY{l+m+mf}{0.2}\PY{p}{,}\PY{n}{edgecolors}\PY{o}{=}\PY{l+s+s1}{\PYZsq{}}\PY{l+s+s1}{blue}\PY{l+s+s1}{\PYZsq{}}\PY{p}{)}
         \PY{n}{ax4}\PY{o}{.}\PY{n}{set}\PY{p}{(}\PY{n}{xlabel}\PY{o}{=}\PY{l+s+s1}{\PYZsq{}}\PY{l+s+s1}{Detergents\PYZus{}Paper}\PY{l+s+s1}{\PYZsq{}}\PY{p}{,} \PY{n}{ylabel}\PY{o}{=}\PY{l+s+s1}{\PYZsq{}}\PY{l+s+s1}{Grocery}\PY{l+s+s1}{\PYZsq{}}\PY{p}{)}
         \PY{n}{plt}\PY{o}{.}\PY{n}{show}\PY{p}{(}\PY{p}{)}
         
         \PY{c+c1}{\PYZsh{} Create some normally distributed data}
         \PY{n}{mean} \PY{o}{=} \PY{p}{[}\PY{l+m+mi}{0}\PY{p}{,} \PY{l+m+mi}{0}\PY{p}{]}
         \PY{n}{cov} \PY{o}{=} \PY{p}{[}\PY{p}{[}\PY{l+m+mi}{1}\PY{p}{,} \PY{l+m+mi}{1}\PY{p}{]}\PY{p}{,} \PY{p}{[}\PY{l+m+mi}{1}\PY{p}{,} \PY{l+m+mi}{2}\PY{p}{]}\PY{p}{]}
         \PY{n}{x}\PY{p}{,} \PY{n}{y} \PY{o}{=} \PY{n}{np}\PY{o}{.}\PY{n}{random}\PY{o}{.}\PY{n}{multivariate\PYZus{}normal}\PY{p}{(}\PY{n}{mean}\PY{p}{,} \PY{n}{cov}\PY{p}{,} \PY{l+m+mi}{3000}\PY{p}{)}\PY{o}{.}\PY{n}{T}
         
         \PY{c+c1}{\PYZsh{} Set up the axes with gridspec}
         \PY{n}{fig} \PY{o}{=} \PY{n}{plt}\PY{o}{.}\PY{n}{figure}\PY{p}{(}\PY{n}{figsize}\PY{o}{=}\PY{p}{(}\PY{l+m+mi}{6}\PY{p}{,} \PY{l+m+mi}{6}\PY{p}{)}\PY{p}{)}
         \PY{n}{grid} \PY{o}{=} \PY{n}{plt}\PY{o}{.}\PY{n}{GridSpec}\PY{p}{(}\PY{l+m+mi}{4}\PY{p}{,} \PY{l+m+mi}{4}\PY{p}{,} \PY{n}{hspace}\PY{o}{=}\PY{l+m+mf}{0.2}\PY{p}{,} \PY{n}{wspace}\PY{o}{=}\PY{l+m+mf}{0.2}\PY{p}{)}
         \PY{n}{main\PYZus{}ax} \PY{o}{=} \PY{n}{fig}\PY{o}{.}\PY{n}{add\PYZus{}subplot}\PY{p}{(}\PY{n}{grid}\PY{p}{[}\PY{p}{:}\PY{o}{\PYZhy{}}\PY{l+m+mi}{1}\PY{p}{,} \PY{l+m+mi}{1}\PY{p}{:}\PY{p}{]}\PY{p}{)}
         \PY{n}{y\PYZus{}hist} \PY{o}{=} \PY{n}{fig}\PY{o}{.}\PY{n}{add\PYZus{}subplot}\PY{p}{(}\PY{n}{grid}\PY{p}{[}\PY{p}{:}\PY{o}{\PYZhy{}}\PY{l+m+mi}{1}\PY{p}{,} \PY{l+m+mi}{0}\PY{p}{]}\PY{p}{,} \PY{n}{xticklabels}\PY{o}{=}\PY{p}{[}\PY{p}{]}\PY{p}{,} \PY{n}{sharey}\PY{o}{=}\PY{n}{main\PYZus{}ax}\PY{p}{)}
         \PY{n}{x\PYZus{}hist} \PY{o}{=} \PY{n}{fig}\PY{o}{.}\PY{n}{add\PYZus{}subplot}\PY{p}{(}\PY{n}{grid}\PY{p}{[}\PY{o}{\PYZhy{}}\PY{l+m+mi}{1}\PY{p}{,} \PY{l+m+mi}{1}\PY{p}{:}\PY{p}{]}\PY{p}{,} \PY{n}{yticklabels}\PY{o}{=}\PY{p}{[}\PY{p}{]}\PY{p}{,} \PY{n}{sharex}\PY{o}{=}\PY{n}{main\PYZus{}ax}\PY{p}{)}
         
         \PY{c+c1}{\PYZsh{} scatter points on the main axes}
         \PY{n}{main\PYZus{}ax}\PY{o}{.}\PY{n}{plot}\PY{p}{(}\PY{n}{x}\PY{p}{,} \PY{n}{y}\PY{p}{,} \PY{l+s+s1}{\PYZsq{}}\PY{l+s+s1}{ok}\PY{l+s+s1}{\PYZsq{}}\PY{p}{,} \PY{n}{markersize}\PY{o}{=}\PY{l+m+mi}{3}\PY{p}{,} \PY{n}{alpha}\PY{o}{=}\PY{l+m+mf}{0.2}\PY{p}{)}
         
         \PY{c+c1}{\PYZsh{} histogram on the attached axes}
         \PY{n}{x\PYZus{}hist}\PY{o}{.}\PY{n}{hist}\PY{p}{(}\PY{n}{x}\PY{p}{,} \PY{l+m+mi}{40}\PY{p}{,} \PY{n}{histtype}\PY{o}{=}\PY{l+s+s1}{\PYZsq{}}\PY{l+s+s1}{stepfilled}\PY{l+s+s1}{\PYZsq{}}\PY{p}{,}
                     \PY{n}{orientation}\PY{o}{=}\PY{l+s+s1}{\PYZsq{}}\PY{l+s+s1}{vertical}\PY{l+s+s1}{\PYZsq{}}\PY{p}{,} \PY{n}{color}\PY{o}{=}\PY{l+s+s1}{\PYZsq{}}\PY{l+s+s1}{blue}\PY{l+s+s1}{\PYZsq{}}\PY{p}{)}
         \PY{n}{x\PYZus{}hist}\PY{o}{.}\PY{n}{invert\PYZus{}yaxis}\PY{p}{(}\PY{p}{)}
         
         \PY{n}{y\PYZus{}hist}\PY{o}{.}\PY{n}{hist}\PY{p}{(}\PY{n}{y}\PY{p}{,} \PY{l+m+mi}{40}\PY{p}{,} \PY{n}{histtype}\PY{o}{=}\PY{l+s+s1}{\PYZsq{}}\PY{l+s+s1}{stepfilled}\PY{l+s+s1}{\PYZsq{}}\PY{p}{,}
                     \PY{n}{orientation}\PY{o}{=}\PY{l+s+s1}{\PYZsq{}}\PY{l+s+s1}{horizontal}\PY{l+s+s1}{\PYZsq{}}\PY{p}{,} \PY{n}{color}\PY{o}{=}\PY{l+s+s1}{\PYZsq{}}\PY{l+s+s1}{red}\PY{l+s+s1}{\PYZsq{}}\PY{p}{)}
         \PY{n}{y\PYZus{}hist}\PY{o}{.}\PY{n}{invert\PYZus{}xaxis}\PY{p}{(}\PY{p}{)}
\end{Verbatim}


    \begin{center}
    \adjustimage{max size={0.9\linewidth}{0.9\paperheight}}{output_19_0.png}
    \end{center}
    { \hspace*{\fill} \\}
    
    \begin{center}
    \adjustimage{max size={0.9\linewidth}{0.9\paperheight}}{output_19_1.png}
    \end{center}
    { \hspace*{\fill} \\}
    
    \begin{itemize}
\item
  The first plot tells us that 'Grocery' and milk are correlated to some
  extent as with the increase in value of 'Grocery' the value of 'Milk'
  tends to increase. The correlation value is 0.7 which is pretty high.
  Thus if we know value of one of the features we can predict the value
  of the other one.

  \begin{itemize}
  \tightlist
  \item
    Also since \texttt{R\^{}2} score of 'Milk' is less as compared to
    'Grocery' we can neglect the 'Grocery' feature and use the feature
    'Milk' to create customer segments.
  \end{itemize}
\item
  The second plot is the plot of 'Milk' and 'Frozen' and apparently the
  values are spread throughout the space. This is verified by the
  correlation coffecient between them being 0.1. Such small correlation
  coffecient and a small \texttt{R\^{}2} score suggest that Frozen is
  indeed an important variable to create customer segments.
\item
  The third plot suggest that there is a good correlation between 'Milk'
  and 'Detergents\_Paper' and hence as the reasons provided in the
  points above we can neglect 'Detergents\_Paper'.
\item
  The fourth graph is the most important graph. We might argue that
  since the correlation between 'Frozen' and 'Grocery', and 'Frozen' and
  'Detergents\_Paper' is zero it would be wise to include 'Grocery' and
  'Detergents\_Paper' as a feature to create customer segments. However
  we can provide two counter arguments to this which are as follows:

  \begin{itemize}
  \tightlist
  \item
    The \texttt{R\^{}2} score that we calculated before was high for
    'Grocery' as well as 'Detergents\_Papers' which indicated that the
    above two features can be predicted using the other features. Thus
    neglecting these two features is a good choice.
  \item
    The other reason is that if we look at the fourth graph between
    'Grocery' and 'Detergents\_Paper' it seems that both the values come
    from a normal distribution with almost the same mean and standard
    deviation\texttt{(for\ better\ visualization\ look\ at\ the\ last\ graph)}
    and this means that knowing the value of one helps us preict the
    value of other. Also in the first point we concluded that since
    there is a correlation between 'Milk' and 'Grocery' we can drop
    'Grocery' as knowing the value of 'Milk' will help us predict the
    value of 'Grocery' and owing to this fact we can also drop
    'Detergents\_Paper' as it comes from the same distribution as that
    of 'Grocery'.
  \end{itemize}
\end{itemize}

    \subsection{Data Preprocessing}\label{data-preprocessing}

In this section, we will preprocess the data to create a better
representation of customers by performing a scaling on the data and
detecting (and optionally removing) outliers. Preprocessing data is
often times a critical step in assuring that results we obtain from your
analysis are significant and meaningful.

    \subsubsection{Implementation: Feature
Scaling}\label{implementation-feature-scaling}

If data is not normally distributed, especially if the mean and median
vary significantly (indicating a large skew), it is most
\href{http://econbrowser.com/archives/2014/02/use-of-logarithms-in-economics}{often
appropriate} to apply a non-linear scaling --- particularly for
financial data. One way to achieve this scaling is by using a
\href{http://scipy.github.io/devdocs/generated/scipy.stats.boxcox.html}{Box-Cox
test}, which calculates the best power transformation of the data that
reduces skewness. A simpler approach which can work in most cases would
be applying the natural logarithm.

    \begin{Verbatim}[commandchars=\\\{\}]
{\color{incolor}In [{\color{incolor}11}]:} \PY{c+c1}{\PYZsh{} TODO: Scale the data using the natural logarithm}
         \PY{n}{log\PYZus{}data} \PY{o}{=} \PY{n}{np}\PY{o}{.}\PY{n}{log}\PY{p}{(}\PY{n}{data}\PY{p}{)}
         
         \PY{c+c1}{\PYZsh{} TODO: Scale the sample data using the natural logarithm}
         \PY{n}{log\PYZus{}samples} \PY{o}{=} \PY{n}{np}\PY{o}{.}\PY{n}{log}\PY{p}{(}\PY{n}{samples}\PY{p}{)}
         
         \PY{c+c1}{\PYZsh{} Produce a scatter matrix for each pair of newly\PYZhy{}transformed features}
         \PY{n}{pd}\PY{o}{.}\PY{n}{plotting}\PY{o}{.}\PY{n}{scatter\PYZus{}matrix}\PY{p}{(}\PY{n}{log\PYZus{}data}\PY{p}{,} \PY{n}{alpha} \PY{o}{=} \PY{l+m+mf}{0.3}\PY{p}{,} \PY{n}{figsize} \PY{o}{=} \PY{p}{(}\PY{l+m+mi}{14}\PY{p}{,}\PY{l+m+mi}{8}\PY{p}{)}\PY{p}{,} \PY{n}{diagonal} \PY{o}{=} \PY{l+s+s1}{\PYZsq{}}\PY{l+s+s1}{kde}\PY{l+s+s1}{\PYZsq{}}\PY{p}{)}\PY{p}{;}
         \PY{n}{plt}\PY{o}{.}\PY{n}{show}\PY{p}{(}\PY{p}{)}
         
         \PY{n}{sns}\PY{o}{.}\PY{n}{heatmap}\PY{p}{(}\PY{n}{data}\PY{o}{=}\PY{n}{log\PYZus{}data}\PY{o}{.}\PY{n}{corr}\PY{p}{(}\PY{p}{)}\PY{p}{,}\PY{n}{annot}\PY{o}{=}\PY{k+kc}{True}\PY{p}{,}\PY{n}{fmt}\PY{o}{=}\PY{l+s+s1}{\PYZsq{}}\PY{l+s+s1}{.1f}\PY{l+s+s1}{\PYZsq{}}\PY{p}{)}
         \PY{n}{plt}\PY{o}{.}\PY{n}{title}\PY{p}{(}\PY{l+s+s1}{\PYZsq{}}\PY{l+s+s1}{Correlation heatmap of features after scaling the data using natural logarithm}\PY{l+s+s1}{\PYZsq{}}\PY{p}{)}
         \PY{n}{plt}\PY{o}{.}\PY{n}{show}\PY{p}{(}\PY{p}{)}
\end{Verbatim}


    \begin{center}
    \adjustimage{max size={0.9\linewidth}{0.9\paperheight}}{output_23_0.png}
    \end{center}
    { \hspace*{\fill} \\}
    
    \begin{center}
    \adjustimage{max size={0.9\linewidth}{0.9\paperheight}}{output_23_1.png}
    \end{center}
    { \hspace*{\fill} \\}
    
    \subsubsection{Observation}\label{observation}

After applying a natural logarithm scaling to the data, the distribution
of each feature should appear much more normal. For any pairs of
features we may have identified earlier as being correlated, observe
here whether that correlation is still present (and whether it is now
stronger or weaker than before).

    \begin{Verbatim}[commandchars=\\\{\}]
{\color{incolor}In [{\color{incolor}12}]:} \PY{c+c1}{\PYZsh{} Display the log\PYZhy{}transformed sample data}
         \PY{n}{display}\PY{p}{(}\PY{n}{log\PYZus{}samples}\PY{p}{)}
\end{Verbatim}


    
    \begin{verbatim}
       Fresh      Milk    Grocery    Frozen  Detergents_Paper  Delicatessen
0   9.492884  7.086738   8.347827  8.764678          6.228511      7.488853
1  10.675099  8.522181   9.001716  8.750208          7.364547      9.571575
2   8.028455  9.490998  10.048756  8.279190          9.206232      6.594413
    \end{verbatim}

    
    \subsubsection{Implementation: Outlier
Detection}\label{implementation-outlier-detection}

Detecting outliers in the data is extremely important in the data
preprocessing step of any analysis. The presence of outliers can often
skew results which take into consideration these data points. There are
many "rules of thumb" for what constitutes an outlier in a dataset.
Here, we will use
\href{http://datapigtechnologies.com/blog/index.php/highlighting-outliers-in-your-data-with-the-tukey-method/}{Tukey's
Method for identfying outliers}: An \emph{outlier step} is calculated as
1.5 times the interquartile range (IQR). A data point with a feature
that is beyond an outlier step outside of the IQR for that feature is
considered abnormal.

    \begin{Verbatim}[commandchars=\\\{\}]
{\color{incolor}In [{\color{incolor}13}]:} \PY{c+c1}{\PYZsh{} For each feature find the data points with extreme high or low values}
         \PY{n}{feature\PYZus{}out\PYZus{}all} \PY{o}{=} \PY{p}{\PYZob{}}\PY{p}{\PYZcb{}}
         \PY{n}{all\PYZus{}outliers} \PY{o}{=} \PY{n+nb}{set}\PY{p}{(}\PY{p}{)}
         \PY{k}{for} \PY{n}{feature} \PY{o+ow}{in} \PY{n}{log\PYZus{}data}\PY{o}{.}\PY{n}{keys}\PY{p}{(}\PY{p}{)}\PY{p}{:}
             
             \PY{c+c1}{\PYZsh{} TODO: Calculate Q1 (25th percentile of the data) for the given feature}
             \PY{n}{Q1} \PY{o}{=} \PY{n}{np}\PY{o}{.}\PY{n}{percentile}\PY{p}{(}\PY{n}{log\PYZus{}data}\PY{p}{[}\PY{n}{feature}\PY{p}{]}\PY{p}{,} \PY{l+m+mi}{25}\PY{p}{)}
             
             \PY{c+c1}{\PYZsh{} TODO: Calculate Q3 (75th percentile of the data) for the given feature}
             \PY{n}{Q3} \PY{o}{=} \PY{n}{np}\PY{o}{.}\PY{n}{percentile}\PY{p}{(}\PY{n}{log\PYZus{}data}\PY{p}{[}\PY{n}{feature}\PY{p}{]}\PY{p}{,} \PY{l+m+mi}{75}\PY{p}{)}
             
             \PY{c+c1}{\PYZsh{} TODO: Use the interquartile range to calculate an outlier step (1.5 times the interquartile range)}
             \PY{n}{step} \PY{o}{=} \PY{p}{(}\PY{n}{Q3} \PY{o}{\PYZhy{}} \PY{n}{Q1}\PY{p}{)}\PY{o}{*}\PY{l+m+mf}{1.5}
             
             \PY{c+c1}{\PYZsh{} Display the outliers}
             \PY{n+nb}{print} \PY{p}{(}\PY{l+s+s2}{\PYZdq{}}\PY{l+s+s2}{Data points considered outliers for the feature }\PY{l+s+s2}{\PYZsq{}}\PY{l+s+si}{\PYZob{}\PYZcb{}}\PY{l+s+s2}{\PYZsq{}}\PY{l+s+s2}{:}\PY{l+s+s2}{\PYZdq{}}\PY{o}{.}\PY{n}{format}\PY{p}{(}\PY{n}{feature}\PY{p}{)}\PY{p}{)}
             \PY{n}{display}\PY{p}{(}\PY{n}{log\PYZus{}data}\PY{p}{[}\PY{o}{\PYZti{}}\PY{p}{(}\PY{p}{(}\PY{n}{log\PYZus{}data}\PY{p}{[}\PY{n}{feature}\PY{p}{]} \PY{o}{\PYZgt{}}\PY{o}{=} \PY{n}{Q1} \PY{o}{\PYZhy{}} \PY{n}{step}\PY{p}{)} \PY{o}{\PYZam{}} \PY{p}{(}\PY{n}{log\PYZus{}data}\PY{p}{[}\PY{n}{feature}\PY{p}{]} \PY{o}{\PYZlt{}}\PY{o}{=} \PY{n}{Q3} \PY{o}{+} \PY{n}{step}\PY{p}{)}\PY{p}{)}\PY{p}{]}\PY{p}{)}
             \PY{n}{current\PYZus{}feat\PYZus{}outliers} \PY{o}{=} \PY{n+nb}{list}\PY{p}{(}\PY{p}{(}\PY{n}{log\PYZus{}data}\PY{o}{.}\PY{n}{index}\PY{p}{[}\PY{o}{\PYZti{}}\PY{p}{(}\PY{p}{(}\PY{n}{log\PYZus{}data}\PY{p}{[}\PY{n}{feature}\PY{p}{]} \PY{o}{\PYZgt{}}\PY{o}{=} \PY{n}{Q1} \PY{o}{\PYZhy{}} \PY{n}{step}\PY{p}{)} \PY{o}{\PYZam{}} \PY{p}{(}\PY{n}{log\PYZus{}data}\PY{p}{[}\PY{n}{feature}\PY{p}{]} \PY{o}{\PYZlt{}}\PY{o}{=} \PY{n}{Q3} \PY{o}{+} \PY{n}{step}\PY{p}{)}\PY{p}{)}\PY{p}{]}\PY{p}{)}\PY{p}{)}
             \PY{n}{all\PYZus{}outliers} \PY{o}{=} \PY{n}{all\PYZus{}outliers}\PY{o}{.}\PY{n}{union}\PY{p}{(}\PY{n+nb}{set}\PY{p}{(}\PY{n}{current\PYZus{}feat\PYZus{}outliers}\PY{p}{)}\PY{p}{)}
             \PY{k}{for} \PY{n}{index} \PY{o+ow}{in} \PY{n}{current\PYZus{}feat\PYZus{}outliers}\PY{p}{:}
                 \PY{k}{if}\PY{p}{(}\PY{n}{feature\PYZus{}out\PYZus{}all}\PY{o}{.}\PY{n}{get}\PY{p}{(}\PY{n}{index}\PY{p}{)} \PY{o}{==} \PY{k+kc}{None}\PY{p}{)}\PY{p}{:}
                     \PY{n}{feature\PYZus{}out\PYZus{}all}\PY{p}{[}\PY{n}{index}\PY{p}{]} \PY{o}{=} \PY{p}{[}\PY{n}{feature}\PY{p}{]}
                 \PY{k}{else}\PY{p}{:}
                     \PY{n}{feature\PYZus{}out\PYZus{}all}\PY{p}{[}\PY{n}{index}\PY{p}{]}\PY{o}{.}\PY{n}{append}\PY{p}{(}\PY{n}{feature}\PY{p}{)}
             
         
         \PY{n}{outliers} \PY{o}{=} \PY{p}{[}\PY{p}{]}
         
         \PY{k}{for} \PY{n}{key} \PY{o+ow}{in} \PY{n}{feature\PYZus{}out\PYZus{}all}\PY{p}{:}
             \PY{k}{if}\PY{p}{(}\PY{n+nb}{len}\PY{p}{(}\PY{n}{feature\PYZus{}out\PYZus{}all}\PY{p}{[}\PY{n}{key}\PY{p}{]}\PY{p}{)} \PY{o}{\PYZgt{}} \PY{l+m+mi}{1}\PY{p}{)}\PY{p}{:}
                 \PY{n}{outliers}\PY{o}{.}\PY{n}{append}\PY{p}{(}\PY{n}{key}\PY{p}{)}
             
         \PY{n}{outliers}\PY{o}{.}\PY{n}{sort}\PY{p}{(}\PY{p}{)}
         \PY{n}{all\PYZus{}outliers} \PY{o}{=} \PY{n+nb}{list}\PY{p}{(}\PY{n}{all\PYZus{}outliers}\PY{p}{)}
         \PY{n}{all\PYZus{}outliers}\PY{o}{.}\PY{n}{sort}\PY{p}{(}\PY{p}{)}
         \PY{c+c1}{\PYZsh{} Remove the outliers, if any were specified}
         \PY{n}{good\PYZus{}data} \PY{o}{=} \PY{n}{log\PYZus{}data}\PY{o}{.}\PY{n}{drop}\PY{p}{(}\PY{n}{log\PYZus{}data}\PY{o}{.}\PY{n}{index}\PY{p}{[}\PY{n}{all\PYZus{}outliers}\PY{p}{]}\PY{p}{)}\PY{o}{.}\PY{n}{reset\PYZus{}index}\PY{p}{(}\PY{n}{drop} \PY{o}{=} \PY{k+kc}{True}\PY{p}{)}
\end{Verbatim}


    \begin{Verbatim}[commandchars=\\\{\}]
Data points considered outliers for the feature 'Fresh':

    \end{Verbatim}

    
    \begin{verbatim}
        Fresh       Milk    Grocery    Frozen  Detergents_Paper  Delicatessen
65   4.442651   9.950323  10.732651  3.583519         10.095388      7.260523
66   2.197225   7.335634   8.911530  5.164786          8.151333      3.295837
81   5.389072   9.163249   9.575192  5.645447          8.964184      5.049856
95   1.098612   7.979339   8.740657  6.086775          5.407172      6.563856
96   3.135494   7.869402   9.001839  4.976734          8.262043      5.379897
128  4.941642   9.087834   8.248791  4.955827          6.967909      1.098612
171  5.298317  10.160530   9.894245  6.478510          9.079434      8.740337
193  5.192957   8.156223   9.917982  6.865891          8.633731      6.501290
218  2.890372   8.923191   9.629380  7.158514          8.475746      8.759669
304  5.081404   8.917311  10.117510  6.424869          9.374413      7.787382
305  5.493061   9.468001   9.088399  6.683361          8.271037      5.351858
338  1.098612   5.808142   8.856661  9.655090          2.708050      6.309918
353  4.762174   8.742574   9.961898  5.429346          9.069007      7.013016
355  5.247024   6.588926   7.606885  5.501258          5.214936      4.844187
357  3.610918   7.150701  10.011086  4.919981          8.816853      4.700480
412  4.574711   8.190077   9.425452  4.584967          7.996317      4.127134
    \end{verbatim}

    
    \begin{Verbatim}[commandchars=\\\{\}]
Data points considered outliers for the feature 'Milk':

    \end{Verbatim}

    
    \begin{verbatim}
         Fresh       Milk    Grocery    Frozen  Detergents_Paper  Delicatessen
86   10.039983  11.205013  10.377047  6.894670          9.906981      6.805723
98    6.220590   4.718499   6.656727  6.796824          4.025352      4.882802
154   6.432940   4.007333   4.919981  4.317488          1.945910      2.079442
356  10.029503   4.897840   5.384495  8.057377          2.197225      6.306275
    \end{verbatim}

    
    \begin{Verbatim}[commandchars=\\\{\}]
Data points considered outliers for the feature 'Grocery':

    \end{Verbatim}

    
    \begin{verbatim}
        Fresh      Milk   Grocery    Frozen  Detergents_Paper  Delicatessen
75   9.923192  7.036148  1.098612  8.390949          1.098612      6.882437
154  6.432940  4.007333  4.919981  4.317488          1.945910      2.079442
    \end{verbatim}

    
    \begin{Verbatim}[commandchars=\\\{\}]
Data points considered outliers for the feature 'Frozen':

    \end{Verbatim}

    
    \begin{verbatim}
         Fresh      Milk    Grocery     Frozen  Detergents_Paper  Delicatessen
38    8.431853  9.663261   9.723703   3.496508          8.847360      6.070738
57    8.597297  9.203618   9.257892   3.637586          8.932213      7.156177
65    4.442651  9.950323  10.732651   3.583519         10.095388      7.260523
145  10.000569  9.034080  10.457143   3.737670          9.440738      8.396155
175   7.759187  8.967632   9.382106   3.951244          8.341887      7.436617
264   6.978214  9.177714   9.645041   4.110874          8.696176      7.142827
325  10.395650  9.728181   9.519735  11.016479          7.148346      8.632128
420   8.402007  8.569026   9.490015   3.218876          8.827321      7.239215
429   9.060331  7.467371   8.183118   3.850148          4.430817      7.824446
439   7.932721  7.437206   7.828038   4.174387          6.167516      3.951244
    \end{verbatim}

    
    \begin{Verbatim}[commandchars=\\\{\}]
Data points considered outliers for the feature 'Detergents\_Paper':

    \end{Verbatim}

    
    \begin{verbatim}
        Fresh      Milk   Grocery    Frozen  Detergents_Paper  Delicatessen
75   9.923192  7.036148  1.098612  8.390949          1.098612      6.882437
161  9.428190  6.291569  5.645447  6.995766          1.098612      7.711101
    \end{verbatim}

    
    \begin{Verbatim}[commandchars=\\\{\}]
Data points considered outliers for the feature 'Delicatessen':

    \end{Verbatim}

    
    \begin{verbatim}
         Fresh       Milk    Grocery     Frozen  Detergents_Paper  \
66    2.197225   7.335634   8.911530   5.164786          8.151333   
109   7.248504   9.724899  10.274568   6.511745          6.728629   
128   4.941642   9.087834   8.248791   4.955827          6.967909   
137   8.034955   8.997147   9.021840   6.493754          6.580639   
142  10.519646   8.875147   9.018332   8.004700          2.995732   
154   6.432940   4.007333   4.919981   4.317488          1.945910   
183  10.514529  10.690808   9.911952  10.505999          5.476464   
184   5.789960   6.822197   8.457443   4.304065          5.811141   
187   7.798933   8.987447   9.192075   8.743372          8.148735   
203   6.368187   6.529419   7.703459   6.150603          6.860664   
233   6.871091   8.513988   8.106515   6.842683          6.013715   
285  10.602965   6.461468   8.188689   6.948897          6.077642   
289  10.663966   5.655992   6.154858   7.235619          3.465736   
343   7.431892   8.848509  10.177932   7.283448          9.646593   

     Delicatessen  
66       3.295837  
109      1.098612  
128      1.098612  
137      3.583519  
142      1.098612  
154      2.079442  
183     10.777768  
184      2.397895  
187      1.098612  
203      2.890372  
233      1.945910  
285      2.890372  
289      3.091042  
343      3.610918  
    \end{verbatim}

    
    \begin{Verbatim}[commandchars=\\\{\}]
{\color{incolor}In [{\color{incolor}14}]:} \PY{k}{for} \PY{n}{key} \PY{o+ow}{in} \PY{n}{feature\PYZus{}out\PYZus{}all}\PY{p}{:}
             \PY{k}{if}\PY{p}{(}\PY{n+nb}{len}\PY{p}{(}\PY{n}{feature\PYZus{}out\PYZus{}all}\PY{p}{[}\PY{n}{key}\PY{p}{]}\PY{p}{)} \PY{o}{\PYZgt{}} \PY{l+m+mi}{1}\PY{p}{)}\PY{p}{:}
                 \PY{n+nb}{print}\PY{p}{(}\PY{l+s+s2}{\PYZdq{}}\PY{l+s+s2}{The point with index }\PY{l+s+s2}{\PYZdq{}}\PY{p}{,}\PY{n}{key}\PY{p}{,}\PY{l+s+s2}{\PYZdq{}}\PY{l+s+s2}{ occurs as outlier in features }\PY{l+s+s2}{\PYZdq{}}\PY{p}{,}\PY{n}{feature\PYZus{}out\PYZus{}all}\PY{p}{[}\PY{n}{key}\PY{p}{]}\PY{p}{)}
\end{Verbatim}


    \begin{Verbatim}[commandchars=\\\{\}]
The point with index  65  occurs as outlier in features  ['Fresh', 'Frozen']
The point with index  66  occurs as outlier in features  ['Fresh', 'Delicatessen']
The point with index  128  occurs as outlier in features  ['Fresh', 'Delicatessen']
The point with index  154  occurs as outlier in features  ['Milk', 'Grocery', 'Delicatessen']
The point with index  75  occurs as outlier in features  ['Grocery', 'Detergents\_Paper']

    \end{Verbatim}

    \subsubsection{Question}\label{question}

\begin{itemize}
\tightlist
\item
  Are there any data points considered outliers for more than one
  feature based on the definition above?
\item
  Should these data points be removed from the dataset?
\end{itemize}

    \textbf{Answer:} * Yes there are data points considered outliers for
more than one feature and they are displayed in the above code. * Yes
these data points should be removed from the dataset because they posses
characteristics which is different from bulk of the data and this
unusual characteristic can affect our prediction to a great extent. *
Data points are added to the outliers list because these data points
posses certain characteristics which the bulk doesn't posses. - For
instance consider a class of 10 students and we are supposed to find the
average intelligence of this class. We aim at doing this by conducting a
test. Consider the following table which shows the exam scores, scored
out of 100 for 10 students in the previously mentioned test.We will make
the assumption that all the students are sincere, taught by the same
teacher and have a high IQ which was measured upon their admission owing
to the highly selective nature of this imaginary class.

\begin{verbatim}
- As shown in the table and because of the assumptions made, since maximum students have scored good grades and since they all were sincere, taught by the same teacher and had high IQs, everyone should have got good grades. However students with id `4` and `6` scored very less due to unforseen circumstances. And if we consider all the students including these two for measuring the intelligence of the class, the class comes out to be an average class with a the class_average of 79. However we know that students with id `4` and `6` are quite intelligent but due to some reason they couldn't perform well. Thus considering this fact if we consider students `4` and `6` as outliers, we will get a class_avg of 97.875 which is pretty good, qualifies as a highly intelligent class and is representative of the entire population of the class.
    
    
\end{verbatim}

\begin{longtable}[]{@{}cc@{}}
\toprule
Student\_id & exam\_score\tabularnewline
\midrule
\endhead
1 & 97\tabularnewline
2 & 99\tabularnewline
3 & 94\tabularnewline
4 & 5\tabularnewline
5 & 98\tabularnewline
6 & 2\tabularnewline
7 & 96\tabularnewline
8 & 100\tabularnewline
9 & 100\tabularnewline
10 & 99\tabularnewline
Class\_avg without removing outliers & 79\tabularnewline
Class\_avg after removing outliers & 97.875\tabularnewline
\bottomrule
\end{longtable}

\begin{itemize}
\tightlist
\item
  Thus all in all outliers are removed so that they do not have adverse
  effects on our prediction and these outliers are on the two left and
  right edges of the data which is \texttt{1.5\ *\ IQR} units away from
  the middle 50\% data. This convention of chosing a data which is
  \texttt{1.5\ *\ IQR} times away helps to ensure the fact that no
  important data is lost and only the data which is the oddest is
  removed.
\end{itemize}

    \subsection{Feature Transformation}\label{feature-transformation}

In this section we will use principal component analysis (PCA) to draw
conclusions about the underlying structure of the wholesale customer
data. Since using PCA on a dataset calculates the dimensions which best
maximize variance, we will find which compound combinations of features
best describe customers.

    \subsubsection{Implementation: PCA}\label{implementation-pca}

Now that the data has been scaled to a more normal distribution and has
had any necessary outliers removed, we can now apply PCA to the
\texttt{good\_data} to discover which dimensions about the data best
maximize the variance of features involved. In addition to finding these
dimensions, PCA will also report the \emph{explained variance ratio} of
each dimension --- how much variance within the data is explained by
that dimension alone. Note that a component (dimension) from PCA can be
considered a new "feature" of the space, however it is a composition of
the original features present in the data.

    \begin{Verbatim}[commandchars=\\\{\}]
{\color{incolor}In [{\color{incolor}15}]:} \PY{k+kn}{from} \PY{n+nn}{sklearn}\PY{n+nn}{.}\PY{n+nn}{decomposition} \PY{k}{import} \PY{n}{PCA}
         \PY{c+c1}{\PYZsh{} TODO: Apply PCA by fitting the good data with the same number of dimensions as features}
         \PY{n}{pca} \PY{o}{=} \PY{n}{PCA}\PY{p}{(}\PY{n}{n\PYZus{}components}\PY{o}{=}\PY{l+m+mi}{6}\PY{p}{)}
         \PY{n}{pca}\PY{o}{.}\PY{n}{fit}\PY{p}{(}\PY{n}{good\PYZus{}data}\PY{p}{)}
         
         \PY{c+c1}{\PYZsh{} TODO: Transform the sample log\PYZhy{}data using the PCA fit above}
         \PY{n}{pca\PYZus{}samples} \PY{o}{=} \PY{n}{pca}\PY{o}{.}\PY{n}{transform}\PY{p}{(}\PY{n}{log\PYZus{}samples}\PY{p}{)}
         
         \PY{c+c1}{\PYZsh{} Generate PCA results plot}
         \PY{n}{pca\PYZus{}results} \PY{o}{=} \PY{n}{vs}\PY{o}{.}\PY{n}{pca\PYZus{}results}\PY{p}{(}\PY{n}{good\PYZus{}data}\PY{p}{,} \PY{n}{pca}\PY{p}{)}
\end{Verbatim}


    \begin{center}
    \adjustimage{max size={0.9\linewidth}{0.9\paperheight}}{output_33_0.png}
    \end{center}
    { \hspace*{\fill} \\}
    
    \subsubsection{Question}\label{question}

\begin{itemize}
\tightlist
\item
  How much variance in the data is explained* \textbf{in total} *by the
  first and second principal component?
\item
  How much variance in the data is explained by the first four principal
  components?
\item
  Using the visualization provided above,we will talk about each
  dimension and the cumulative variance explained by each, stressing
  upon which features are well represented by each dimension(both in
  terms of positive and negative variance explained).We will discuss
  what the first four dimensions best represent in terms of customer
  spending.
\end{itemize}

\textbf{Note:} A positive increase in a specific dimension corresponds
with an \emph{increase} of the \emph{positive-weighted} features and a
\emph{decrease} of the \emph{negative-weighted} features. The rate of
increase or decrease is based on the individual feature weights.

    \textbf{Answer:}

\begin{itemize}
\item
  The total variance in the data explained by first and second principal
  component is:

  \begin{itemize}
  \tightlist
  \item
    0.4993+0.2259 = 0.7252 (72.52\%)
  \end{itemize}
\item
  The total variance in the data explained by first four principal
  component is:

  \begin{itemize}
  \tightlist
  \item
    0.4993+0.2259+0.1049+0.0978 = 0.9279 (92.79\%)
  \end{itemize}
\item
  If we look closely at the graph above we can see that there is an
  explained variance for each dimension generated by the PCA. This
  explained variance tells us how much information is retained in a
  particular dimension when we change the basis of our original space.
  The feature weights for a particular dimension gives us insight about
  how much a customer belonging to that particular dimension will spend
  on a particular product. Due to all these reasons we can conclude that
  explained variance and feature weights together are describing the
  spending patterns of customer rather than representing the customers.

  \begin{itemize}
  \item
    Looking at \textbf{dimension 1} which accounts for \textbf{49.93\%}
    of the entire variance or \textbf{49.93\%} of the spending pattern,
    we can see that \texttt{Milk}, \texttt{Grocery},
    \texttt{Detergents\_Paper} and \texttt{Delicatessen} have feature
    weights that are positive while all the other features have weights
    that are negative.

    \begin{itemize}
    \tightlist
    \item
      The positive feature weights of \texttt{Milk}, \texttt{Grocery},
      \texttt{Detergents\_Paper} and \texttt{Delicatessen} mean that
      customer that are concenterated in dimension 1 will be spending
      more on these category of products. Maximum spending will be done
      for \texttt{Detergents\_Paper} as the feature weight corresponding
      to it is the maximum and in the same way the second highest
      spending for customers clustered near dimension 1 would be for
      feature \texttt{Grocery} as the feature weight corresponding to it
      is second highest , thierd would be \texttt{Milk} and fourth would
      be \texttt{Delicatessen}.
    \item
      The negative feature weights of all the other features indicate
      that the customers in dimension 1 will avoid buying products from
      \texttt{Fresh} and \texttt{Frozen}.\\
    \item
      \emph{Thus we can generate a heurestic that whatever the dimension
      is the pattern of customer spending will be such that spending on
      a particular category of product is directly proportional to the
      feature weights corresponding to that feature. Thus lower the
      feature weight of a category lower is the spending on that
      category and if the feature weights are negative, the customer
      avoids spending on such product category alltogether and if there
      is any its very less.}
    \end{itemize}
  \end{itemize}
\item
  \textbf{Dimension 2} ranks second in accounting for the explained
  variance and thus the spending pattern with a value of
  \textbf{22.59\%}.

  \begin{itemize}
  \tightlist
  \item
    In this dimension all the feature weights are positive except
    \texttt{Detergents\_Paper}. Thus spending will in descending order
    as follows:
    \texttt{Frozen},\texttt{Fresh},\texttt{Delicatessen},\texttt{Milk}
    and the lest spending would on \texttt{Grocery}
  \item
    The feature with negative feature weight \texttt{Detergents\_Paper}
    would be neglected alltogether by the customers clustered near this
    dimension.
  \end{itemize}
\item
  \textbf{Dimension 3} comes next with an explained ratio of
  \textbf{10.49\%}.

  \begin{itemize}
  \tightlist
  \item
    As we can see from the figure above \texttt{Delicatessen},
    \texttt{Frozen} and \texttt{Milk} have positive feature weights with
    their values descending in the above mentioned order. Thus customers
    concenterated near this dimension will be spending the maximum in
    the three features mentioned above. Eventhough the feature weight of
    milk is positive, its value is very less and hence the spending on
    milk wont be very high, but ther will still be some spending for
    \texttt{Milk}.
  \item
    All the other categories have negative feature weights and hence the
    customers in this dimension avoid buying these categories with
    negative feature weights.
  \end{itemize}
\item
  The fourth dimension \textbf{Dimension 4} accounts for \textbf{9.78\%}
  of the spending pattern.

  \begin{itemize}
  \tightlist
  \item
    The customers in this dimension spend the highest in
    \texttt{Delicatessen}, \texttt{Fresh}, \texttt{Milk} and
    \texttt{Grocery} prodcuts with their values descending in above
    mentioned order. \texttt{Milk} and \texttt{Grocery} with positive
    but low feature weights will have a low spending from the customers
    in this dimension.
  \item
    Finally the categories with negative feature weights which are
    \texttt{Frozen} and \texttt{Detergents\_Paper} will be avoided by
    the customers all together.
  \end{itemize}
\item
  It is to be noted that the spending patterns are unique to each
  dimenion. The spending patterns in one dimension will not repeat
  itself in other dimension.
\end{itemize}

    \subsubsection{Observation}\label{observation}

Run the code below to see how the log-transformed sample data has
changed after having a PCA transformation applied to it in six
dimensions. Observe the numerical value for the first four dimensions of
the sample points.

    \begin{Verbatim}[commandchars=\\\{\}]
{\color{incolor}In [{\color{incolor}16}]:} \PY{c+c1}{\PYZsh{} Display sample log\PYZhy{}data after having a PCA transformation applied}
         
         \PY{n+nb}{print}\PY{p}{(}\PY{l+s+s1}{\PYZsq{}}\PY{l+s+s1}{PCA transformed dimension of three samples}\PY{l+s+s1}{\PYZsq{}}\PY{p}{)}
         \PY{n}{display}\PY{p}{(}\PY{n}{pd}\PY{o}{.}\PY{n}{DataFrame}\PY{p}{(}\PY{n}{np}\PY{o}{.}\PY{n}{round}\PY{p}{(}\PY{n}{pca\PYZus{}samples}\PY{p}{,} \PY{l+m+mi}{4}\PY{p}{)}\PY{p}{,} \PY{n}{columns} \PY{o}{=} \PY{n}{pca\PYZus{}results}\PY{o}{.}\PY{n}{index}\PY{o}{.}\PY{n}{values}\PY{p}{)}\PY{p}{)}
\end{Verbatim}


    \begin{Verbatim}[commandchars=\\\{\}]
PCA transformed dimension of three samples

    \end{Verbatim}

    
    \begin{verbatim}
   Dimension 1  Dimension 2  Dimension 3  Dimension 4  Dimension 5  \
0      -0.9986       1.3694       0.2854      -0.3997      -0.6781   
1       0.9642       3.2453       0.4714       0.9451      -0.8370   
2       3.0820       0.1314       0.3994      -1.4197       0.4747   

   Dimension 6  
0       0.6194  
1       0.0961  
2       0.2263  
    \end{verbatim}

    
    \subsubsection{Implementation: Dimensionality
Reduction}\label{implementation-dimensionality-reduction}

When using principal component analysis, one of the main goals is to
reduce the dimensionality of the data --- in effect, reducing the
complexity of the problem. Dimensionality reduction comes at a cost:
Fewer dimensions used implies less of the total variance in the data
being explained. Because of this, the \emph{cumulative explained
variance ratio} is extremely important for knowing how many dimensions
are necessary for the problem. Additionally, if a signifiant amount of
variance is explained by only two or three dimensions, the reduced data
can be visualized afterwards.

    \begin{Verbatim}[commandchars=\\\{\}]
{\color{incolor}In [{\color{incolor}17}]:} \PY{c+c1}{\PYZsh{} TODO: Apply PCA by fitting the good data with only two dimensions}
         \PY{n}{pca} \PY{o}{=} \PY{n}{PCA}\PY{p}{(}\PY{n}{n\PYZus{}components}\PY{o}{=}\PY{l+m+mi}{2}\PY{p}{)}
         \PY{n}{pca}\PY{o}{.}\PY{n}{fit}\PY{p}{(}\PY{n}{good\PYZus{}data}\PY{p}{)}
         
         \PY{c+c1}{\PYZsh{} TODO: Transform the good data using the PCA fit above}
         \PY{n}{reduced\PYZus{}data} \PY{o}{=} \PY{n}{pca}\PY{o}{.}\PY{n}{transform}\PY{p}{(}\PY{n}{good\PYZus{}data}\PY{p}{)}
         
         \PY{c+c1}{\PYZsh{} TODO: Transform log\PYZus{}samples using the PCA fit above}
         \PY{n}{pca\PYZus{}samples} \PY{o}{=} \PY{n}{pca}\PY{o}{.}\PY{n}{transform}\PY{p}{(}\PY{n}{log\PYZus{}samples}\PY{p}{)}
         
         \PY{c+c1}{\PYZsh{} Create a DataFrame for the reduced data}
         \PY{n}{reduced\PYZus{}data} \PY{o}{=} \PY{n}{pd}\PY{o}{.}\PY{n}{DataFrame}\PY{p}{(}\PY{n}{reduced\PYZus{}data}\PY{p}{,} \PY{n}{columns} \PY{o}{=} \PY{p}{[}\PY{l+s+s1}{\PYZsq{}}\PY{l+s+s1}{Dimension 1}\PY{l+s+s1}{\PYZsq{}}\PY{p}{,} \PY{l+s+s1}{\PYZsq{}}\PY{l+s+s1}{Dimension 2}\PY{l+s+s1}{\PYZsq{}}\PY{p}{]}\PY{p}{)}
\end{Verbatim}


    \subsubsection{Observation}\label{observation}

Run the code below to see how the log-transformed sample data has
changed after having a PCA transformation applied to it using only two
dimensions. Observe how the values for the first two dimensions remains
unchanged when compared to a PCA transformation in six dimensions.

    \begin{Verbatim}[commandchars=\\\{\}]
{\color{incolor}In [{\color{incolor}18}]:} \PY{c+c1}{\PYZsh{} Display sample log\PYZhy{}data after applying PCA transformation in two dimensions}
         \PY{n}{display}\PY{p}{(}\PY{n}{pd}\PY{o}{.}\PY{n}{DataFrame}\PY{p}{(}\PY{n}{np}\PY{o}{.}\PY{n}{round}\PY{p}{(}\PY{n}{pca\PYZus{}samples}\PY{p}{,} \PY{l+m+mi}{4}\PY{p}{)}\PY{p}{,} \PY{n}{columns} \PY{o}{=} \PY{p}{[}\PY{l+s+s1}{\PYZsq{}}\PY{l+s+s1}{Dimension 1}\PY{l+s+s1}{\PYZsq{}}\PY{p}{,} \PY{l+s+s1}{\PYZsq{}}\PY{l+s+s1}{Dimension 2}\PY{l+s+s1}{\PYZsq{}}\PY{p}{]}\PY{p}{)}\PY{p}{)}
\end{Verbatim}


    
    \begin{verbatim}
   Dimension 1  Dimension 2
0      -0.9986       1.3694
1       0.9642       3.2453
2       3.0820       0.1314
    \end{verbatim}

    
    \subsection{Visualizing a Biplot}\label{visualizing-a-biplot}

A biplot is a scatterplot where each data point is represented by its
scores along the principal components. The axes are the principal
components (in this case \texttt{Dimension\ 1} and
\texttt{Dimension\ 2}). In addition, the biplot shows the projection of
the original features along the components. A biplot can help us
interpret the reduced dimensions of the data, and discover relationships
between the principal components and original features.

Run the code cell below to produce a biplot of the reduced-dimension
data.

    \begin{Verbatim}[commandchars=\\\{\}]
{\color{incolor}In [{\color{incolor}19}]:} \PY{c+c1}{\PYZsh{} Create a biplot}
         \PY{n}{vs}\PY{o}{.}\PY{n}{biplot}\PY{p}{(}\PY{n}{good\PYZus{}data}\PY{p}{,} \PY{n}{reduced\PYZus{}data}\PY{p}{,} \PY{n}{pca}\PY{p}{)}
\end{Verbatim}


\begin{Verbatim}[commandchars=\\\{\}]
{\color{outcolor}Out[{\color{outcolor}19}]:} <matplotlib.axes.\_subplots.AxesSubplot at 0x7fc1c0a21358>
\end{Verbatim}
            
    \begin{center}
    \adjustimage{max size={0.9\linewidth}{0.9\paperheight}}{output_43_1.png}
    \end{center}
    { \hspace*{\fill} \\}
    
    \subsubsection{Observation}\label{observation}

Once we have the original feature projections (in red), it is easier to
interpret the relative position of each data point in the scatterplot.
For instance, a point the lower right corner of the figure will likely
correspond to a customer that spends a lot on
\texttt{\textquotesingle{}Milk\textquotesingle{}},
\texttt{\textquotesingle{}Grocery\textquotesingle{}} and
\texttt{\textquotesingle{}Detergents\_Paper\textquotesingle{}}, but not
so much on the other product categories.

From the biplot, which of the original features are most strongly
correlated with the first component? What about those that are
associated with the second component? Do these observations agree with
the pca\_results plot you obtained earlier?

\begin{itemize}
\tightlist
\item
  Since we have plotte the original feature projections in red on the
  biplot, we can say that a feature is maximally correlated(+1) with a
  particular dimension when it is parallel to the dimension or is in the
  direction of the increase of that dimension. This is because with the
  increase in the value of dimension the feature pointing in that
  direction will also increase therby pointing to a positive
  correlation.
\item
  In the same way if the feature is orthogonal to a particular dimension
  then the correlation is zero because no matter the direction of
  increase or decrease of that dimension the feature value corresponding
  to it will remain constsnt.
\item
  Finally if the feature is pointing in the direction opposite(180
  degrees) to the increase of a particular dimension then we can say
  that the particular feature is negatively correlated with the
  dimension in consideration and has minimum correlation(-1).
\end{itemize}

We can see from the above biplot and the original feature projections in
red that there are three features in maximum correlation with dimension
1 with the following order of decreasing correlation: -
Detergents\_Paper(max\_correlation with dimension 1) - Grocery (second
best correlation with dimension 1) - Milk (Third best correlation with
dimension 1)

In the same way the three features with maximum correlation with
dimension 2 with their are as follows: - Frozen(max\_correlation with
dimension 2) - Fresh (second best correlation with dimension 2) -
delicatessen (Third best correlation with dimension 2)

\begin{itemize}
\tightlist
\item
  The above mentioned points above correlation of features with
  \texttt{dimension\ 1} and \texttt{dimension\ 2} is proved by the PCA
  plot above where the feature weights of the respective features in
  \texttt{dimension\ 1} and \texttt{dimension\ 2} correspond to the
  correlations mentioned in points above.
\end{itemize}

    \subsection{Clustering}\label{clustering}

In this section, we will choose to use either a K-Means clustering
algorithm or a Gaussian Mixture Model clustering algorithm to identify
the various customer segments hidden in the data. We will then recover
specific data points from the clusters to understand their significance
by transforming them back into their original dimension and scale.

    \begin{itemize}
\tightlist
\item
  HARD AND SOFT CLUSTERING:

  \begin{itemize}
  \tightlist
  \item
    Hard Clustering: Hard clustering is a method where every point in
    the data is assigned to a particular cluster center. Morever a data
    point can belong to one and only one cluster center.
  \item
    Soft Clustering: Unlike hard clustering where all data points belong
    to unique centers, the soft clustering method assign clusters on the
    basis of probability. This means that a data point will belong to
    all the cluster centers with some probability. After this depending
    on the probability of a data point being in a particular cluster we
    tag it with that cluster in which the probability of the data point
    being in that cluster is maximum. However this does not mean that
    the point into consideration does not belong to other cluster
    centers, it always has some probability associated with it however
    small it is.
  \end{itemize}
\item
  K-Means: This method of clustering is a hard clustering method. This
  means that every data point when clustered using a K-means will be
  assigned a unique cluster center. However the assignment depends
  highly on the initial selection of cluster centers. This is because
  K-means selects its initial centers randomly and after that it does
  the following steps:

  \begin{itemize}
  \tightlist
  \item
    It tries to reduce the squared distance between the cluster centers
    and the surrounding points and assigns new coordinates to the
    centers(OPTIMIZATION step).
  \item
    After this the data points are assigned to these new
    clusters(ASSIGNMENT step).
  \end{itemize}
\end{itemize}

These two steps optimization and assignment are carried out again and
again till there is no change in the coordinates of the cluster center.
This is when we are done with clustering with K-means and we finally
have our two clusters. However due to the initial random selection of
cluster centers, if the data is spread out i.e there is no clear
boundaries between the two clusters, K-means clustering will give us a
different result everytime. These results may be what we want or they
may not be what we expected. Emperical results suggests that K-means
tends to work when we have our data set which is separated in such a way
that, not only the clusters have a defining separating boundary, but
also when the clusters are in the shape of circular blobs. Even when we
have such conditions for our data it is not guranteed that K-means will
cluster properly. The clustering quality depends very much on the
initial selection of the cluster centers. Thus since our data is highly
spread K-means clustering will not be a good choice because of the
following reasons: - When we look at our dataset we can see that K-means
will not make sense as there is no separating boundary. And as mentioned
earlier K-means will not work properly when there isn't a proper
separating boundary. - K-means is a hard clustering method and it will
try to assign every point to a unique cluster. And since our data points
are too spread out we dont want to be too harsh on assigning the points
to a particluar center. Instead we might try to assign every point in
the data to every cluster center with some probability which is nothing
but soft clustering. The reason for going with soft clustering is that
since the data is too spread out, we are uncertain about a particular
point belonging to a specific category. Under such uncertain
circumstances, what better way to define a datas' category with
probability and likelihood.

\begin{itemize}
\tightlist
\item
  Gaussian-Mixture Model: As the name suggests the Gaussian Mixture
  model uses multivariate normal probability distribution to cluster
  data. Depending on the number of clusters chosen initially this model
  assigns a specific probability to all the data points and this
  probability suggests how much a data point belongs to a specific
  cluster. A familiarity with the normal model suggests that the
  probability distribution never touches 0 instead it reduces
  exponentially, reaches very close to zero and touches the zero
  probability point only at infinity. Due to this fact there is never a
  zero probability of a data point being in any one of the cluster.
  Instead there is only a relative probability that suggests the data
  point being more or less in a particular cluster. This points to the
  fact that Gaussian-Mixture Model clustering is a soft clustering
  method.

  \begin{itemize}
  \tightlist
  \item
    Thus since our data is highly spread out we can think of using a
    gaussina clustering model and check its validity by using the
    silhouette score for different numbers of clusters.
  \end{itemize}
\item
  WE FINALLY CHOOSE THE GUSAASIAN MIXTURE MODEL CLUSTERING BECAUSE OF
  THE ABOVE DISCUSSION.
\end{itemize}

    \subsubsection{Implementation: Creating
Clusters}\label{implementation-creating-clusters}

Depending on the problem, the number of clusters that you expect to be
in the data may already be known. When the number of clusters is not
known \emph{a priori}, there is no guarantee that a given number of
clusters best segments the data, since it is unclear what structure
exists in the data --- if any. However, we can quantify the "goodness"
of a clustering by calculating each data point's \emph{silhouette
coefficient}. The
\href{http://scikit-learn.org/stable/modules/generated/sklearn.metrics.silhouette_score.html}{silhouette
coefficient} for a data point measures how similar it is to its assigned
cluster from -1 (dissimilar) to 1 (similar). Calculating the \emph{mean}
silhouette coefficient provides for a simple scoring method of a given
clustering.

    \begin{Verbatim}[commandchars=\\\{\}]
{\color{incolor}In [{\color{incolor}20}]:} \PY{c+c1}{\PYZsh{} TODO: Apply your clustering algorithm of choice to the reduced data}
         \PY{k+kn}{from} \PY{n+nn}{sklearn}\PY{n+nn}{.}\PY{n+nn}{mixture} \PY{k}{import} \PY{n}{GaussianMixture}
         \PY{k+kn}{from} \PY{n+nn}{sklearn}\PY{n+nn}{.}\PY{n+nn}{metrics} \PY{k}{import} \PY{n}{silhouette\PYZus{}score}
         
         \PY{c+c1}{\PYZsh{} Choose the range of k values to test.}
         \PY{n}{possible\PYZus{}n\PYZus{}values} \PY{o}{=} \PY{n+nb}{range}\PY{p}{(}\PY{l+m+mi}{2}\PY{p}{,} \PY{n+nb}{len}\PY{p}{(}\PY{n}{reduced\PYZus{}data}\PY{p}{)}\PY{o}{+}\PY{l+m+mi}{1}\PY{p}{,} \PY{l+m+mi}{5}\PY{p}{)}
         
         \PY{c+c1}{\PYZsh{}Calculating the silhouette\PYZus{}score for every possible value of clusters possible}
         \PY{n}{errors\PYZus{}per\PYZus{}n} \PY{o}{=} \PY{p}{[}\PY{p}{]}
         \PY{k}{for} \PY{n}{n} \PY{o+ow}{in} \PY{n}{possible\PYZus{}n\PYZus{}values}\PY{p}{:}
             \PY{n}{clusterer} \PY{o}{=} \PY{n}{GaussianMixture}\PY{p}{(}\PY{n}{n\PYZus{}components}\PY{o}{=}\PY{n}{n}\PY{p}{)}
             \PY{n}{clusterer}\PY{o}{.}\PY{n}{fit}\PY{p}{(}\PY{n}{reduced\PYZus{}data}\PY{p}{)}
             \PY{n}{preds} \PY{o}{=} \PY{n}{clusterer}\PY{o}{.}\PY{n}{predict}\PY{p}{(}\PY{n}{reduced\PYZus{}data}\PY{p}{)}
             \PY{n}{score} \PY{o}{=} \PY{n}{silhouette\PYZus{}score}\PY{p}{(}\PY{n}{reduced\PYZus{}data}\PY{p}{,}\PY{n}{preds}\PY{p}{)}
             \PY{n}{errors\PYZus{}per\PYZus{}n}\PY{o}{.}\PY{n}{append}\PY{p}{(}\PY{n}{score}\PY{p}{)}
             
         \PY{c+c1}{\PYZsh{} Plot the each value of n vs. the silhouette score at that value}
         \PY{n}{fig}\PY{p}{,} \PY{n}{ax} \PY{o}{=} \PY{n}{plt}\PY{o}{.}\PY{n}{subplots}\PY{p}{(}\PY{n}{figsize}\PY{o}{=}\PY{p}{(}\PY{l+m+mi}{16}\PY{p}{,} \PY{l+m+mi}{6}\PY{p}{)}\PY{p}{)}
         \PY{n}{ax}\PY{o}{.}\PY{n}{set\PYZus{}xlabel}\PY{p}{(}\PY{l+s+s1}{\PYZsq{}}\PY{l+s+s1}{n \PYZhy{} number of clusters}\PY{l+s+s1}{\PYZsq{}}\PY{p}{)}
         \PY{n}{ax}\PY{o}{.}\PY{n}{set\PYZus{}ylabel}\PY{p}{(}\PY{l+s+s1}{\PYZsq{}}\PY{l+s+s1}{Silhouette Score (higher is better)}\PY{l+s+s1}{\PYZsq{}}\PY{p}{)}
         \PY{n}{ax}\PY{o}{.}\PY{n}{plot}\PY{p}{(}\PY{n}{possible\PYZus{}n\PYZus{}values}\PY{p}{,} \PY{n}{errors\PYZus{}per\PYZus{}n}\PY{p}{)}
         
         \PY{c+c1}{\PYZsh{} Ticks and grid}
         \PY{n}{xticks} \PY{o}{=} \PY{n}{np}\PY{o}{.}\PY{n}{arange}\PY{p}{(}\PY{n+nb}{min}\PY{p}{(}\PY{n}{possible\PYZus{}n\PYZus{}values}\PY{p}{)}\PY{p}{,} \PY{n+nb}{max}\PY{p}{(}\PY{n}{possible\PYZus{}n\PYZus{}values}\PY{p}{)}\PY{o}{+}\PY{l+m+mi}{1}\PY{p}{,} \PY{l+m+mf}{20.0}\PY{p}{)}
         \PY{n}{ax}\PY{o}{.}\PY{n}{set\PYZus{}xticks}\PY{p}{(}\PY{n}{xticks}\PY{p}{,} \PY{n}{minor}\PY{o}{=}\PY{k+kc}{False}\PY{p}{)}
         \PY{n}{ax}\PY{o}{.}\PY{n}{set\PYZus{}xticks}\PY{p}{(}\PY{n}{xticks}\PY{p}{,} \PY{n}{minor}\PY{o}{=}\PY{k+kc}{True}\PY{p}{)}
         \PY{n}{ax}\PY{o}{.}\PY{n}{xaxis}\PY{o}{.}\PY{n}{grid}\PY{p}{(}\PY{k+kc}{True}\PY{p}{,} \PY{n}{which}\PY{o}{=}\PY{l+s+s1}{\PYZsq{}}\PY{l+s+s1}{both}\PY{l+s+s1}{\PYZsq{}}\PY{p}{)}
         \PY{n}{yticks} \PY{o}{=} \PY{n}{np}\PY{o}{.}\PY{n}{arange}\PY{p}{(}\PY{n+nb}{round}\PY{p}{(}\PY{n+nb}{min}\PY{p}{(}\PY{n}{errors\PYZus{}per\PYZus{}n}\PY{p}{)}\PY{p}{,} \PY{l+m+mi}{2}\PY{p}{)}\PY{p}{,} \PY{n+nb}{max}\PY{p}{(}\PY{n}{errors\PYZus{}per\PYZus{}n}\PY{p}{)}\PY{p}{,} \PY{o}{.}\PY{l+m+mi}{05}\PY{p}{)}
         \PY{n}{ax}\PY{o}{.}\PY{n}{set\PYZus{}yticks}\PY{p}{(}\PY{n}{yticks}\PY{p}{,} \PY{n}{minor}\PY{o}{=}\PY{k+kc}{False}\PY{p}{)}
         \PY{n}{ax}\PY{o}{.}\PY{n}{set\PYZus{}yticks}\PY{p}{(}\PY{n}{yticks}\PY{p}{,} \PY{n}{minor}\PY{o}{=}\PY{k+kc}{True}\PY{p}{)}
         \PY{n}{ax}\PY{o}{.}\PY{n}{yaxis}\PY{o}{.}\PY{n}{grid}\PY{p}{(}\PY{k+kc}{True}\PY{p}{,} \PY{n}{which}\PY{o}{=}\PY{l+s+s1}{\PYZsq{}}\PY{l+s+s1}{both}\PY{l+s+s1}{\PYZsq{}}\PY{p}{)}
             
             
\end{Verbatim}


    \begin{center}
    \adjustimage{max size={0.9\linewidth}{0.9\paperheight}}{output_48_0.png}
    \end{center}
    { \hspace*{\fill} \\}
    
    \begin{Verbatim}[commandchars=\\\{\}]
{\color{incolor}In [{\color{incolor}21}]:} \PY{n}{clusterer} \PY{o}{=} \PY{n}{GaussianMixture}\PY{p}{(}\PY{n}{n\PYZus{}components}\PY{o}{=}\PY{l+m+mi}{2}\PY{p}{)}
         \PY{n}{clusterer}\PY{o}{.}\PY{n}{fit}\PY{p}{(}\PY{n}{reduced\PYZus{}data}\PY{p}{)}
         \PY{n}{preds} \PY{o}{=} \PY{n}{clusterer}\PY{o}{.}\PY{n}{predict}\PY{p}{(}\PY{n}{reduced\PYZus{}data}\PY{p}{)}
         \PY{n}{score} \PY{o}{=} \PY{n}{silhouette\PYZus{}score}\PY{p}{(}\PY{n}{reduced\PYZus{}data}\PY{p}{,}\PY{n}{preds}\PY{p}{)}
         \PY{n}{centers} \PY{o}{=} \PY{n}{clusterer}\PY{o}{.}\PY{n}{means\PYZus{}}
         \PY{n}{sample\PYZus{}preds} \PY{o}{=} \PY{n}{clusterer}\PY{o}{.}\PY{n}{predict}\PY{p}{(}\PY{n}{pca\PYZus{}samples}\PY{p}{)}
         \PY{n+nb}{print}\PY{p}{(}\PY{l+s+s1}{\PYZsq{}}\PY{l+s+s1}{Highest silhouette score is with 2 cluster centers and its value is: }\PY{l+s+s1}{\PYZsq{}}\PY{p}{,}\PY{n}{score}\PY{p}{)}
\end{Verbatim}


    \begin{Verbatim}[commandchars=\\\{\}]
Highest silhouette score is with 2 cluster centers and its value is:  0.447411995571

    \end{Verbatim}

    \begin{itemize}
\tightlist
\item
  The silhouette scores of all the possible clusters numbers are shown
  in the above graph.
\item
  As we can see from the graph above, the best silhouette score is for
  number of clusters being 2 and its value is 0.447
\end{itemize}

    \subsubsection{Cluster Visualization}\label{cluster-visualization}

Once we've chosen the optimal number of clusters for our clustering
algorithm using the scoring metric above, we can now visualize the
results by executing the code block below.

    \begin{Verbatim}[commandchars=\\\{\}]
{\color{incolor}In [{\color{incolor}22}]:} \PY{c+c1}{\PYZsh{} Display the results of the clustering from implementation}
         \PY{n}{vs}\PY{o}{.}\PY{n}{cluster\PYZus{}results}\PY{p}{(}\PY{n}{reduced\PYZus{}data}\PY{p}{,} \PY{n}{preds}\PY{p}{,} \PY{n}{centers}\PY{p}{,} \PY{n}{pca\PYZus{}samples}\PY{p}{)}
\end{Verbatim}


    \begin{center}
    \adjustimage{max size={0.9\linewidth}{0.9\paperheight}}{output_52_0.png}
    \end{center}
    { \hspace*{\fill} \\}
    
    \subsubsection{Implementation: Data
Recovery}\label{implementation-data-recovery}

Each cluster present in the visualization above has a central point.
These centers (or means) are not specifically data points from the data,
but rather the \emph{averages} of all the data points predicted in the
respective clusters. For the problem of creating customer segments, a
cluster's center point corresponds to \emph{the average customer of that
segment}. Since the data is currently reduced in dimension and scaled by
a logarithm, we can recover the representative customer spending from
these data points by applying the inverse transformations.

    \begin{Verbatim}[commandchars=\\\{\}]
{\color{incolor}In [{\color{incolor}23}]:} \PY{c+c1}{\PYZsh{} TODO: Inverse transform the centers}
         \PY{n}{log\PYZus{}centers} \PY{o}{=} \PY{n}{pca}\PY{o}{.}\PY{n}{inverse\PYZus{}transform}\PY{p}{(}\PY{n}{centers}\PY{p}{)}
         
         \PY{c+c1}{\PYZsh{} TODO: Exponentiate the centers}
         \PY{n}{true\PYZus{}centers} \PY{o}{=} \PY{n}{np}\PY{o}{.}\PY{n}{exp}\PY{p}{(}\PY{n}{log\PYZus{}centers}\PY{p}{)}
         
         \PY{c+c1}{\PYZsh{} Display the true centers}
         \PY{n}{segments} \PY{o}{=} \PY{p}{[}\PY{l+s+s1}{\PYZsq{}}\PY{l+s+s1}{Segment }\PY{l+s+si}{\PYZob{}\PYZcb{}}\PY{l+s+s1}{\PYZsq{}}\PY{o}{.}\PY{n}{format}\PY{p}{(}\PY{n}{i}\PY{p}{)} \PY{k}{for} \PY{n}{i} \PY{o+ow}{in} \PY{n+nb}{range}\PY{p}{(}\PY{l+m+mi}{0}\PY{p}{,}\PY{n+nb}{len}\PY{p}{(}\PY{n}{centers}\PY{p}{)}\PY{p}{)}\PY{p}{]}
         \PY{n}{true\PYZus{}centers} \PY{o}{=} \PY{n}{pd}\PY{o}{.}\PY{n}{DataFrame}\PY{p}{(}\PY{n}{np}\PY{o}{.}\PY{n}{round}\PY{p}{(}\PY{n}{true\PYZus{}centers}\PY{p}{)}\PY{p}{,} \PY{n}{columns} \PY{o}{=} \PY{n}{data}\PY{o}{.}\PY{n}{keys}\PY{p}{(}\PY{p}{)}\PY{p}{)}
         \PY{n}{true\PYZus{}centers\PYZus{}new} \PY{o}{=} \PY{n}{pd}\PY{o}{.}\PY{n}{DataFrame}\PY{p}{(}\PY{n}{np}\PY{o}{.}\PY{n}{round}\PY{p}{(}\PY{n}{true\PYZus{}centers}\PY{p}{)}\PY{p}{,} \PY{n}{columns} \PY{o}{=} \PY{n}{data}\PY{o}{.}\PY{n}{keys}\PY{p}{(}\PY{p}{)}\PY{p}{)}
         \PY{n}{true\PYZus{}centers}\PY{o}{.}\PY{n}{index} \PY{o}{=} \PY{n}{segments}
         \PY{n}{display}\PY{p}{(}\PY{n}{true\PYZus{}centers}\PY{p}{)}
         \PY{n}{true\PYZus{}good\PYZus{}data} \PY{o}{=} \PY{n}{pd}\PY{o}{.}\PY{n}{DataFrame}\PY{p}{(}\PY{n}{np}\PY{o}{.}\PY{n}{exp}\PY{p}{(}\PY{n}{good\PYZus{}data}\PY{p}{)}\PY{p}{)}
         \PY{n}{new\PYZus{}true\PYZus{}good\PYZus{}data} \PY{o}{=} \PY{n}{true\PYZus{}good\PYZus{}data}\PY{o}{.}\PY{n}{append}\PY{p}{(}\PY{n}{true\PYZus{}centers\PYZus{}new}\PY{p}{,}\PY{n}{ignore\PYZus{}index}\PY{o}{=}\PY{k+kc}{True}\PY{p}{)}
         
         \PY{c+c1}{\PYZsh{}Printing the heat map of the percentile values of the averages from Segment 0 and Segment 1 customers.}
         \PY{n}{indices\PYZus{}new} \PY{o}{=} \PY{p}{[}\PY{l+m+mi}{398}\PY{p}{,}\PY{l+m+mi}{399}\PY{p}{]}
         \PY{c+c1}{\PYZsh{}Percentile values of the the sampled data}
         \PY{n}{percentile\PYZus{}values} \PY{o}{=} \PY{l+m+mf}{100.} \PY{o}{*}\PY{n}{new\PYZus{}true\PYZus{}good\PYZus{}data}\PY{o}{.}\PY{n}{rank}\PY{p}{(}\PY{n}{axis}\PY{o}{=}\PY{l+m+mi}{0}\PY{p}{,} \PY{n}{pct}\PY{o}{=}\PY{k+kc}{True}\PY{p}{)}\PY{o}{.}\PY{n}{iloc}\PY{p}{[}\PY{n}{indices\PYZus{}new}\PY{p}{]}\PY{o}{.}\PY{n}{round}\PY{p}{(}\PY{n}{decimals}\PY{o}{=}\PY{l+m+mi}{3}\PY{p}{)}
         \PY{c+c1}{\PYZsh{}heatmap of percentiled value}
         \PY{n}{sns}\PY{o}{.}\PY{n}{heatmap}\PY{p}{(}\PY{n}{data}\PY{o}{=}\PY{n}{percentile\PYZus{}values}\PY{p}{,}\PY{n}{annot}\PY{o}{=}\PY{k+kc}{True}\PY{p}{,}\PY{n}{fmt}\PY{o}{=}\PY{l+s+s1}{\PYZsq{}}\PY{l+s+s1}{.1f}\PY{l+s+s1}{\PYZsq{}}\PY{p}{)}
         \PY{n}{plt}\PY{o}{.}\PY{n}{yticks}\PY{p}{(}\PY{p}{[}\PY{l+m+mf}{0.5}\PY{p}{,}\PY{l+m+mf}{1.5}\PY{p}{,}\PY{l+m+mf}{2.5}\PY{p}{]}\PY{p}{,}\PY{p}{[}\PY{l+s+s1}{\PYZsq{}}\PY{l+s+s1}{Segment 0}\PY{l+s+s1}{\PYZsq{}}\PY{p}{,}\PY{l+s+s1}{\PYZsq{}}\PY{l+s+s1}{Segment 1}\PY{l+s+s1}{\PYZsq{}}\PY{p}{]}\PY{p}{,}\PY{n}{rotation}\PY{o}{=}\PY{l+s+s1}{\PYZsq{}}\PY{l+s+s1}{horizontal}\PY{l+s+s1}{\PYZsq{}}\PY{p}{)}
         \PY{n}{plt}\PY{o}{.}\PY{n}{title}\PY{p}{(}\PY{l+s+s1}{\PYZsq{}}\PY{l+s+s1}{Percentile scores of customers from Segment 0 and 1}\PY{l+s+s1}{\PYZsq{}}\PY{p}{)}
         \PY{n}{plt}\PY{o}{.}\PY{n}{show}\PY{p}{(}\PY{p}{)}
         
         \PY{n}{display}\PY{p}{(}\PY{n}{true\PYZus{}good\PYZus{}data}\PY{o}{.}\PY{n}{describe}\PY{p}{(}\PY{p}{)}\PY{p}{)}
\end{Verbatim}


    
    \begin{verbatim}
            Fresh    Milk  Grocery  Frozen  Detergents_Paper  Delicatessen
Segment 0  9468.0  2067.0   2624.0  2196.0             343.0         799.0
Segment 1  5174.0  7776.0  11581.0  1068.0            4536.0        1101.0
    \end{verbatim}

    
    \begin{center}
    \adjustimage{max size={0.9\linewidth}{0.9\paperheight}}{output_54_1.png}
    \end{center}
    { \hspace*{\fill} \\}
    
    
    \begin{verbatim}
               Fresh          Milk       Grocery        Frozen  \
count     398.000000    398.000000    398.000000    398.000000   
mean    12430.630653   5486.314070   7504.907035   3028.809045   
std     12552.698266   6410.878177   9263.803670   3712.563636   
min       255.000000    201.000000    223.000000     91.000000   
25%      4043.500000   1597.250000   2125.000000    830.000000   
50%      9108.000000   3611.500000   4573.000000   1729.500000   
75%     16969.000000   6802.500000   9762.250000   3745.000000   
max    112151.000000  54259.000000  92780.000000  35009.000000   

       Detergents_Paper  Delicatessen  
count        398.000000     398.00000  
mean        2725.376884    1454.71608  
std         4644.023066    1746.45365  
min            5.000000      46.00000  
25%          263.250000     448.25000  
50%          788.000000     997.50000  
75%         3660.500000    1830.00000  
max        40827.000000   16523.00000  
    \end{verbatim}

    
    \subsubsection{Question}\label{question}

\begin{itemize}
\tightlist
\item
  Considering the total purchase cost of each product category for the
  representative data points above, and referencing the statistical
  description of the dataset at the beginning of this
  project(specifically looking at the mean values for the various
  feature points). We will see what set of establishments could each of
  the customer segments represent?
\end{itemize}

\textbf{Note:} A customer who is assigned to
\texttt{\textquotesingle{}Cluster\ X\textquotesingle{}} should best
identify with the establishments represented by the feature set of
\texttt{\textquotesingle{}Segment\ X\textquotesingle{}}. Think about
what each segment represents in terms their values for the feature
points chosen. Reference these values with the mean values to get some
perspective into what kind of establishment they represent.

    \textbf{Answer:}

\begin{itemize}
\item
  The customers from \texttt{Segment\ 0} have maximum spending in(above
  75th percentile) Detergents\_paper, Grocery and Milk. It also has
  above 50th percentile spending in Delicatessen products. This segment
  of customers likely represents a big cafe where we get everything,
  ranging from milk products like coffee, fast food items and also most
  probably desserts.
\item
  As we can see from the above data the customers in \texttt{Segment\ 1}
  have above average spending in Fresh and Frozen products only. In all
  the other categories, these customers spend below average. It is to be
  noted that these customers spend a bit higher in Delicatessen products
  which is almost 42.5th percentile. Thus we can say that such customers
  are small shop owners whose main business is to serve fruit-vegetables
  and items made from frozen things like meat etc.
\end{itemize}

    \subsubsection{Predicting our sample data
points}\label{predicting-our-sample-data-points}

\begin{itemize}
\tightlist
\item
  For each sample point, which customer segment from* \textbf{previous
  question} *best represents it?
\item
  Are the predictions for each sample point consistent with this?*
\end{itemize}

    \begin{Verbatim}[commandchars=\\\{\}]
{\color{incolor}In [{\color{incolor}24}]:} \PY{c+c1}{\PYZsh{} Display the predictions}
         \PY{k}{for} \PY{n}{i}\PY{p}{,} \PY{n}{pred} \PY{o+ow}{in} \PY{n+nb}{enumerate}\PY{p}{(}\PY{n}{sample\PYZus{}preds}\PY{p}{)}\PY{p}{:}
             \PY{n+nb}{print}\PY{p}{(}\PY{l+s+s2}{\PYZdq{}}\PY{l+s+s2}{Sample point}\PY{l+s+s2}{\PYZdq{}}\PY{p}{,} \PY{n}{i}\PY{p}{,} \PY{l+s+s2}{\PYZdq{}}\PY{l+s+s2}{predicted to be in Cluster}\PY{l+s+s2}{\PYZdq{}}\PY{p}{,} \PY{n}{pred}\PY{p}{)}
             
         \PY{c+c1}{\PYZsh{}Percentile values of the the sampled data}
         \PY{n}{percentile\PYZus{}values} \PY{o}{=} \PY{l+m+mf}{100.} \PY{o}{*}\PY{n}{data}\PY{o}{.}\PY{n}{rank}\PY{p}{(}\PY{n}{axis}\PY{o}{=}\PY{l+m+mi}{0}\PY{p}{,} \PY{n}{pct}\PY{o}{=}\PY{k+kc}{True}\PY{p}{)}\PY{o}{.}\PY{n}{iloc}\PY{p}{[}\PY{n}{indices}\PY{p}{]}\PY{o}{.}\PY{n}{round}\PY{p}{(}\PY{n}{decimals}\PY{o}{=}\PY{l+m+mi}{3}\PY{p}{)}
         \PY{c+c1}{\PYZsh{}heatmap of percentiled value}
         \PY{n}{sns}\PY{o}{.}\PY{n}{heatmap}\PY{p}{(}\PY{n}{data}\PY{o}{=}\PY{n}{percentile\PYZus{}values}\PY{p}{,}\PY{n}{annot}\PY{o}{=}\PY{k+kc}{True}\PY{p}{,}\PY{n}{fmt}\PY{o}{=}\PY{l+s+s1}{\PYZsq{}}\PY{l+s+s1}{.1f}\PY{l+s+s1}{\PYZsq{}}\PY{p}{)}
         \PY{n}{plt}\PY{o}{.}\PY{n}{yticks}\PY{p}{(}\PY{p}{[}\PY{l+m+mf}{0.5}\PY{p}{,}\PY{l+m+mf}{1.5}\PY{p}{,}\PY{l+m+mf}{2.5}\PY{p}{]}\PY{p}{,}\PY{p}{[}\PY{l+s+s1}{\PYZsq{}}\PY{l+s+s1}{Customer 0, Index }\PY{l+s+s1}{\PYZsq{}}\PY{o}{+}\PY{n+nb}{str}\PY{p}{(}\PY{n}{indices}\PY{p}{[}\PY{l+m+mi}{0}\PY{p}{]}\PY{p}{)}\PY{p}{,}\PY{l+s+s1}{\PYZsq{}}\PY{l+s+s1}{Customer 1, Index }\PY{l+s+s1}{\PYZsq{}}\PY{o}{+}\PY{n+nb}{str}\PY{p}{(}\PY{n}{indices}\PY{p}{[}\PY{l+m+mi}{1}\PY{p}{]}\PY{p}{)}\PY{p}{,}\PY{l+s+s1}{\PYZsq{}}\PY{l+s+s1}{Customer 2, Index }\PY{l+s+s1}{\PYZsq{}}\PY{o}{+}\PY{n+nb}{str}\PY{p}{(}\PY{n}{indices}\PY{p}{[}\PY{l+m+mi}{2}\PY{p}{]}\PY{p}{)}\PY{p}{]}\PY{p}{,}\PY{n}{rotation}\PY{o}{=}\PY{l+s+s1}{\PYZsq{}}\PY{l+s+s1}{horizontal}\PY{l+s+s1}{\PYZsq{}}\PY{p}{)}
         \PY{n}{plt}\PY{o}{.}\PY{n}{title}\PY{p}{(}\PY{l+s+s1}{\PYZsq{}}\PY{l+s+s1}{Percentile scores of every value in the sampled data frame}\PY{l+s+s1}{\PYZsq{}}\PY{p}{)}
         
         \PY{n+nb}{print}\PY{p}{(}\PY{l+s+s2}{\PYZdq{}}\PY{l+s+s2}{Chosen samples of wholesale customers dataset:}\PY{l+s+s2}{\PYZdq{}}\PY{p}{)}
         \PY{n}{display}\PY{p}{(}\PY{n}{samples}\PY{p}{)}
\end{Verbatim}


    \begin{Verbatim}[commandchars=\\\{\}]
Sample point 0 predicted to be in Cluster 0
Sample point 1 predicted to be in Cluster 0
Sample point 2 predicted to be in Cluster 1
Chosen samples of wholesale customers dataset:

    \end{Verbatim}

    
    \begin{verbatim}
   Fresh   Milk  Grocery  Frozen  Detergents_Paper  Delicatessen
0  13265   1196     4221    6404               507          1788
1  43265   5025     8117    6312              1579         14351
2   3067  13240    23127    3941              9959           731
    \end{verbatim}

    
    \begin{center}
    \adjustimage{max size={0.9\linewidth}{0.9\paperheight}}{output_58_2.png}
    \end{center}
    { \hspace*{\fill} \\}
    
    \textbf{Answer:}

\begin{longtable}[]{@{}cc@{}}
\toprule
Customer(index) & Segment\tabularnewline
\midrule
\endhead
Customer 0(3) & 1\tabularnewline
Customer 1(87) & 1\tabularnewline
Customer 2(200) & 0\tabularnewline
\bottomrule
\end{longtable}

\begin{itemize}
\tightlist
\item
  The segments of the above samples are consistent with out predictions:

  \begin{itemize}
  \tightlist
  \item
    \texttt{Customer\ 0} and \texttt{Customer\ 1} are clustered in
    \texttt{Cluster\ 1}. This is consistent with our perdicted segment
    because as you can see from the above heatmap, Customer 0 and 1 have
    higher spending in Fresh, Frozen and Delicatessen. All other
    categories have lower spending as compared to the three categories
    mentioned before. This is in sync with our predicted
    \texttt{Segment\ 0} customer where an average customer from this
    segment tends to follow this trend in spending.
  \item
    Customer 2 is clustered as a \texttt{Cluster\ 0} customer. This is
    because \texttt{Customer\ 2} tends to spend the maximum in Milk,
    Grocery and Detergents\_paper which is again in sync with trend of
    spending of what an average customer of \texttt{Segment\ 1} would
    do.
  \end{itemize}
\end{itemize}

    \subsection{Conclusion}\label{conclusion}

    In this final section, we will investigate ways that we can make use of
the clustered data. First, we will consider how the different groups of
customers, the \textbf{\emph{customer segments}}, may be affected
differently by a specific delivery scheme. Next, we will consider how
giving a label to each customer (which \emph{segment} that customer
belongs to) can provide for additional features about the customer data.
Finally, we will compare the \textbf{\emph{customer segments}} to a
hidden variable present in the data, to see whether the clustering
identified certain relationships.

    \subsubsection{Question}\label{question}

Companies will often run
\href{https://en.wikipedia.org/wiki/A/B_testing}{A/B tests} when making
small changes to their products or services to determine whether making
that change will affect its customers positively or negatively. The
wholesale distributor is considering changing its delivery service from
currently 5 days a week to 3 days a week. However, the distributor will
only make this change in delivery service for customers that react
positively.

\begin{itemize}
\tightlist
\item
  How can the wholesale distributor use the customer segments to
  determine which customers, if any, would react positively to the
  change in delivery service?*
\end{itemize}

    \textbf{Answer:}

\begin{itemize}
\tightlist
\item
  The wholesale distributor is considering changing its delivery service
  from currently 5 days a week(i.e) weekdays to 3 days a week. If we
  think about it this service is likely going to affect those customers
  who need dairy(i.e Milk) products. This is because these customers are
  the ones who will want fresh products daily as these are the products
  that tend to spoil earlier then thier counterparts. Thus changing the
  5 day delivery to 3 day delivery for these customers will likely
  affect these customers negatively and hence finally result in the
  wholesale distributor losing its customers.
\item
  On the basis of the discussion above our best guess is that the
  Customers from \texttt{Segment\ 0} who tend to spend more on Milk
  products will be unhappy with this new service while the customers
  from \texttt{Segment\ 1} who spend less on Milk products and more on
  Frozen products will be indifferent to this new service.
\item
  However to get statistically accurate readings we can apply the A/B
  split testing in which the distributor will sample randomly from both
  the segments, assign customers from one of the segment to the new
  delivery schedule while keeping the customers from another segment
  same and after getting the feedback from these customers, the
  distributor can conduct a hypothesis testing to check if its
  assumption was right and carry the same process again but this time
  changing the assignment of delivery schedule.
\end{itemize}

    \subsubsection{Question}\label{question}

Additional structure is derived from originally unlabeled data when
using clustering techniques. Since each customer has a
\textbf{\emph{customer segment}} it best identifies with (depending on
the clustering algorithm applied), we can consider \emph{'customer
segment'} as an \textbf{engineered feature} for the data. Assume the
wholesale distributor recently acquired ten new customers and each
provided estimates for anticipated annual spending of each product
category. Knowing these estimates, the wholesale distributor wants to
classify each new customer to a \textbf{\emph{customer segment}} to
determine the most appropriate delivery service.\\
* How can the wholesale distributor label the new customers using only
their estimated product spending and the \textbf{customer segment} data?

    \textbf{Answer:}

\begin{itemize}
\tightlist
\item
  Since now that we have divided the customers in two segments and found
  out which customers are likely going to be affected by the delivery
  schedule, we can label our original customers as customers from
  \texttt{Segment\ 0} and \texttt{Segment\ 1}.
\item
  In this case the traget variable for our original customers will
  become these Segments.
\item
  Thus, as we have converted our unlabelled data into a labelled data by
  performing unsupervised learning, we can now train a supervised model
  on this new labeled data.
\item
  It is to be noted that now since we have labelled data we can train a
  supervised model on the full data without feature reduction, or we can
  train a model on the reduced data that we obtained after PCA.
\item
  If we try to train a supervised learning model on the reduced data, we
  can simply go with a perceptron classifier, as there is a clear
  separating boundary between the two clusters formed from the GMM
  clustering technique.
\item
  After training this model we can simply predict the Segments of the
  new customers and assign an appropriate delivery schedule to these new
  customers. It is to be noted that the delivery schedule comes from the
  A/B testing.
\end{itemize}

    \subsubsection{Visualizing Underlying
Distributions}\label{visualizing-underlying-distributions}

At the beginning of this project, it was discussed that the
\texttt{\textquotesingle{}Channel\textquotesingle{}} and
\texttt{\textquotesingle{}Region\textquotesingle{}} features would be
excluded from the dataset so that the customer product categories were
emphasized in the analysis. By reintroducing the
\texttt{\textquotesingle{}Channel\textquotesingle{}} feature to the
dataset, an interesting structure emerges when considering the same PCA
dimensionality reduction applied earlier to the original dataset.

Run the code block below to see how each data point is labeled either
\texttt{\textquotesingle{}HoReCa\textquotesingle{}}
(Hotel/Restaurant/Cafe) or
\texttt{\textquotesingle{}Retail\textquotesingle{}} the reduced space.
In addition, we will find the sample points are circled in the plot,
which will identify their labeling.

    \begin{Verbatim}[commandchars=\\\{\}]
{\color{incolor}In [{\color{incolor}26}]:} \PY{c+c1}{\PYZsh{} Display the clustering results based on \PYZsq{}Channel\PYZsq{} data}
         
         \PY{n}{vs}\PY{o}{.}\PY{n}{channel\PYZus{}results}\PY{p}{(}\PY{n}{reduced\PYZus{}data}\PY{p}{,} \PY{n}{all\PYZus{}outliers}\PY{p}{,} \PY{n}{pca\PYZus{}samples}\PY{p}{)}
\end{Verbatim}


    \begin{center}
    \adjustimage{max size={0.9\linewidth}{0.9\paperheight}}{output_67_0.png}
    \end{center}
    { \hspace*{\fill} \\}
    
    \subsubsection{Question}\label{question}

\begin{itemize}
\tightlist
\item
  How well does the clustering algorithm and number of clusters we've
  chosen compare to this underlying distribution of
  Hotel/Restaurant/Cafe customers to Retailer customers?
\item
  Are there customer segments that would be classified as purely
  'Retailers' or 'Hotels/Restaurants/Cafes' by this distribution?
\item
  Would we consider these classifications as consistent with our
  previous definition of the customer segments?
\end{itemize}

    \textbf{Answer:}

\begin{itemize}
\tightlist
\item
  As we can see the clustering algorithm and the number of clusters that
  we chose are almost indentical to this true underlying distribution.
  Thus our model selection is perfect.
\item
  I think there are two answers to this. There are customers that can be
  classified as belonging to true \texttt{Retail} category whereas the
  same is not true for \texttt{Hotels/Restaurants/Cafes}. This is
  visible from the graph above, which shows us that there are a lot of
  data points belonging to \texttt{Hotels/Restaurants/Cafes} category
  which are present in the side where the \texttt{Retail} cluster is,
  however not a lot of \texttt{Retail} category data points are present
  in the cluster of \texttt{Hotels/Restaurants/Cafes}. Now since we have
  a GMM clustering this means that the probability of
  \texttt{Hotels/Restaurants/Cafes} selling retail products is higher as
  compared to the probability of a \texttt{Retail} customer opening a
  hotel/resturant/cafe. This is visible in our day to day life too,
  there are cafes smaller or bigger that tend to sell products to their
  customers however we do not see a lot of vendors selling coffee or
  similar items.
\item
  I would consider these classifications as consistent to our previous
  definition of customer segments. This is because the predictions of
  \texttt{Customer\ 0} and \texttt{Customer\ 1} is \texttt{Segment\ 0}
  which as we specified earlier may belong to category of cafe and that
  of \texttt{Customer\ 2} is \texttt{Segment\ 1} which we specified
  earlier to be some vendor. These assumptions of ours are consistent
  with this true underlying distribution which shows us that
  \texttt{Customer\ 0} and \texttt{Customer\ 1} belong to
  \texttt{Hotel/Restaurant/Cafe} category and \texttt{Customer\ 2}
  belongs tp \texttt{Retail} category.
\end{itemize}


    % Add a bibliography block to the postdoc
    
    
    
    \end{document}
